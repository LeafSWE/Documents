\documentclass[../ClipsManualeUtente.tex]{subfiles}

\begin{document}
\section{Applicazione}
	Nella sezione seguente vengono raccolte tutte le istruzioni necessarie per poter usufruire pienamente di tutte le funzionalità offerte dall'applicazione CLIPS.
	
	\newpage
	\subsection{Preferenze}
		È possibile configurare la navigazione attraverso l'impostazione di determinate preferenze. Tali preferenze impatteranno nel calcolo del percorso nella navigazione.
		Le preferenze disponibili sono:
			\begin{itemize}
				\item ???;
				\item ???.
			\end{itemize}
		
		Per impostare le proprie preferenze:
		\begin{enumerate}
			\item apri il menu dell'applicazione selezionando l'icona $\equiv$;
			\item seleziona \textbf{Preferenze};
			\item seleziona ??? per ???;
			\item seleziona ??? per ???.
		\end{enumerate}

	\begin{framed} 
	\textbf{Nota:} una volta impostate le preferenze, esse saranno attive solo dalla prossima navigazione avviata.
	\end{framed}
	
	%\newpage
	%\subsection{Gestire le mappe}
	
	\newpage
	\subsection{Avviare una navigazione}
		
	\newpage
	\subsection{Area sviluppatore}
		L'applicazione rende disponibile a chi è in possesso della \textbf{password sviluppatore} un'area per poter accedere ai file log creati dall'applicazione durante il suo utilizzo.
		
		\begin{framed}
			\textbf{Nota:} questa funzionalità è creata specificatamente per chi dovrà effettuare dei test sul campo dell'applicazione e raccogliere dati sull'applicazione. Il normale utilizzo dell'applicazione \textbf{non richiede} l'uso di questa funzionalità.
		\end{framed}
		
		Tale materiale ha lo scopo di fornire tutte le informazioni ritenute utili sugli ultimi utilizzi della  funzionalità di navigazione. Questo può essere d'aiuto per espandere il suo potenziale o migliorare la mappatura dell'edificio con i dispositivi beacon.
		
		Per accedere all'area sviluppatore:
		\begin{enumerate}
			\item apri il menu dell'applicazione selezionando l'icona $\equiv$;
			\item seleziona \textbf{Area sviluppatore}; %figure inline 
			\item inserisci la \textbf{password sviluppatore}, ti verrà mostrata la lista dei file log in ordine di data e ora;
			\item seleziona il log di interesse per visualizzarlo; 
		\end{enumerate}

\end{document}