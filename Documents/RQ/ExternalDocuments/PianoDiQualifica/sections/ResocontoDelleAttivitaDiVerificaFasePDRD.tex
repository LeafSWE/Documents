\documentclass[../PianoDiQualifica.tex]{subfiles}

\begin{document}
\begin{appendices}
\section{Resoconto delle attività di verifica - fase PDRD}
All'interno di questa fase\g, secondo quanto riportato nel documento \pianodiprogetto, sono previsti più momenti in cui viene attivato il processo\g\ di verifica. Si è cercato di riportare in questa sezione tutti i risultati che sono stati ottenuti durante questi momenti. Ove fosse necessario, si sono tratte anche delle conclusioni sui risultati ottenuti e su come essi possono essere migliorati.
	
	\subsection{Verifica sui processi}
		\subsubsection{Processo di documentazione}
			\paragraph{Miglioramento costante}
			All'inizio della fase\g\ PDRD il processo\g\ di documentazione si posizionava al livello 3 della scala CMM\g.
			Nonostante l'introduzione delle nuove norme e la ricerca di strumenti per migliorare le modalità di lavoro, l'iniziale apprendimento dell'attività di codifica in gruppo ha comportato disorganizzazione e incomprensioni tra i componenti del gruppo. Il gruppo quindi non è riuscito a raggiungere il livello 4 della scala CMM\g.
			
			\paragraph{Rispetto della pianificazione}
			Per capire se le attività di un processo\g\ sono in ritardo rispetto a quanto pianificato all'interno del \pianodiprogetto\ viene utilizzata la seguente metrica: Schedule Variance.\\
			Si desidera che il ritardo accumulato sia minore del 5\% rispetto al totale pianificato. Sarebbe invece ottimale essere esattamente in linea con quanto prevede il \pianodiprogetto, o essere addirittura in anticipo.\\
			Di seguito sono riportati i valori ottenuti calcolando la Schedule Variance sui tempi di stesura di ogni documento nella fase\g\ PDRD:
			\begin{table}[H]
				\centering
				\begin{tabular}{l * {2}{c}}
					\toprule
					\textbf{Documento} & \textbf{Schedule Variance} & \textbf{Esito} \\
					\midrule
					\textit{Piano di progetto v5.00} & 0\% &  Ottimale \\
					\textit{Norme di progetto v5.00} & 0\% &  Ottimale \\
					\textit{Analisi dei requisiti v5.00} & 0\% &  Ottimale \\
					\textit{Piano di qualifica v5.00} & 0\% &  Ottimale \\
					\textit{Glossario v5.00} & 0\% &  Ottimale \\
					\textit{Definizione di prodotto v2.00} & 0\% &  Ottimale \\
					\textit{Manuale utente v1.00} & 0\% &  Ottimale \\
					\textit{Manuale sviluppatore v1.00} & 0\% &  Ottimale \\
					\bottomrule
				\end{tabular}
				\caption{Esiti del calcolo della Schedule Variance sul processo di documentazione durante la fase PDRD}
				\label{tab:esiti_schedule_variance}
			\end{table}
			
			Come è possibile osservare dai dati della tabella, tutti i documenti hanno dato esito \textbf{ottimale}.
			
			\paragraph{Rispetto del budget}
			Per capire se i costi di un processo\g\ rientrano nel budget previsto dal \pianodiprogetto\ viene utilizzata la seguente metrica: Budget Variance.\\
			L'obiettivo minimo è quello di avere dei costi che non superano il budget a disposizione per più del 10\%. Sarebbe invece ottimale che i costi fossero esattamente in linea con il preventivo o che addirittura si avesse speso meno.\\
			Di seguito sono riportati i valori ottenuti calcolando la Budget Variance sui tempi di stesura di ogni documento nella fase\g\ PDRD:
			\begin{table}[H]
				\centering
				\begin{tabular}{l * {2}{c}}
					\toprule
					\textbf{Documento} & \textbf{Budget Variance} & \textbf{Esito} \\
					\midrule
					\textit{Piano di progetto v5.00} & 0\% &  Ottimale \\
					\textit{Norme di progetto v5.00} & -50\% & Ottimale \\
					\textit{Analisi dei requisiti v5.00} & -31\% & Ottimale \\
					\textit{Piano di qualifica v5.00} & -29\% & Ottimale \\
					\textit{Glossario v5.00} & 0\% & Ottimale \\
					\textit{Definizione di prodotto v2.00} & -33\% & Ottimale \\
					\textit{Manuale utente v1.00} & -50\% & Ottimale \\
					\textit{Manuale sviluppatore v1.00} & -23\% & Ottimale \\
					Totale processo\g\ di documentazione & -27\% & Ottimale \\
					\bottomrule
				\end{tabular}
				\caption{Esiti del calcolo della Budget Variance sul processo di documentazione durante la fase PDRD}
				\label{tab:esiti_budget_variance}
			\end{table}
			
			Come è possibile osservare dai dati della tabella, tutti i documenti hanno dato esito \textbf{ottimale}.
						
		\subsubsection{Processo di verifica}
			\paragraph{Miglioramento costante}
			Il gruppo non ha rilevato miglioramenti tali da raggiungere il quarto livello CMM\g, pertanto il processo\g\ di verifica rimane al terzo livello (Defined).
			
			\paragraph{Rispetto della pianificazione}
			Per capire se le attività di un processo\g\ sono in ritardo rispetto a quanto pianificato all'interno del \pianodiprogetto\ viene utilizzata la seguente metrica: Schedule Variance.\\
			Si desidera che il ritardo accumulato sia minore del 5\% rispetto al totale pianificato. Sarebbe invece ottimale essere esattamente in linea con quanto prevede il \pianodiprogetto, o essere addirittura in anticipo.\\
			Di seguito sono riportati i valori ottenuti calcolando la Schedule Variance sui tempi di verifica nella fase\g\ PDRD:
			
			\begin{table}[H]
				\centering
				\begin{tabular}{l * {2}{c}}
					\toprule
					\textbf{Processo} & \textbf{Schedule Variance} & \textbf{Esito} \\
					\midrule
					Processo\g\ di verifica & 0\% &  Ottimale \\
					\bottomrule
				\end{tabular}
				\caption{Esiti del calcolo della Schedule Variance sul processo di verifica durante la fase PDRD}
				\label{tab:esiti_schedule_variance}
			\end{table}
			
			\paragraph{Rispetto del budget}
			Per il processo\g\ di verifica è stato investito un minor numero di risorse rispetto a quanto preventivato, di conseguenza il valore della Budget Variance risulta \textbf{ottimale}.\\
			Di seguito sono riportati i valori ottenuti:
			\begin{table}[H]
				\centering
				\begin{tabular}{l * {2}{c}}
					\toprule
					\textbf{Processo} & \textbf{Budget Variance} & \textbf{Esito} \\
					\midrule
					Processo\g\ di verifica & -13\% &  Ottimale \\
					\bottomrule
				\end{tabular}
				\caption{Esiti del calcolo della Budget Variance sul processo di verifica durante la fase PDRD}
				\label{tab:esiti_budget_variance}
			\end{table}
	
		\subsubsection{Processo di codifica}
			\paragraph{Miglioramento costante}
			Il gruppo ha cercato di valutare la qualità del processo\g\ di codifica secondo le metriche stabilite dal modello CMM\g: chiaramente, all'inizio della fase\g\ PDRD il processo\g\ si posizionava al livello 1.\\
			In seguito alla redazione di specifiche norme e metriche per la codifica, è stato possibile controllare maggiormente il processo\g\ di codifica, che ha in questo modo guadagnato ripetibilità (richiesta dal livello 2 di CMM\g).
			Possiamo quindi affermare di aver raggiunto il livello 2 della scala CMM\g, perché il processo\g\ di codifica non possiede ancora la principale caratteristica richiesta dal terzo livello, ovvero la proattività.\\
			
			\paragraph{Rispetto della pianificazione}
			Per capire se le attività di un processo\g\ sono in ritardo rispetto a quanto pianificato all'interno del \pianodiprogetto\ viene utilizzata la seguente metrica: Schedule Variance.\\
			Si desidera che il ritardo accumulato sia minore del 5\% rispetto al totale pianificato. Sarebbe invece ottimale essere esattamente in linea con quanto prevede il \pianodiprogetto, o essere addirittura in anticipo.\\
			Di seguito sono riportati i valori ottenuti calcolando la Schedule Variance sui tempi di codifica nella fase\g\ PDRD:
			
			\begin{table}[H]
				\centering
				\begin{tabular}{l * {2}{c}}
					\toprule
					\textbf{Processo} & \textbf{Schedule Variance} & \textbf{Esito} \\
					\midrule
					Processo\g\ di codifica & 0\% &  Ottimale \\
					\bottomrule
				\end{tabular}
				\caption{Esiti del calcolo della Schedule Variance sul processo di verifica durante la fase PDRD}
				\label{tab:esiti_schedule_variance}
			\end{table}
			
			\paragraph{Rispetto del budget}
			Per il processo\g\ di codifica è stato investito un maggior numero di risorse rispetto a quanto preventivato, più del doppio, e questo è dovuto ad un ritardo accumulato nella fase precedente: di conseguenza il valore della Budget Variance risulta \textbf{non accettabile}.\\
			Di seguito sono riportati i valori ottenuti:
			\begin{table}[H]
				\centering
				\begin{tabular}{l * {2}{c}}
					\toprule
					\textbf{Processo} & \textbf{Budget Variance} & \textbf{Esito} \\
					\midrule
					Processo\g\ di codifica & +111\% &  Non accettabile \\
					\bottomrule
				\end{tabular}
				\caption{Esiti del calcolo della Budget Variance sul processo di codifica durante la fase PDRD}
				\label{tab:esiti_budget_variance}
			\end{table}
			
			
	\subsection{Verifica sui prodotti}
	In questa sezione verranno riportati i dati emessi dalle procedure di controllo della qualità di prodotto\g.
	
		\subsubsection{Documenti}
		In questa sezione vengono riportati gli esiti delle attività di verifica svolte sui documenti.\\
		Tali esiti sono strettamente correlati agli obiettivi di qualità dei documenti enunciati alla sezione \ref{ObiettiviDiQualità} del presente documento.
			
			\paragraph{Leggibilità e comprensibilità}
			Per cercare di capire quanto i documenti siano effettivamente leggibili e comprensibili da persone dotate di una licenza superiore viene utilizzato l'indice Gulpease\g.\\
			Si desidera che i documenti posseggano costantemente un indice maggiore a 40 (soglia di accettabilità). Si dovrebbe tuttavia cercare di raggiungere un valore più alto, considerato ottimale, ovvero 60.\\
			Il documento \textit{Glossario v5.00} ha dato esito \textbf{ottimale}, mentre tutti gli altri documenti prodotti hanno dato esito \textbf{accettabile}.
			
			\paragraph{Correttezza ortografica}
			Per capire quanto i documenti siano effettivamente corretti a livello ortografico viene utilizzata la seguente metrica: percentuale di errori ortografici rinvenuti e non corretti.\\
			Si desidera che tutti gli errori ortografici che sono stati trovati siano corretti. In questo caso, dunque, l'obiettivo minimo coincide con l'obiettivo ottimale.\\
			Di seguito sono riportati gli errori ortografici trovati tramite verifica automatica dei documenti durante la fase\g\ PDRD.
			\begin{table}[H]
				\centering
				\begin{tabular}{l * {2}{c}}
					\midrule
					Errori ortografici & 5 \\
					\midrule
				\end{tabular}
				\caption{Errori ortografici trovati tramite verifica automatica dei documenti durante la fase PDRD}
				\label{tab:errori_automatica}
			\end{table}
			Tutti gli errori ortografici rinvenuti sono stati corretti, quindi è stato raggiunto l'obiettivo \textbf{ottimale}.
			
			\paragraph{Correttezza concettuale}
			Per capire quanto i documenti siano effettivamente corretti a livello concettuale viene utilizzata la seguente metrica: percentuale di errori concettuali rinvenuti e non corretti.\\
			Si desidera che al massimo il 5\% degli errori concettuali rinvenuti non siano corretti. L'obiettivo ottimale sarebbe quello di correggere tutti gli errori trovati. \\
			Di seguito sono riportati gli errori concettuali trovati dei documenti durante la fase\g\ PDRD.
			\begin{table}[H]
				\centering
				\begin{tabular}{l * {2}{c}}
					\midrule
					Errori concettuali & 2 \\
					\midrule
				\end{tabular}
				\caption{Errori concettuali trovati tramite verifica manuale dei documenti durante la fase PDRD}
				\label{tab:errori_concettuali}
			\end{table}
			Tutti gli errori concettuali rinvenuti sono stati corretti, quindi è stato raggiunto l'obiettivo \textbf{ottimale}.
			
			
		\subsubsection{Software}
		In questa sezione vengono riportati gli esiti delle attività di verifica svolte sul software.\\
		Tali esiti sono strettamente correlati agli obiettivi di qualità dei documenti enunciati alla sezione \ref{ObiettiviDiQualità} del presente documento.
		
			\paragraph{Funzionalità obbligatorie}
				Il prodotto\g\ deve ricoprire tutte le funzionalità descritte nei requisiti obbligatori. Per monitorare lo stato di completamento delle funzionalità richieste, il gruppo ha pensato di rapportare i requisiti completati con quelli ancora da completare.
				\begin{table}[H]
				\centering
				\begin{tabular}{l * {2}{c}}
					\midrule
					Copertura requisiti obbligatori & 72,8 \% \\
					\midrule
				\end{tabular}
				\caption{Copertura requisiti obbligatori al termine della fase PDRD}
				\label{tab:copertura_requisiti_obbligatori}
			\end{table}
			I requisiti obbligatori non sono stati completamente implementati, perchè si sono riscontrati problemi durante il processo di codifica.L'origine di questi problemi può essere derivata dalla progettazione poco matura e dalla complessità del sistema Android\g.  
			
			\paragraph{Funzionalità desiderabili}
				Il prodotto\g\ deve ricoprire tutte le funzionalità descritte nei requisiti desiderabili. Per monitorare lo stato di completamento delle funzionalità richieste, il gruppo ha pensato di rapportare i requisiti completati con quelli ancora da completare.
			\begin{table}[H]
				\centering
				\begin{tabular}{l * {2}{c}}
					\midrule
					Copertura requisiti desiderabili & 44,4 \% \\
					\midrule
				\end{tabular}
				\caption{Copertura requisiti desiderabili al termine della fase PDRD}
				\label{tab:copertura_requisiti_desiderabili}
			\end{table}
			Sono stati implementati circa la metà dei requisiti desiderabili e per questo l'esito risulta negativo. I motivi di questo risultato possono essere ricondotti ai medesimi citati in precedenza.
			
			\paragraph{Manutenibilità e Comprensibilità del codice}
				Il prodotto\g\ deve avere codice manutenibile e non deve generare incomprensioni al suo interno. Per questo si tiene conto della sua complessità e della sua lunghezza. Codice poco manutenibile può portare all'abbandono dello sviluppo del prodotto\g.
			\begin{table}[H]
				\centering
				\begin{tabular}{l * {2}{c}}
					\toprule
					\textbf{Metrica} & \textbf{Valore} & \textbf{Esito} \\
					\midrule
					Numero di statement per metodo & max 53 & Accettabile \\
					Numero di parametri per metodo & max 9 & Accettabile \\
					Numero di campi dati per classe & max 13 & Accettabile \\
					Grado di accoppiamento medio & 9,41 & Accettabile \\
					Cyclomatic number (medio) & 1,46 & Ottimale \\
					Adequacy of variable names & 100\% & Ottimale \\
					Average Module Size & 16,69 & Ottimale \\
					\midrule
				\end{tabular}
				\caption{Risultati delle metriche per il codice durante la fase PDRD}
				\label{tab:numero_statement_metodo}
			\end{table}
			La qualità del codice secondo le metriche prese in considerazione risulta essere relativamente buona. Nelle prime tre metriche il dato preso in considerazione è il massimo valore riscontrato, mentre nelle altre si usa il valore medio. Dalla penultima metrica si nota che la codifica segue di pari passo la progettazione e non vi sono particolari incongruenze tra le due. Si noti inoltre che ci sono alcuni valori che tendono al limite dell'accettabilità e sono: il "numero di statement per metodo", il "numero di parametri per metodo"  e il "grado di accoppiamento". Essi dovranno essere monitorati durante le prossime fasi.  
			
			\paragraph{Copertura dei test richiesti}
				Il prodotto\g\ deve essere testato in ogni sua parte per garantirne il funzionamento. I test presi in considerazioni sono quelli che testano le funzionalità previste dai requisiti.
			\begin{table}[H]
				\centering
				\begin{tabular}{l * {2}{c}}
					\midrule
					Test passati & 24,5\% \\
					\midrule
				\end{tabular}
				\caption{Test passati al termine della fase PDRD}
				\label{tab:copertura_test}
			\end{table}
			L'esito dato da questa metrica risulta negativo, perchè i test presi in considerazione riguardano tutte le tipologie. Al momento sono implementati e superati con successo circa il 98\% dei test di unità. 
			
			\paragraph{Robustezza}
				Il prodotto\g\ deve essere robusto e non deve interrompere il suo funzionamento in seguito al verificarsi di situazioni anomale\g. Il prodotto\g\ deve essere in grado inoltre di gestire le situazioni di errore.
			\begin{table}[H]
				\centering
				\begin{tabular}{l * {2}{c}}
					\midrule
					Failure avoidance & 80\% \\
					\midrule
				\end{tabular}
				\caption{Failure avoidance al termine della fase PDRD}
				\label{tab:failure_avoidance}
			\end{table}
			Il prodotto in seguito ai vari stress test eseguiti riesce ad evitare nel 80\% dei casi un'interruzione critica. L'esito risulta pertanto accettabile. 
			
			\paragraph{Funzionamento senza interruzioni}
				Il prodotto\g\ deve garantire un funzionamento senza interruzioni. Questo livello è considerato ottimale ma secondo la metrica scelta possono esserci al massimo il 20\% di interruzioni dovute al verificarsi di situazioni anomale\g.
			\begin{table}[H]
				\centering
				\begin{tabular}{l * {2}{c}}
					\midrule
					Breakdown avoidance & 80\% \\
					\midrule
				\end{tabular}
				\caption{Breakdown avoidance al termine della fase PDRD}
				\label{tab:breakdown_avoidance}
			\end{table}	
			Il prodotto è in grado di gestire nel 80\% dei casi delle situazioni anomale e garantire il continuo e corretto funzionamento. Il valore riscontrato risulta pertanto accettabile.
				
	
\end{appendices}
\end{document}