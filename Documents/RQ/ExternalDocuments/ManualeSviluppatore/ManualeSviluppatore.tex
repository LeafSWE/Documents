\documentclass[a4paper,12pt]{article}

\usepackage{leaf}
\usepackage{listings}
\usepackage{enumitem}

\usepackage{listings}
\usepackage{xcolor}

\definecolor{mygreen}{rgb}{0,0.6,0}
\definecolor{mygray}{rgb}{0.5,0.5,0.5}
\definecolor{mymauve}{rgb}{0.58,0,0.82}
\definecolor{lightlightgray}{HTML}{F2F2F2}

\lstset{ %
  backgroundcolor=\color{lightlightgray},   % choose the background color; you must add \usepackage{color} or \usepackage{xcolor}
  basicstyle=\footnotesize,        % the size of the fonts that are used for the code
  breakatwhitespace=false,         % sets if automatic breaks should only happen at whitespace
  breaklines=true,                 % sets automatic line breaking
  captionpos=b,                    % sets the caption-position to bottom
  commentstyle=\color{mygreen},    % comment style
  deletekeywords={...},            % if you want to delete keywords from the given language
  escapeinside={\%*}{*)},          % if you want to add LaTeX within your code
  extendedchars=true,              % lets you use non-ASCII characters; for 8-bits encodings only, does not work with UTF-8
  frame=single,	                   % adds a frame around the code
  keepspaces=true,                 % keeps spaces in text, useful for keeping indentation of code (possibly needs columns=flexible)
  keywordstyle=\color{blue},       % keyword style
  language=Octave,                 % the language of the code
  otherkeywords={*,...},           % if you want to add more keywords to the set
  %numbers=left,                    % where to put the line-numbers; possible values are (none, left, right)
  numbersep=5pt,                   % how far the line-numbers are from the code
  numberstyle=\tiny\color{mygray}, % the style that is used for the line-numbers
  rulecolor=\color{black},         % if not set, the frame-color may be changed on line-breaks within not-black text (e.g. comments (green here))
  showspaces=false,                % show spaces everywhere adding particular underscores; it overrides 'showstringspaces'
  showstringspaces=false,          % underline spaces within strings only
  showtabs=false,                  % show tabs within strings adding particular underscores
  stepnumber=2,                    % the step between two line-numbers. If it's 1, each line will be numbered
  stringstyle=\color{mygreen},	       % string literal style
  tabsize=2,	                   % sets default tabsize to 2 spaces
  title=\lstname                   % show the filename of files included with \lstinputlisting; also try caption instead of title
}


\titlepage{}

\author{Eduard Bicego}
\date{08/05/2016}
\intestazioni{Manuale utente}
\pagenumbering{gobble}

\begin{document}
\begin{titlepage}
	\centering
	{\huge\bfseries CLIPS\par}
	Communication \& Localization with Indoor Positioning Systems \\*
	\line(1,0){350} \\
	{\scshape\LARGE Università di Padova \par}
	\vspace{1cm}
	%devono essere cambiato il titolo ogni volta
	{\scshape\Large Manuale Sviluppatore 1.00 \par}
	\logo
	\newpage
	%devono essere compilati questi campi ogni volta
	\begin{tabular}{c|c}
		{\hfill \textbf{Versione}} 			& 1.00						\\
		{\hfill\textbf{Data Redazione}} 	& 2016-05-08  				\\
		{\hfill\textbf{Redazione}} 			& Eduard Bicego				\\
		{\hfill\textbf{Verifica}} 			&  							\\
		{\hfill\textbf{Approvazione}} 		& Davide Castello			\\
		{\hfill\textbf{Uso}} 				& Esterno					\\
		{\hfill\textbf{Distribuzione}} 		& Prof. Vardanega Tullio	\\
											& Prof. Cardin Riccardo 	\\
											& Miriade S.p.A. 			\\
	\end{tabular}
\end{titlepage}
	
	\newpage
	\pagestyle{myfront}
	
		\newpage
			
\newcolumntype{V}{>{\hsize=.40\hsize}X[cm]}
	\section*{Diario delle modifiche}
\begin{longtabu} to \textwidth {V X[c m 0.8cm] X[c m 0.7cm] X[c m 0.8cm] X[cm]}
	\toprule
	\textbf{Versione} & \textbf{Data}  & \textbf{Autore} & \textbf{Ruolo} & \textbf{Descrizione}\\
	\midrule
	\endhead
	\arrayrulecolor{gray}

1.00 & 2016-05-09 & Davide Castello & Responsabile di Progetto & Approvazione del documento \\
\midrule
0.13 & 2016-05-09 & Federico Tavella & Verificatore & Verifica delle correzioni \\
\midrule
0.12 & 2016-05-08 & Eduard Bicego & Amministratore & Correzioni varie \\
\midrule
0.11 & 2016-05-08 & Federico Tavella & Verificatore & Verifica del documento \\
\midrule
0.10 & 2016-05-08 & Marco Zanella & Progettista & Aggiunta funzionalità Gestione preferenze \\
\midrule
0.09 & 2016-05-08 & Eduard Bicego & Amministratore & Aggiunta funzionalità Localizzazione utente \& Navigazione \\
\midrule
0.08 & 2016-05-07 & Eduard Bicego & Amministratore & Aggiunta sezione Funzionalità \\
\midrule
0.07 & 2016-05-07 & Marco Zanella & Progettista & Aggiunta sezione Strumenti di sviluppo e Componenti Esterni \\ 
\midrule
0.06 & 2016-04-18 & Federico Tavella & Verificatore & Verifica del documento \\
\midrule
0.05 & 2016-04-17 & Eduard Bicego & Amministratore & Stesura sezione Introduzione e Per iniziare \\
\midrule
0.04 & 2016-04-17 & Marco Zanella & Progettista & Stesura sezione Persistenza dei dati \\
\midrule
0.03 & 2016-04-16 & Marco Zanella & Progettista & Aggiunti contenuti sezione Strumenti di sviluppo \\
\midrule
0.02 & 2016-04-16 & Eduard Bicego & Amministratore & Ristrutturato documento \\
\midrule
0.01 & 2016-04-15 & Eduard Bicego & Amministratore & Aggiunta struttura documento \\ 

\arrayrulecolor{black}
	\bottomrule
\end{longtabu}

		\newpage
			\tableofcontents
		\newpage
			\listoffigures
	\label{LastFrontPage}

	\newpage
		\pagestyle{mymain}
	\newpage
		\subfile{sections/Introduzione}
	\newpage
		\subfile{sections/StrumentiDiSviluppo}
	\newpage
		\subfile{sections/ComponentiEsterneUtilizzate}
	%\newpage
		%\subfile{sections/Architettura}
	\newpage
		\subfile{sections/Funzionalita}
		
	% appendices
	\newpage
		\subfile{appendices/FondamentiDiAndroid}
	\newpage
		\subfile{appendices/Json}
	\newpage
		\subfile{appendices/DependencyInjection}

	% Glossario
	% Indice

		
	\label{LastPage}

\end{document}
