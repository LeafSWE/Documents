\documentclass[../ManualeSviluppatore.tex]{subfiles}

\begin{document}
\section{Funzionalità}
	Nella presente sezione vengono spiegate nel dettaglio le componenti dell'applicazione e il loro scopo. Esse sono presentate suddivise nelle funzionalità che offre l'applicazione.

	\subsection{Localizzazione utente}
		
		\subsubsection{Panoramica}
			L'applicazione offre la funzionalità di localizzare l'utente all'interno di un edifico in cui risiedono i beacon riconosciuti dall'applicazione e di mostrare semplici informazioni sull'edifico.
			
			La localizzazione utente avviene seguendo le seguenti fasi:
			\begin{enumerate}
				\item l'utente avvia l'applicazione;
				\item l'applicazione avvia il monitoring per poter rilevare i beacon circostanti;
				\item l'applicazione reperisce l'identificativo major;
				\item l'applicazione si accerta che i beacon rilevati siano pertinenti all'applicazione attraverso un confronto tra major rilevato e major dei beacon nel database locale;
				\item se il beacon è riconosciuto ed esiste un match nel database locale:
					\begin{itemize}
						\item viene costruito il grafico dal database;
						\item vengono mostrate all'utente semplici informazioni sull'edificio;
					\end{itemize}
				\item se il beacon è riconosciuto e non esiste un match nel database locale:
					\begin{itemize}
						\item viene segnalato all'utente che la mappa dell'edificio non è scaricata nel device;
						\item l'utente se lo desidera è reindirizzato alla gestione delle mappe;
					\end{itemize}
				\item se il beacon non è riconosciuto:
					\begin{itemize}
						\item viene ignorato.
					\end{itemize}
			\end{enumerate}
			
		\subsection{Interfaccia grafica}
		
			\paragraph*{Componenti interne}
			\begin{description}
				\item[Package:]
				\item[Classi:]
			\end{description}
			
			\paragraph*{Componenti esterne}
			\begin{description}
				\item[Classi SDK:] 
			\end{description}
			
			??? Immagine dell'applicazione nella home vuota
			
			??? Immagine dell'applicazione nella home con le informazioni dell'edificio
			
			??? Immagini dell'applicazione nella home vuota con messaggio di avviso mappa non disponibile nel device
			
		\subsubsection{Rilevamento beacon}
			Per il rilevamento interno 
			
			\paragraph*{Componenti interne}
			\begin{description}
				\item[Package:] \verb|model|, \verb|beacon|, \verb|dataaccess|;
				\item[Classi:] \href{http://leafswe.github.io/clips/}{BeaconManagerAdapter}
			\end{description}
			
			\paragraph*{Componenti esterne}
			\begin{description}
				\item[Classi SDK]
				\item[Classi AltBeacon:] 
			\end{description}
			
			???
			
		\subsection{Costruzione grafo}
		
		
		%\subsection{Reindirizzamento alla gestione mappe}
			
								
			
			Spiegare ranging e monitoring beacon
			
			avvio app
			information manager e service
			preso un beacon letto il major
			i beacon rilevati appartengono alla nostra app attraverso il setting della region
			una volta riconosciuto un beacon 
			letto i major (uno dei tre identificativi del beacon)
			InforamtionManager chiede a dataaccess DBService se il major corrisponde ad un beacon all'interno di una delle mappe disponibili nel database locale
			Se c'è
				Caricamento della mappa e creazione del grafo
				Restituite le informazioni base, mostrate all'utente
				
			Se non c'è
				
				
			Information manager 
				major in db
					carica mappa
					restituisce info base edificio
				major non presente
					segnalato al presenter la non presenza
					sengalata all'utente attraverso UI
					indirizzato alla gestione mappe
					
			
			

	\subsection{Gestione preferenze}

				
		Settings
		utilizzo
		
	
	\subsection{Gestione delle mappe}
			
		
	
	\subsection{Navigazione}

		\subsubsection{Panoramica}
			La funzionalità di navigazione è resa disponibile dalle componenti dei package:
			\begin{itemize}
				\item \verb|navigator|;
				\item \verb|beacon|;
				\item \verb|compass|;
				\item \verb|dataaccess|;
				\item \verb|setting|.
			\end{itemize}
			Esse permettono di guidare l'utente all'interno di un edificio.
			La navigazione è gestita attraverso queste fasi:
			\begin{enumerate}
				\item l'utente interagendo con l'interfaccia grafica avvia la navigazione;
				\item la business logic dell'applicazione costruisce un grafo;
				\item alle componenti di \verb|navigator| viene passato il grafo;
				\item viene calcolato il percorso utilizzando la libraria \verb|JgraphT|;
				\item vengono restituite le informazioni necessarie per guidare l'utente verso la destinazione da lui scelta;
				\item l'interfaccia mostra all'utente le informazioni.
			\end{enumerate}
			
		\subsubsection{Interfaccia grafica}
		
		\subsection{Presenter}
		
		\subsubsection{Costruzione grafo}
		
			Componenti coinvolte:
			\begin{description}
				\item[Package:] \verb|dataacess|, \verb|dao|, \verb|service|;
				\item[Classi:] \verb||;
				\item[Componenti esterne:] .
			\end{description}
			
			La costruzione del grafo avviene 
				MapGraph
				
				
			dal database
			parti del grafo
			spiegare package graph
			
			
		
		\subsubsection{Calcolo percorso}
		
		\subsubsection{Bussola}
		
			\paragraph*{Componenti coinvolte}
			\begin{description}
				\item[Package:] \verb|compass|;
				\item[Classi:] \verb|Compass|.
			\end{description}

			\paragraph*{Componenti esterne}
			\begin{description}
				\item[Classi SDK:] \verb|SensorManager|, \verb|Sensor|, \verb|SensorEventListener|.
			\end{description}
			
			
			
		
		\subsubsection{Navigazione}
		
			\paragraph*{Componenti interne}
			\begin{description}
				\item[Package:] \verb|navigator|, \verb|graph|, \verb|edge|, \verb|vertex|, \verb|area|;
				\item[Classi:] \verb||.
			\end{description}
			
			\paragraph*{Componenti esterne}
			\begin{description}
				\item[Classi JGraphT:] \verb|SimpleDirectedWeightedGraph|, \verb|DijkstraPathFinder|, \verb|DefaultWeightedEdge|;
				\item[Classi JDK:] \verb|Exception|.
			\end{description}

			
	
	\subsection{Area sviluppatore}

\end{document}