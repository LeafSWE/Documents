\documentclass[../ManualeSviluppatore.tex]{subfiles}

\begin{document}
\section{Funzionalità}
	Nella presente sezione vengono spiegate nel dettaglio le componenti dell'applicazione e il loro scopo, presentate per funzionalità offerte dall'applicazione.
	Ogni funzionalità viene prima descritta in una sottosezione "Panoramica" e successivamente in sottosezioni che approfondiscono gli aspetti che compongono la funzionalità.

	\subsection{Localizzazione utente}
		
		\subsubsection{Panoramica}
			L'applicazione offre la funzionalità di localizzare l'utente all'interno di un edifico in cui risiedono i beacon riconosciuti dall'applicazione e di mostrare semplici informazioni sull'edifico.
			
			La localizzazione utente avviene seguendo le seguenti fasi:
			\begin{enumerate}
				\item l'utente avvia l'applicazione;
				\item l'applicazione avvia il monitoring per poter rilevare i beacon circostanti;
				\item l'applicazione reperisce l'identificativo major;
				\item l'applicazione si accerta che i beacon rilevati siano pertinenti all'applicazione attraverso un confronto tra major rilevato e major dei beacon nel database locale;
				\item se il beacon è riconosciuto ed esiste un match nel database locale:
					\begin{itemize}
						\item viene costruito il grafico dal database;
						\item vengono mostrate all'utente semplici informazioni sull'edificio;
					\end{itemize}
				\item se il beacon è riconosciuto e non esiste un match nel database locale:
					\begin{itemize}
						\item viene segnalato all'utente che la mappa dell'edificio non è scaricata nel device;
						\item l'utente se lo desidera è reindirizzato alla gestione delle mappe;
					\end{itemize}
				\item se il beacon non è riconosciuto:
					\begin{itemize}
						\item viene ignorato.
					\end{itemize}
			\end{enumerate}
			
		\subsection{Interfaccia grafica}
		
			\paragraph*{Componenti interne}
			\begin{description}
				\item[Package:]
				\item[Classi:]
			\end{description}
			
			\paragraph*{Componenti esterne}
			\begin{description}
				\item[Classi SDK:] 
			\end{description}
			
			??? Immagine dell'applicazione nella home vuota
			
			??? Immagine dell'applicazione nella home con le informazioni dell'edificio
			
			??? Immagini dell'applicazione nella home vuota con messaggio di avviso mappa non disponibile nel device
			
		\subsection{Presenter}
		
			??? Activity
			
		\subsubsection{Rilevamento beacon}
			Per il rilevamento interno 
			
			\paragraph*{Componenti interne}
			\begin{description}
				\item[Package:] \verb|model|, \verb|beacon|, \verb|dataaccess|;
				\item[Classi:] \href{http://leafswe.github.io/clips/com/leaf/clips/model/beacon/BeaconManagerAdapter.html}{BeaconManagerAdapter}
			\end{description}
			
			\paragraph*{Componenti esterne}
			\begin{description}
				\item[Classi JDK:] \verb|PriorityQueue|;
				\item[Classi SDK:] \verb|Intent|, \verb|LocalBroadcastManager|, \verb|Service|, \verb|Binder|, \\ \verb|LocalBroadcastManager|;
				\item[Classi AltBeacon:] \verb|BeaconManager|, \verb|BootstrapNotifier|, \verb|BeaconConsumer|, \verb|RangeNotifier|, \verb|Region|, \verb|BeaconParser|,  \verb|DistanceCalculator|, \verb|Beacon|.
			\end{description}
			
			%Attraverso la classe BeaconManagerImp viene avviato il Service BeaconManagerAdapter il quale inizia il monitoring costante in background.
			%Se la region dei beacon rilevati è coincidente a quella dell'applicativo inizia il ranging.
			
			La classe \href{http://leafswe.github.io/clips/com/leaf/clips/model/beacon/BeaconManagerAdapter.html}{BeaconManagerAdapter} si occupa di effettuare il ranging e il monitoring dei beacon circostanti. 
			Essa estende un Service (vedi appendice )per cui presenta un ciclo di vita così gestito:
			
			
			
			
			
			
			I principali metodi sono:
			\begin{itemize}
				\item 
				\item 
				\item 
			\end{itemize}						
			
			Classe AltBeacon:
				???
			Classe AltBeacon:
				???
			
		\subsection{Costruzione grafo}
		
		
		%\subsection{Reindirizzamento alla gestione mappe}
						
			
			

	\subsection{Gestione preferenze}

				
		Settings
		utilizzo
		
	
	\subsection{Gestione delle mappe}
			
		
	
	\subsection{Navigazione}

		\subsubsection{Panoramica}
			La funzionalità di navigazione è resa disponibile dalle componenti dei package:
			\begin{itemize}
				\item \verb|navigator|;
				\item \verb|beacon|;
				\item \verb|compass|;
				\item \verb|dataaccess|;
				\item \verb|setting|.
			\end{itemize}
			Esse permettono di guidare l'utente all'interno di un edificio.
			La navigazione è gestita attraverso queste fasi:
			\begin{enumerate}
				\item l'utente interagendo con l'interfaccia grafica avvia la navigazione;
				\item la business logic dell'applicazione costruisce un grafo;
				\item alle componenti di \verb|navigator| viene passato il grafo;
				\item viene calcolato il percorso utilizzando la libraria \verb|JgraphT|;
				\item vengono restituite le informazioni necessarie per guidare l'utente verso la destinazione da lui scelta;
				\item l'interfaccia mostra all'utente le informazioni.
			\end{enumerate}
			
		\subsubsection{Interfaccia grafica}
		
		\subsection{Presenter}
		
		\subsubsection{Costruzione grafo}
		
			Componenti coinvolte:
			\begin{description}
				\item[Package:] \verb|dataacess|, \verb|dao|, \verb|service|;
				\item[Classi:] \verb||;
				\item[Componenti esterne:] .
			\end{description}
			
			La costruzione del grafo avviene 
				MapGraph
				
				
			dal database
			parti del grafo
			spiegare package graph
			
			
		
		\subsubsection{Calcolo percorso}
		
		\subsubsection{Bussola}
		
			\paragraph*{Componenti coinvolte}
			\begin{description}
				\item[Package:] \verb|compass|;
				\item[Classi:] \verb|Compass|.
			\end{description}

			\paragraph*{Componenti esterne}
			\begin{description}
				\item[Classi SDK:] \verb|SensorManager|, \verb|Sensor|, \verb|SensorEventListener|.
			\end{description}
			
			
			
		
		\subsubsection{Navigazione}
		
			\paragraph*{Componenti interne}
			\begin{description}
				\item[Package:] \verb|navigator|, \verb|graph|, \verb|edge|, \verb|vertex|, \verb|area|;
				\item[Classi:] \verb||.
			\end{description}
			
			\paragraph*{Componenti esterne}
			\begin{description}
				\item[Classi JGraphT:] \verb|SimpleDirectedWeightedGraph|, \verb|DijkstraPathFinder|, \verb|DefaultWeightedEdge|;
				\item[Classi JDK:] \verb|Exception|.
			\end{description}

			
	
	\subsection{Area sviluppatore}

\end{document}