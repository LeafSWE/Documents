\documentclass[../Funzionalita.tex]{subfiles}

\begin{document}

\subsection{Navigazione}
\label{subsec:Navigazione}

		\subsubsection{Panoramica}
			La funzionalità di navigazione è resa disponibile dalle componenti dei package:
			\begin{itemize}
				\item \verb|navigator|;
				\item \verb|beacon|;
				\item \verb|compass|;
				\item \verb|dataaccess|;
				\item \verb|setting|.
			\end{itemize}
			Esse permettono di guidare l'utente all'interno di un edificio.
			La navigazione è gestita attraverso queste fasi:
			\begin{enumerate}
				\item l'utente interagendo con l'interfaccia grafica avvia la navigazione;
				\item la business logic dell'applicazione costruisce un grafo;
				\item alle componenti di \verb|navigator| viene passato il grafo;
				\item viene calcolato il percorso utilizzando la libraria \verb|JgraphT|;
				\item vengono restituite le informazioni necessarie per guidare l'utente verso la destinazione da lui scelta;
				\item l'interfaccia mostra all'utente le informazioni.
			\end{enumerate}
			
		\subsubsection{Interfaccia grafica}
			
			\paragraph*{Componenti interne}
			\begin{itemize}
			
				\item Package:
				\begin{itemize}
					\item[]
					\item[] ...
				\end{itemize}
				
				\item Interfacce e classi:
				\begin{itemize}
					\item[] ...
				\end{itemize}
				
			\end{itemize}
			
			
			\paragraph*{Componenti esterne}
			
			\begin{itemize}
				\item Interfacce e classi SDK:
				\begin{itemize}
					\item[] ...
				\end{itemize}
			\end{itemize}
		
		\subsection{Presenter}
		
			\paragraph*{Componenti interne}
			\begin{itemize}
			
				\item Package:
				\begin{itemize}
					\item[]
					\item[] ...
				\end{itemize}
				
				\item Interfacce e classi:
				\begin{itemize}
					\item[] ...
				\end{itemize}
				
			\end{itemize}
			
			
			\paragraph*{Componenti esterne}
			
			\begin{itemize}
				\item Interfacce e classi SDK:
				\begin{itemize}
					\item[] ...
				\end{itemize}
			\end{itemize}
					
		
		\subsubsection{Calcolo percorso}
		
			\paragraph*{Componenti interne}
			\begin{itemize}
			
				\item Package:
				\begin{itemize}
					\item[] \navigator;
					\item[] \algorithm;
				\end{itemize}
				
				\item Interfacce e classi:
				\begin{itemize}
					\item[] ...
				\end{itemize}
				
			\end{itemize}
			
			
			\paragraph*{Componenti esterne}
			
			\begin{itemize}
				\item Interfacce e classi SDK:
				\begin{itemize}
					\item[] ...
				\end{itemize}
			\end{itemize}
		
		
		\subsubsection{Bussola}
		
			\paragraph*{Componenti interne}
			\begin{itemize}
			
				\item Package:
				\begin{itemize}
					\item[] \compass;
				\end{itemize}
				
				\item Interfacce e classi:
				\begin{itemize}
					\item[] \Compass;
				\end{itemize}
				
			\end{itemize}
			
			
			\paragraph*{Componenti esterne}
			\begin{itemize}
			
				\item Interfacce e classi SDK:
				\begin{itemize}
					\item[] \SensorManager, \Sensor, \SensorEventListener.
				\end{itemize}
				
			\end{itemize}
			
			La classe \Compass\ permette all'applicazione di ricevere dati dai sensori hardware del device gestiti grazie alla classe \Sensor. \Compass\ rende disponibili i metodi per registrare i listener ai sensori e per rimuoverla. Poiché i sensori comunicano attraverso eventi tramite interfaccia \SensorEventListener\ i dati recuperati della bussola non corrispondono all'istante in cui sono recuperati.
			
		
		\subsubsection{Navigazione}
		
			\paragraph*{Componenti interne}
			\begin{description}
				\item[Package:] \verb|navigator|, \verb|graph|, \verb|edge|, \verb|vertex|, \verb|area|;
				\item[Classi:] \verb||.
			\end{description}
			
			\paragraph*{Componenti esterne}
			\begin{description}
				\item[Classi JGraphT:] \verb|SimpleDirectedWeightedGraph|, \verb|DijkstraPathFinder|, \verb|DefaultWeightedEdge|;
				\item[Classi JDK:] \verb|Exception|.
			\end{description}
			
		\subsubsection{Eccezioni e gestione}
		\paragraph*{Componenti interne}
			\begin{itemize}
			
				\item Package:
				\begin{itemize}
					\item[] \model;
					\item[] \navigator;
				\end{itemize}
				
				\item Interfacce e classi:
				\begin{itemize}
					\item[] \NavigationManagerImp, \Navigator, \NavigatorImp, \NavigationExceptions, \NoGraphSetException, \PathException, \NoNavigationInformationException;
				\end{itemize}
				
			\end{itemize}
			
			
			\paragraph*{Componenti esterne}
			\begin{itemize}
			
				\item Interfacce e classi JDK:
				\begin{itemize}
					\item[] \Exception.
				\end{itemize}
				
			\end{itemize}
			
			Nel package \navigator vengono lanciate delle eccezioni per far sì che chiunque le utilizzi rispetti un particolare ordine. Tale ordine coinvolge le seguenti operazioni:
			\begin{itemize}
				\item Set del grafo in \NavigatorImp;
				\item Calcolo del percorso attraverso \NavigatorImp;
				\item Esecuzione della navigazione.
			\end{itemize}
			Il non rispetto di tale ordine può sollevare diversi tipi di eccezioni:
			\begin{itemize}
				\item \NoGraphSetException\ se il grafo non è stato settato e si richiede il calcolo del percorso o l'esecuzione della navigazione;
				\item \NoNavigationInformationException\ se si avvia la navigazione ma non si è calcolato il percorso precedentemente.
			\end{itemize}
			Mentre se il rilevamento dei beacon non corrisponde con quanto previsto, a significare che l'utente sta sbagliando percorso, viene lanciata l'eccezione:
			\begin{itemize}
				\item \PathException.
			\end{itemize}
			Nell'applicazione tali operazioni sono gestite da \NavigationManagerImp.

\end{document}