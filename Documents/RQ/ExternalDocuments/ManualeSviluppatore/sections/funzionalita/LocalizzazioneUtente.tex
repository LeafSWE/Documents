\documentclass[../Funzionalita.tex]{subfiles}

\begin{document}

\subsection{Localizzazione utente}
\label{subsec:LocalizzazioneUtente}
		
		\subsubsection{Panoramica}
			L'applicazione offre la funzionalità di localizzare l'utente all'interno di un edifico in cui risiedono i beacon riconosciuti dall'applicazione e di mostrare semplici informazioni sull'edifico.
			
			La localizzazione utente avviene seguendo le seguenti fasi:
			\begin{enumerate}
				\item l'utente avvia l'applicazione;
				\item l'applicazione avvia il monitoring per poter rilevare i beacon circostanti;
				\item l'applicazione reperisce l'identificativo major;
				\item l'applicazione si accerta che i beacon rilevati siano pertinenti all'applicazione attraverso un confronto tra major rilevato e major dei beacon nel database locale;
				\item se il beacon è riconosciuto ed esiste un match nel database locale:
					\begin{itemize}
						\item viene costruito il grafico dal database;
						\item vengono mostrate all'utente semplici informazioni sull'edificio;
					\end{itemize}
				\item se il beacon è riconosciuto e non esiste un match nel database locale:
					\begin{itemize}
						\item viene segnalato all'utente che la mappa dell'edificio non è scaricata nel device;
						\item l'utente se lo desidera è reindirizzato alla gestione delle mappe;
					\end{itemize}
				\item se il beacon non è riconosciuto:
					\begin{itemize}
						\item viene ignorato.
					\end{itemize}
			\end{enumerate}
			
		\newpage
		\subsubsection{Interfaccia grafica}
		
			\paragraph*{Componenti interne}
			\begin{itemize}
			
				\item Package:
				\begin{itemize}
					\item[]
					\item[] ...
				\end{itemize}
				
				\item Interfacce e classi:
				\begin{itemize}
					\item[] ...
				\end{itemize}
				
			\end{itemize}
			
			
			\paragraph*{Componenti esterne}
			
			\begin{itemize}
				\item Interfacce e classi SDK:
				\begin{itemize}
					\item[] ...
				\end{itemize}
			\end{itemize}
			
			??? Immagine dell'applicazione nella home vuota
			
			??? Immagine dell'applicazione nella home con le informazioni dell'edificio
			
			??? Immagini dell'applicazione nella home vuota con messaggio di avviso mappa non disponibile nel device
			
		\subsubsection{Presenter}
		
			\paragraph*{Componenti interne}
			\begin{itemize}
			
				\item Package:
				\begin{itemize}
					\item[]
					\item[] ...
				\end{itemize}
				
				\item Interfacce e classi:
				\begin{itemize}
					\item[] ...
				\end{itemize}
				
			\end{itemize}
			
			
			\paragraph*{Componenti esterne}
			
			\begin{itemize}
				\item Interfacce e classi SDK:
				\begin{itemize}
					\item[] ...
				\end{itemize}
			\end{itemize}
			
			\begin{itemize}[label={--}]
				\item \verb|onCreate()| 
				\item \verb|onDestroy()|
				\item ...
			\end{itemize}
			
		\subsubsection{Rilevamento beacon}
			La classe \BeaconManagerAdapter\ estende un bind \Service\ (vedi appendice \ref{sec:FondamentiDiAndroid}) ed ha il compito di effettuare il ranging e il monitoring dei beacon circostanti. Il ranging è l'operazione svolta in background per riconoscere i beacon circostanti senza eccessiva precisione mentre il monitoring è l'operazione che segue in cui i beacon vengono invece rilevati con tutte le informazioni e con più precisamente.
			La comunicazione dei beacon rilevati dal \model\ avviene attraverso l'uso degli oggetti \MyBeacon\ inviati tramite \Intent\ per cui serializzati. Gli \Intent\ vengono recuperati tramite \BroadcastReceiver\ implementato in altre classi.
			
			\paragraph*{Componenti interne}
			\begin{itemize}
			
				\item Package:
				\begin{itemize}
					\item[] \model;
					\item[] \beacon;
				\end{itemize}
				
				\item Interfacce e classi:
				\begin{itemize}
					\item[] \BeaconManagerAdapter, \MyBeacon, \MyBeaconImp, \MyDistanceCalculator, \LocalBinder, \BeaconRanger;
				\end{itemize}
												
			\end{itemize}
			
			\paragraph*{Componenti esterne}
			\begin{itemize}
			
				\item Interfacce e classi AltBeacon:
				\begin{itemize}
					\item[] \BeaconManager, \BootstrapNotifier, \BeaconConsumer, \RangeNotifier, \Region, \BeaconParser, \DistanceCalculator, \Beacon;
				\end{itemize}
			
				\item Interfacce e classi JDK:
				\begin{itemize}
					\item[] \PriorityQueue;
				\end{itemize}
				
				\item Interfacce e classi SDK:
				\begin{itemize}
					\item[] \Intent, \LocalBroadcastManager, \Service, \Binder, \LocalBroadcastManager, \IBinder.
				\end{itemize}
				
				
				
			\end{itemize}
			
			
		\subsubsection{Costruzione grafo}
		
			\paragraph*{Componenti interne}
			\begin{itemize}
			
				\item Package:
				\begin{itemize}
					\item[] \model
					\item[] \dataaccess
					\item[] \service
					\item[] \dao
					\item[] \graph
					\item[] \edge
					\item[] \vertex
					\item[] \area
					\item[] \navigationinformation
				\end{itemize}
				
				\item Interfacce e classi:
				\begin{itemize}
					\item[] \MapGraph, 
				\end{itemize}
			\end{itemize}
			
			\paragraph*{Componenti esterne}
			\begin{itemize}
			
				\item Interfacce e classi JDK:
				\begin{itemize}
					\item[]	???
				\end{itemize}
			
				\item Interfacce e classi JGraphT:
				\begin{itemize}
					\item[] ???
				\end{itemize}
				
				\item Interfacce e classi SDK:
				\begin{itemize}
					\item[]	???
				\end{itemize}
				
			\end{itemize}
			
			La costruzione del grafo avviene MapGraph
				
			dal database
			parti del grafo
			spiegare package graph
		
		%\subsection{Reindirizzamento alla gestione mappe}
		
\end{document}