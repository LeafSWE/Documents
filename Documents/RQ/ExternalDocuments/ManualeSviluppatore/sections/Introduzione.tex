\documentclass[../ManualeSviluppatore.tex]{subfiles}

\begin{document}
\section{Introduzione}
\label{sec:Introduzione}
	\subsection{Panoramica generale}
	
	CLIPS è un'applicazione che offre funzionalità riguardanti la navigazione guidata all'interno di edifici, attraverso l'utilizzo di dispositivi mobile \gls{Android} e dei \gls{beacon}.
	
	Tale applicazione è stata sviluppata per avere la possibilità di accedere alle informazioni per raggiungere una specifica area di interesse di un edificio offrendo indicazione testuali, sonore e visuali.
	
	Questo manuale ha lo scopo di illustrare le parti che compongono tale applicazione e far comprendere al lettore il funzionamento di tali parti con fine ultimo la manutenzione e l'estensione.
	
	\subsection{Struttura del manuale}
		Il manuale è strutturato in diverse sezioni:
		\begin{enumerate}
			\item \textbf{Introduzione}
			\item \textbf{Per iniziare:} spiega come scaricare il progetto e configurare l'IDE \gls{Android Studio} per aprirlo;
			\item \textbf{Strumenti di sviluppo:} presenta tutti gli strumenti e kit di sviluppo utilizzati per supportare lo sviluppo dell'applicazione e la sua creazione;
			\item \textbf{Componenti esterne:} raccoglie tutte le librerie esterne utilizzate nel progetto presentando il loro utilizzo all'interno dell'applicazione e la versione utilizzata;
			\item \textbf{Mappatura edificio:} raccoglie la soluzione proposta per realizzare un sistema di navigazione interna e delle istruzioni per estendere gli edifici supportati dall'applicazione;
			\item \textbf{Architettura applicazione:} presenta una panoramica astratta per comprendere come siano strutturate le componenti all'interno dell'applicazione;
			\item \textbf{Funzionalità:} presenta le componenti dell'applicazione suddivise per funzionalità, ogni funzionalità è descritta attraverso diversi passaggi, prima raccolti in sequenza nella sottosezione \textit{Panoramica} e successivamente descritti approfonditamente in una sottosezione per ognuno;
			\item \textbf{Persistenza dei dati:} presenta e descrive come i dati necessari per il funzionamento dell'applicazione siano salvati in un database locale e remoto. Inoltre descrive la struttura degli oggetti (\textit{row} delle \textit{table} dei database SQL) trasmessi via rete Internet in formato Json.
			
		\end{enumerate}
		
	\subsection{Glossario e documentazione}
		Allo scopo di rendere più semplice e chiara la comprensione del manuale
viene allegato il glossario nel quale sono raccolte le spiegazioni di
terminologia tecnica o ambigua, abbreviazioni ed acronimi. I termini evidenziati in \textcolor{myred}{rosso} sono link alla voce in esso.
		Per agevolare l'uso e facilitare l'approfondimento del contenuto descritto nel manuale, ogni componente, classe o interfaccia interna (appartenente a codice sviluppato dal gruppo \leaf) o esterna (codice sviluppato da terzi) è evidenziata in \textcolor{blue}{blu} e corrisponde ad un collegamento ipertestuale alla documentazione di tale componente.
	
	
	\subsection{Riferimenti utili}
		\begin{itemize}
			\item Documentazione \gls{AltBeacon}: \\ \url{https://altbeacon.github.io/android-beacon-library/javadoc/};
			\item Documentazione Clips: \\ \url{http://leafswe.github.io/clips/};
			\item Documentazione Dagger: \\ \url{http://google.github.io/dagger/api/2.0/};
			\item Documentazione Gson: \\ \url{http://google-gson.googlecode.com/svn/trunk/gson/docs/javadocs/index.html};
			\item Documentazione \gls{Java} JDK \\ \url{https://docs.oracle.com/javase/8/docs/api/overview-summary.html};
			\item Documentazione Picasso: \\ \url{https://square.github.io/picasso/2.x/picasso/};
			\item Documentazione \gls{Android} SDK: \\ \url{http://developer.android.com/reference/packages.html}

			\item Documentazione Gradle: \\ \url{http://gradle.org/documentation/}			

		\end{itemize}
\end{document}