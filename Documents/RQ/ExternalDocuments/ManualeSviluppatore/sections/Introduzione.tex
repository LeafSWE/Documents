\documentclass[../ManualeSviluppatore.tex]{subfiles}

\begin{document}
\section{Introduzione}
\label{sec:Introduzione}
	\subsection{Panoramica generale}
	
	CLIPS è un'applicazione che offre funzionalità riguardanti la navigazione guidata all'interno di edifici, attraverso l'utilizzo di dispositivi mobile Android e dei beacon.
	
	Tale applicazione è stata sviluppata per avere la possibilità di accedere alle informazioni per raggiungere una specifica area di interesse di un edificio offrendo indicazione testuali, sonore e visuali.
	
	Questo manuale ha lo scopo di illustrare le parti che compongono tale applicazione e far comprendere al lettore il funzionamento di tali parti con fine ultimo la manutenzione e l'estensione.
	
	\subsection{Struttura del manuale}
		Il manuale è strutturato in TOT sezioni:
		\begin{enumerate}
			\item \textbf{Introduzione}
			\item \textbf{Per iniziare:} spiega come scaricare il progetto e configurare l'IDE Android Studio per aprirlo;
			\item \textbf{Strumenti di sviluppo:} presenta tutti gli strumenti e kit di sviluppo utilizzati per supportare lo sviluppo dell'applicazione e la sua creazione;
			\item \textbf{Componenti esterne:} raccoglie tutte le librerie esterne utilizzate nel progetto presentando il loro utilizzo all'interno dell'applicazione e la versione utilizzata;
			\item \textbf{Architettura applicazione:} presenta una panoramica astratta per comprendere come siano strutturate le componenti all'interno dell'applicazione.
			\item \textbf{Funzionalità:} presenta le componenti dell'applicazione suddivise per funzionalità, ogni funzionalità è descritta attraverso diversi passaggi, prima raccolti in sequenza nella sottosezione \textit{Panoramica} e successivamente descritti approfonditamente in una sottosezione per ognuno.
			\item \textbf{Persistenza dei dati:} presenta e descrive come i dati necessari per il funzionamento dell'applicazione siano salvati in un database locale e remoto. Inoltre descrive la struttura degli oggetti (\textit{row} delle \textit{table} dei database SQL) trasmessi via rete Internet in formato Json.
			
		\end{enumerate}
	
	
	\subsection{Riferimenti utili}
		\begin{itemize}
			\item Javadoc AltBeacon: \\ \url{https://altbeacon.github.io/android-beacon-library/javadoc/};
			\item Javadoc Clips: \\ \url{http://leafswe.github.io/clips/};
			\item Javadoc Dagger: \\ \url{http://google.github.io/dagger/api/2.0/};
			\item Javadoc Gson: \\ \url{http://google-gson.googlecode.com/svn/trunk/gson/docs/javadocs/index.html};
			\item Javadoc Java JDK \\ \url{https://docs.oracle.com/javase/8/docs/api/overview-summary.html};
			\item Javadoc Picasso: \\ \url{https://square.github.io/picasso/2.x/picasso/};
			\item Javadoc Android SDK: \\ \url{http://developer.android.com/reference/packages.html}

			\item Documentazione Gradle: \\ \url{http://gradle.org/documentation/}			

		\end{itemize}
\end{document}