
\subsection{Navigazione}

		\subsubsection{Panoramica}
			La funzionalità di navigazione è resa disponibile dalle componenti dei package:
			\begin{itemize}
				\item \verb|navigator|;
				\item \verb|beacon|;
				\item \verb|compass|;
				\item \verb|dataaccess|;
				\item \verb|setting|.
			\end{itemize}
			Esse permettono di guidare l'utente all'interno di un edificio.
			La navigazione è gestita attraverso queste fasi:
			\begin{enumerate}
				\item l'utente interagendo con l'interfaccia grafica avvia la navigazione;
				\item la business logic dell'applicazione costruisce un grafo;
				\item alle componenti di \verb|navigator| viene passato il grafo;
				\item viene calcolato il percorso utilizzando la libraria \verb|JgraphT|;
				\item vengono restituite le informazioni necessarie per guidare l'utente verso la destinazione da lui scelta;
				\item l'interfaccia mostra all'utente le informazioni.
			\end{enumerate}
			
		\subsubsection{Interfaccia grafica}
		
		\subsection{Presenter}
		
		\subsubsection{Costruzione grafo}
		
			Componenti coinvolte:
			\begin{description}
				\item[Package:] \verb|dataacess|, \verb|dao|, \verb|service|;
				\item[Classi:] \verb||;
				\item[Componenti esterne:] .
			\end{description}
			
			La costruzione del grafo avviene 
				MapGraph
				
				
			dal database
			parti del grafo
			spiegare package graph
			
			
		
		\subsubsection{Calcolo percorso}
		
		\subsubsection{Bussola}
		
			\paragraph*{Componenti coinvolte}
			\begin{description}
				\item[Package:] \verb|compass|;
				\item[Classi:] \verb|Compass|.
			\end{description}

			\paragraph*{Componenti esterne}
			\begin{description}
				\item[Classi SDK:] \verb|SensorManager|, \verb|Sensor|, \verb|SensorEventListener|.
			\end{description}
			
			
			
		
		\subsubsection{Navigazione}
		
			\paragraph*{Componenti interne}
			\begin{description}
				\item[Package:] \verb|navigator|, \verb|graph|, \verb|edge|, \verb|vertex|, \verb|area|;
				\item[Classi:] \verb||.
			\end{description}
			
			\paragraph*{Componenti esterne}
			\begin{description}
				\item[Classi JGraphT:] \verb|SimpleDirectedWeightedGraph|, \verb|DijkstraPathFinder|, \verb|DefaultWeightedEdge|;
				\item[Classi JDK:] \verb|Exception|.
			\end{description}