\documentclass[../PianoProgetto.tex]{subfiles}

\begin{document}

\section{Preventivo a finire}

	\subsection{Fase AD}
		Alla luce dei risultati negativi del consuntivo si è resa necessaria la ripianificazione delle successive fasi per portare il bilancio in pari o in positivo. Visto il numero di ore per il ruolo di analista durante questa fase si è scelto di diminuirle per la prossima.
	
	\subsection{Fase PA}
		Alla luce dei risultati positivi del consuntivo il risparmio di \euro\ 250,00 riporta il bilancio totale in positivo con \euro\ 40 che consentiranno investimenti nelle prossime fasi o di far fronte a rischi non preventivati.
	
	\subsection{Fase PDROB}
		Alla luce dei risultati positivi del consuntivo il risparmio di \euro\ 308,00 sommato al risultato positivo precedente di \euro\ 40 ci permette di avere un risparmio totale di \euro 348,00. 300,00 euro di tale risultato saranno investiti nella successiva fase (fase PDRD) nelle ore da programmatore. Infatti nell'ultima fase è stata posticipata gran parte dell'attività di codifica dei requisiti obbligatori. In totale quindi si prevede un incremento di 20 ore al ruolo di programmatore nella fase PDRD (40 ore in totale).
	
	%\subsection{Fase PDRD}
	
	%\subsection{Fase PDROP}

\end{document}