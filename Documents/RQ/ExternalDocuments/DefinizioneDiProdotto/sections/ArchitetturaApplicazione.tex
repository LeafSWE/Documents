\documentclass[../DefinizioneDiProdotto.tex]{subfiles}

\begin{document}

\section{Architettura applicazione}

	\subsection{Architettura ad alto livello}
		L'architettura dell'applicazione rispecchia il pattern architetturale a livelli presentando quattro livelli:
		\begin{description}
			\item[Presentation layer:] contiene le componenti dell'interfaccia grafica utente passiva e le componenti facenti parte dell'interfaccia di comunicazione tra vista e il layer sottostante;
			\item[Business layer:] contiene le componenti logiche del model tranne quelle adibite all'accesso al database;
			\item[Service layer:] contiene le componenti adibite a interfacciare le componenti della logica dell'applicativo con quelle che si interfacciano direttamente al database;
			\item[Persistance layer:] contiene le componenti che comunicano direttamente con il database e vengono utilizzate dal Service layer; 
			\item[Database layer:] corrisponde al database SQLite all'interno del dispositivo mobile.
		\end{description}
	La scelta di tale pattern architetturale deriva dai vantaggi che ne derivano:
	\begin{itemize}
		\item Garantisce un ambiente facile da testare;
		\item Facilita lo sviluppo grazie alla sua semplicità teorica;
		\item Disaccoppiamento delle componenti con diversi scopi;
	\end{itemize}
	
	\begin{figure} [h]
		\centering
		\includegraphics[scale=0.4]{img/LayeredArchitecture}
		\caption{Architettura a layer adottata nell'applicazione}
		\label{fig:LayerArchitecture}
	\end{figure}
		
	
\subsection{Diagrammi riassuntivi package significativi}
Di seguito sono riportati tutti i package dell'applicativo ad eccezione della \verb|view|, essendo applicato il pattern MVP, per chiarire la relazione tra le componenti e le classi al suo interno. Per chiarezza ed esigenza di spazio le classi rappresentate all'interno dei package sono senza metodi e attributi.

\begin{figure}[H]
	\includegraphics[angle=90,width=\textwidth, height=\textheight, keepaspectratio]{diagrams/ModelCompleteNoMethods/PNGpackage/model}
	\label{modelPackage}
	\caption{Diagramma delle classi - model}
\end{figure}

\begin{figure}[H]
	\includegraphics[angle=90,width=\textwidth,, height=580pt, keepaspectratio]{diagrams/ModelCompleteNoMethods/PNGpackage/navigator}
	\label{navigatorPackage}
	\caption{Diagramma delle classi - model::navigator}
\end{figure}

\begin{figure}[H]
	\includegraphics[width=\textwidth]{diagrams/ModelCompleteNoMethods/PNGpackage/dataaccess}
	\label{dataaccessPackage}
	\caption{Diagramma delle classi - model::dataaccess}
\end{figure}

\begin{figure}[H]
	\includegraphics[width=\textwidth]{diagrams/ModelCompleteNoMethods/PNGpackage/presenter}
	\label{presenterPackage}
	\caption{Diagramma delle classi - presenter}
\end{figure}

\begin{figure}[H]
\centering
\includegraphics[height=20cm]{diagrams/ModelCompleteNoMethods/PNGpackage/view}
\label{viewPackage}
\caption{Diagramma delle classi - view}
\end{figure}

\begin{figure}[H]
	\centering
	\includegraphics[height=19cm,width=\textwidth]{diagrams/ModelCompleteNoMethods/PNGpackage/di}
	\label{diPackage}
	\caption{Diagramma delle classi - dependency injection}
\end{figure}


\end{document}