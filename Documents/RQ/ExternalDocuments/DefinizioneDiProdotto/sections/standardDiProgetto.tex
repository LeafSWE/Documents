\documentclass[../DefinizioneDiProdotto.tex]{subfiles}
\begin{document}
\section{Standard di progetto}

	\subsection{Standard di documentazione del codice}
		Per gli standard di documentazione del codice si fa riferimento al documento \normediprogettov.

	\subsection{Standard di denominazione di entità e relazioni}
		Per tutte le entità e le relazioni valgono gli standard di denominazione seguenti:
		\begin{itemize}
			\item per le entità definite come package\g\, classi, attributi e metodi è necessario fornire denominazioni chiare e concise;
			\item per la denominazione delle entità sono da preferire i sostantivi mentre per le relazioni i verbi;
			\item eventuali abbreviazioni sono preferibilmente da evitare nonostante siano ammesse nei casi in cui siano comprensibili e non ambigue.
			\item per le regole tipografiche sui nomi si fa riferimento al documento \normediprogettov.
		\end{itemize}	
		

	\subsection{Standard di programmazione}
		Per gli standard di programmazione si fa riferimento al documento \normediprogettov.

	\subsection{Strumenti di lavoro e procedure}
		Per gli strumenti di lavoro e le procedure per la realizzazione del progetto si fa riferimento al documento \normediprogettov.
	
\end{document}