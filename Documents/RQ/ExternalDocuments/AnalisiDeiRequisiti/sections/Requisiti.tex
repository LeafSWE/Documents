\documentclass[../AnalisiDeiRequisiti.tex]{subfiles}
\hyphenation{major}
\hyphenation{beacon}
\begin{document}
\section{Requisiti}
Di seguito sono riportati tutti i requisiti, i quali sono stati individuati dopo un'attenta analisi sulla base del capitolato, dei casi d’uso, degli incontri con il proponente e delle riunioni interne.

I requisiti, come indica il documento \normediprogettov , sono classificati in base al tipo e alla priorità, utilizzando la seguente notazione:
\begin{center}\textbf{R[X][Y][Z]}\end{center} dove:
\begin{enumerate}
	\item \textbf{X} indica l'importanza strategica del requisito. Deve assumere solo i seguenti valori:
	\begin{itemize}
		\item \textbf{Obb}: Indica un requisito obbligatorio;
		\item \textbf{Des}: Indica un requisito desiderabile;
		\item \textbf{Opz}: Indica un requisito opzionale.
	\end{itemize}
	\item \textbf{Y} indica la tipologia del requisito. Deve assumere solo i seguenti valori:
	\begin{itemize}
		\item \textbf{F}: Indica un requisito funzionale;
		\item \textbf{Q}: Indica un requisito di qualità;
		\item \textbf{P}: Indica un requisito prestazionale;
		\item \textbf{V}: Indica un requisito vincolo.
	\end{itemize}
	\item \textbf{Z} rappresenta il codice univoco di ogni requisito in forma gerarchica.
\end{enumerate}
\subsection{Tabella dei requisiti}
\begin{longtabu} to \textwidth {X X[2] X}
	\toprule
	\textbf{Requisito} & \textbf{Descrizione} & \textbf{Fonti}\\
	\midrule
	\endhead
	\arrayrulecolor{gray}
	RObbV1 & Il sistema deve prevedere l'utilizzo della tecnologia Beacon & Capitolato \\ 
	\midrule 
	RObbQ2 & Deve essere allestito un laboratorio & Capitolato \\ 
	\midrule 
	RObbQ2.1 & Deve essere individuato lo scenario funzionale da sviluppare & Capitolato \\ 
	\midrule 
	RObbQ2.2 & Deve essere definita un'area indoor che verrà coperta dal servizio & Capitolato \\ 
	\midrule 
	RObbQ2.3 & Deve essere scelta la tipologia di beacon tra quelli forniti & Capitolato \\ 
	\midrule 
	RObbQ2.4 & Devono essere scelti i componenti software & Capitolato \\ 
	\midrule 
	RObbQ2.5 & Devono essere effettuati i test di fattibilità tecnica & Capitolato \\ 
	\midrule 
	RObbQ2.6 & Devono essere documentate e motivate le scelte relative al laboratorio & Capitolato \\ 
	\midrule 
	RObbV3 & Deve essere sviluppato un prototipo software che operi all'interno dell'area indoor scelta & Capitolato \\ 
	\midrule 
	RObbV3.1 & Il prototipo deve permettere di indicare la posizione, anche approssimativa, di un utente all'interno di un edificio (IPS) & Capitolato \\ 
	\midrule 
	RObbV3.2 & Il prototipo deve permettere di fornire informazioni all'utente relative all'area mappata dal beacon (smart places) & Capitolato \\ 
	\midrule 
	RObbV3.3 & Il prototipo deve essere sviluppato per dispositivo mobile & Capitolato \\ 
	\midrule 
	RObbV3.3.1 & Il sistema operativo del dispositivo mobile deve essere Android o iOS & Capitolato \\ 
	\midrule 
	ROpzV3.3.1.1 & Il prototipo deve essere sviluppato sia per Android che per iOS & Capitolato \\ 
	\midrule 
	RObbV3.3.1.2 & La versione del sistema operativo Android deve essere compresa tra la versione 4.4 e la versione 5.1 & Riunione esterna(Verbale2016-01-18) \\ 
	\midrule 
	ROpzV3.3.1.3 & La versione del sistema operativo Android può essere superiore alla versione 5.1 & Riunione esterna(Verbale2016-01-18) \\ 
	\midrule 
	ROpzV3.3.1.4 & La versione del sistema operativo iOS deve essere compresa tra la versione 7.0 e la versione 9.0 & Riunione interna(Verbale2016-03-17) \\ 
	\midrule 
	RObbV4 & Deve essere effettuate delle sperimentazioni pratiche del prototipo sulla base delle scelte fatte & Capitolato \\ 
	\midrule 
	RObbV4.1 & Devono essere individuate le problematiche hardware & Capitolato \\ 
	\midrule 
	RObbV4.1.1 & Devono essere individuate le problematiche hardware riguardanti il dispositivo mobile & Capitolato \\ 
	\midrule 
	RObbV4.1.2 & Devono essere individuate le problematiche hardware riguardanti i beacon & Capitolato \\ 
	\midrule 
	RObbV4.2 & Devono essere individuate le problematiche software & Capitolato \\ 
	\midrule 
	RObbV4.2.1 & Devono essere individuate le problematiche software riguardanti la navigazione & Capitolato \\ 
	\midrule 
	RObbV4.3 & Devono essere individuate le problematiche lato user experience & Capitolato \\ 
	\midrule 
	RObbV4.3.1 & L'area indoor scelta deve essere presentata in modo intuitivo  & Capitolato \\ 
	\midrule 
	RObbV4.3.2 & L'applicazione deve proporre delle destinazioni verso le quali l'utente può muoversi & Capitolato \\ 
	\midrule 
	RObbV4.3.3 & L'applicazione deve presentare il percorso di navigazione in maniera intuitiva & Capitolato \\ 
	\midrule 
	RObbV4.4 & Devono essere eseguite almeno due sperimentazioni pratiche & Capitolato \\ 
	\midrule 
	RObbV4.5 & Devono essere documentate le scelte e le variazioni introdotte per ogni sperimentazione & Capitolato \\ 
	\midrule 
	RObbV4.6 & Devono essere simulate condizioni dell'ambiente il più possibili realistiche per quanto riguarda il numero di persone all'interno dell'area indoor scelta durante le sperimentazioni & Capitolato \\ 
	\midrule 
	RObbV5 & Deve essere prodotta la documentazione dello studio & Capitolato \\ 
	\midrule 
	RObbV5.1 & Devono essere documentate e motivate le scelte fatte & Capitolato \\ 
	\midrule 
	RObbV5.1.1 & Devono essere documentate e motivate le scelte fatte riguardo l'allestimento dell'impianto & Capitolato \\ 
	\midrule 
	RObbV5.1.2 & Devono essere documentate e motivate le scelte fatte riguardo il software sviluppato & Capitolato \\ 
	\midrule 
	RObbV5.1.3 & Devono essere documentate e motivate le scelte fatte riguardo l'interazione con l'utente & Capitolato \\ 
	\midrule 
	RObbV5.2 & Deve essere documentato un eventuale fallimento nella prova sperimentale & Capitolato \\ 
	\midrule 
	RObbV5.2.1 & Devono essere documentate le cause del fallimento & Capitolato \\ 
	\midrule 
	RObbV5.2.2 & Deve essere individuato un possibile scenario alternativo che eviti il verificarsi degli errori individuati & Capitolato \\ 
	\midrule 
	RObbV5.3 & Devono essere documentati eventuali punti deboli della soluzione & Capitolato \\ 
	\midrule 
	RObbV5.3.1 & Devono essere documentati eventuali punti deboli della soluzione riguardanti l'hardware & Capitolato \\ 
	\midrule 
	RObbV5.3.2 & Devono essere documentati eventuali punti deboli della soluzione riguardanti il software & Capitolato \\ 
	\midrule 
	RObbV5.3.3 & Devono essere documentati eventuali punti deboli della soluzione riguardanti l'user experience & Capitolato \\ 
	\midrule 
	RObbV5.4 & Deve essere prodotta la documentazione dell'impianto realizzato in varie versioni & Capitolato \\ 
	\midrule 
	RObbV5.4.1 & Deve essere prodotta la documentazione della mappa di distribuzione degli apparecchi beacon & Capitolato \\ 
	\midrule 
	RObbV5.4.2 & Deve essere prodotta la documentazione riguardante la densità dell'impianto & Capitolato \\ 
	\midrule 
	RObbV5.4.3 & Deve essere prodotta la documentazione riguardante le scelte nel posizionamento dei beacon & Capitolato \\ 
	\midrule 
	RObbV5.4.4 & Deve essere prodotta la documentazione riguardante le note operative dell'impianto & Capitolato \\ 
	\midrule 
	RObbV5.5 & Deve essere prodotta la documentazione relativa al percorso di ricerca e miglioramento & Capitolato \\ 
	\midrule 
	RObbV6 & I luoghi all'interno di un edificio devono essere raggruppati in insiemi disgiunti che ne indichino la categoria & Riunione interna(Verbale2016-01-07) \\ 
	\midrule 
	RObbF7 & L'applicazione deve fornire la funzionalità di navigazione & UC1 \\ 
	\midrule 
	RObbF7.1 & L'applicazione deve fornire la possibilità di ricercare una destinazione tra quelle possibili all'interno dell'edificio & UC1.1 \\ 
	\midrule 
	RDesF7.1.1 & L'applicazione deve permettere di ricercare la destinazione per nome & UC1.1.3 \\ 
	\midrule 
	RDesF7.1.1.1 & L'applicazione deve permettere di inserire il nome di una destinazione & UC1.1.3 \\ 
	\midrule 
	RObbF7.1.2 & L'applicazione deve permettere di ricercare la destinazione per categoria & UC1.1.4 \\ 
	\midrule 
	RObbF7.1.2.1 & L'applicazione deve permettere di selezionare una categoria di POI tra quelle disponibili per il dato edificio. Le categorie dipendono fortemente dall'edificio e possono prevedere per esempio servizi igienici, aule, uffici & UC1.1.4 \\ 
	\midrule 
	RObbF7.1.3 & L'applicazione deve permettere di selezionare un'opzione tra i risultati della ricerca & UC1.1.1 \\ 
	\midrule 
	RObbF7.1.4 & L'applicazione deve segnalare un messaggio d'errore esplicito nel caso in cui venga inserita una destinazione non prevista dal sistema & UC1.1.5 \\ 
	\midrule 
	RObbF7.2 & L'applicazione deve permettere di confermare la destinazione scelta & UC1.1.2 \\ 
	\midrule 
	RObbF7.3 & Dato un POI di partenza A ed un POI da raggiungere B, l'applicazione deve calcolare un percorso da A a B & UC1.2 \\ 
	\midrule 
	RDesF7.3.1 & Data un POI di partenza A ed un POI da raggiungere B, l'applicazione deve calcolare un percorso da A a B secondo le preferenze dell'utente & UC1.2 \par UC3.2.2 \\ 
	\midrule 
	RDesF7.3.1.1 & Data una posizione di partenza A ed una destinazione da raggiungere B, l'applicazione deve calcolare un percorso da A a B scegliendo il percorso con meno barriere architettoniche & UC1.2 \par UC3.2.2.1 \\ 
	\midrule 
	RDesF7.3.1.2 & Data una posizione di partenza A ed una destinazione da raggiungere B, l'applicazione deve calcolare un percorso da A a B scegliendo il percorso con meno ascensori & UC1.2 \par UC3.2.2.2 \\ 
	\midrule 
	RDesF7.3.1.3 & Data una posizione di partenza A ed una destinazione da raggiungere B, l'applicazione deve calcolare un percorso da A a B scegliendo il percorso più veloce & UC1.2 \par UC3.2.2.3 \\ 
	\midrule 
	RObbF7.4 & L'applicazione deve permettere di iniziare la navigazione & UC1 \\ 
	\midrule 
	RObbF7.4.1 & L'applicazione deve richiedere l'attivazione dei sensori & UC1 \\ 
	\midrule 
	RObbF7.4.1.1 & L'applicazione deve richiedere l'attivazione della modalità di geolocalizzazione & UC1 \\ 
	\midrule 
	RObbF7.4.1.2 & L'applicazione deve richiedere l'attivazione del sensore Bluetooth & UC1 \\ 
	\midrule 
	RObbF7.4.1.3 & L'applicazione deve richiedere l'attivazione del sensore GPS, se il dispositivo Android ha una versione del sistema operativo uguale o superiore alla 6.0 & Riunione esterna(Verbale2016-01-18) \\ 
	\midrule 
	RObbF7.4.2 & L'applicazione deve fornire all'utente le indicazioni utili alla navigazione, a partire dal POI in cui esso si trova & UC1.5 \\ 
	\midrule 
	RObbF7.4.2.1 & L'applicazione deve fornire all'utente indicazioni basilari per raggiungere la destinazione scelta & UC1.5 \\ 
	\midrule 
	RDesF7.4.2.2 & L'applicazione deve fornire all'utente una lista contenente le indicazioni utili per raggiungere la destinazione scelta percorrendo tutti i POI che compongono il percorso previsto & UC1.5 \\ 
	\midrule 
	RDesF7.4.2.3 & L'applicazione deve avvisare l'utente nel caso in cui rilevi un beacon diverso da quelli previsti nel percorso calcolato & UC1.5 \par UC1.6 \\ 
	\midrule 
	ROpzF7.4.2.4 & L'applicazione deve fornire all'utente indicazioni per raggiungere il prossimo POI & UC1.5 \\ 
	\midrule 
	ROpzF7.4.2.5 & L'applicazione deve avvisare l'utente nel caso in cui si allontani dal percorso ideale per raggiungere il prossimo beacon & UC1.5 \\ 
	\midrule 
	ROpzF7.4.2.6 & L'applicazione deve avvisare l'utente con un messaggio d'errore se il sistema non rileva beacon nell'area in cui l'utente si trova & UC1.7 \\ 
	\midrule 
	RDesF7.4.3 & L'applicazione deve fornire le indicazioni dettagliate che lo possano aiutare durante la navigazione & UC1.4 \\ 
	\midrule 
	RDesF7.4.3.1 & L'applicazione deve fornire la foto del prossimo POI previsto dal percorso & UC1.4.1 \\ 
	\midrule 
	ROpzF7.4.3.2 & L'applicazione deve fornire l'indicazione da seguire in maniera testuale, specificando più indicazioni e punti di riferimento possibili & UC1.4.2 \\ 
	\midrule 
	ROpzF7.4.3.3 & L'applicazione deve fornire la lista di tutte le prossime indicazioni da seguire per raggiungere la destinazione scelta & UC1.4.3 \\ 
	\midrule 
	RDesF7.4.3.4 & L'applicazione deve avvisare l'utente con un errore esplicativo nel caso in cui durante una navigazione si cerchi di accedere alle foto del prossimo POI e la connessione Internet non sia attiva & UC1.4.4 \\ 
	\midrule 
	RObbF7.4.4 & L'applicazione deve fornire la possibilità di confermare l'avvio della navigazione & UC1.2.1 \\ 
	\midrule 
	RObbF7.5 & L'applicazione deve permettere di interrompere la navigazione in corso & UC1.3 \\ 
	\midrule 
	RObbF7.6 & L'applicazione deve avvisare l'utente con un messaggio d'errore esplicito nel caso in cui si avvii una navigazione con connessione Internet del dispositivo non attiva & UC7 \\ 
	\midrule 
	RObbF7.7 & L'applicazione deve avvisare l'utente con un messaggio d'errore esplicito nel caso in cui si avvii la navigazione in un edificio mappato e la mappa dell'edificio installata nel dispositivo differisce dall'ultima versione online & UC8 \\ 
	\midrule 
	RObbF7.8 & L'applicazione deve avvisare l'utente con un messaggio d'errore esplicito nel caso in cui si rilevi un beacon in un edificio mappato e la mappa dell'edificio non sia installata nel dispositivo & UC9 \\ 
	\midrule 
	RObbF8 & L'applicazione deve permettere di interagire coi beacon & Capitolato \par UC1 \par UC2 \par UC6 \\ 
	\midrule 
	RObbF8.1 & L'applicazione deve permettere di rilevare l'identificativo (UUID, major, minor) di un beacon & Capitolato \par UC1 \par UC2 \par UC6 \\ 
	\midrule 
	RObbF8.1.1 & L'applicazione deve, rilevato l'identificativo di un beacon, riuscire a reperire le informazioni riguardanti il POI a cui è associato quel beacon & Capitolato \par UC1 \par UC2 \\ 
	\midrule 
	RObbF8.1.2 & L'applicazione deve, rilevato l'identificativo di un beacon, riuscire a reperire le informazioni riguardanti i POI circostanti quel beacon & UC2.3 \par UC2.4 \\ 
	\midrule 
	RObbF8.2 & L'applicazione deve permettere di rilevare il livello di potenza del segnale dei beacon rilevati
	& UC6 \\ 
	\midrule 
	RObbF8.3 & L'applicazione deve permettere di rilevare il livello di batteria dei beacon rilevati & UC6 \\ 
	\midrule 
	RObbF8.4 & L'applicazione deve permettere di rilevare la distanza approssimativa dell'utente dai beacon rilevati & UC6 \\ 
	\midrule 
	RObbF8.5 & L'applicazione deve permettere di rilevare il formato dei beacon rilevati & UC6 \\ 
	\midrule 
	RObbF8.6 & L'applicazione deve permettere di rilevare l'area coperta dal segnale dei beacon rilevati & UC6 \\ 
	\midrule 
	RObbF9 & L'applicazione deve fornire informazioni sull'edificio mappato in cui si trova l'utente & UC2.1 \\ 
	\midrule 
	RObbF9.1 & L'applicazione deve fornire un elenco dei POI raggiungibili all'interno dell'edificio in cui si trova l'utente & Capitolato \par UC1.1 \par UC2.2 \\ 
	\midrule 
	ROpzF9.2 & L'applicazione deve fornire informazioni su tutti i luoghi interni all'edificio in cui si trova l'utente & UC2.2 \par UC2.4 \\ 
	\midrule 
	ROpzF9.2.1 & L'applicazione deve fornire informazioni relative ad uno specifico POI & UC2.4 \\ 
	\midrule 
	ROpzF9.2.2 & L'applicazione deve fornire un elenco dei POI appartenenti all'edificio in cui si trova l'utente e rilevati alla posizione dell'utente & UC2.3 \\ 
	\midrule 
	RObbF9.2.3 & L'applicazione deve fornire il nome di uno specifico POI & UC2.4.1 \\ 
	\midrule 
	RObbF9.2.4 & L'applicazione deve fornire la descrizione di uno specifico POI & UC2.4.2 \\ 
	\midrule 
	RObbF9.3 & L'applicazione deve fornire l'orario di apertura dell'edificio in cui si trova l'utente & UC2.1.1 \\ 
	\midrule 
	RObbF9.4 & L'applicazione deve fornire l'indirizzo dell'edificio in cui si trova l'utente & UC2.1.2 \\ 
	\midrule 
	RObbF9.5 & L'applicazione deve fornire il nome dell'edificio in cui si trova l'utente & UC2.1.3 \\ 
	\midrule 
	RObbF9.6 & L'applicazione deve fornire la descrizione dell'edificio in cui si trova l'utente & UC2.1.4 \\ 
	\midrule 
	RObbQ9.7 & Ad ogni edificio dotato del sistema CLIPS deve essere associato un nome & UC2.1.3 \\ 
	\midrule 
	RObbQ9.8 & Ad ogni POI interno ad un edificio dotato del sistema CLIPS deve essere associato un nome & UC2.4.1 \\ 
	\midrule 
	RObbF9.9 & L'applicazione deve avvisare l'utente con un messaggio d'errore esplicito nel caso in cui si acceda alle informazioni dell'edificio mappato in cui si trova con connessione Internet del dispositivo non attiva & UC7 \\ 
	\midrule 
	RObbF9.10 & L'applicazione deve avvisare l'utente con un messaggio d'errore esplicito nel caso in cui si acceda alle informazioni dell'edificio mappato in cui si trova e la mappa dell'edificio installata nel dispositivo differisce dall'ultima versione online & UC8 \\ 
	\midrule 
	ROpzF10 & Deve essere possibile gestire gli aspetti relativi all'applicazione & UC3 \\ 
	\midrule 
	ROpzF10.1 & L'applicazione deve permettere di gestire le preferenze navigazione & UC3.2 \\ 
	\midrule 
	ROpzF10.1.1 & L'applicazione deve permettere di impostare le preferenze di percorso & UC3.2.2 \\ 
	\midrule 
	ROpzF10.1.1.1 & L'applicazione deve permettere di scegliere il percorso col minor numero di barriere architettoniche & UC3.2.2.1 \\ 
	\midrule 
	ROpzF10.1.1.2 & L'applicazione deve permettere di scegliere il percorso che presenta il minor numero di ascensori & UC3.2.2.2 \\ 
	\midrule 
	RObbF10.1.1.3 & L'applicazione deve permettere di scegliere il percorso più veloce & UC3.2.2.3 \\ 
	\midrule 
	RDesF10.1.2 & L'applicazione deve permettere di impostare le preferenze di fruizione delle indicazioni & UC3.2.1 \\ 
	\midrule 
	RObbF10.1.2.1 & L'applicazione deve fornire informazioni testuali & UC1.2 \par UC1.5 \\ 
	\midrule 
	RDesF10.1.2.2 & L'applicazione deve permettere all'utente di attivare le indicazioni vocali per la prossima navigazione & UC3.2.1.1 \\ 
	\midrule 
	RDesF10.1.2.3 & L'applicazione deve permettere all'utente di attivare le indicazioni sonore per la prossima navigazione & UC3.2.1.2 \\ 
	\midrule 
	RDesF10.1.2.4 & L'applicazione deve permettere all'utente di disattivare le indicazioni vocali per la prossima navigazione & UC3.2.1.3 \\ 
	\midrule 
	RDesF10.1.2.5 & L'applicazione deve permettere all'utente di disattivare le indicazioni sonore per la prossima navigazione & UC3.2.1.4 \\ 
	\midrule 
	RDesF10.2 & L'applicazione deve permettere di gestire le mappe & UC3.1 \\ 
	\midrule 
	RDesF10.2.1 & L'applicazione deve permettere di gestire le mappe installate nel dispositivo & UC3.1.1 \\ 
	\midrule 
	RDesF10.2.1.1 & L'applicazione deve permettere di reperire le mappe installate nel dispositivo & UC3.1.1.1 \\ 
	\midrule 
	RDesF10.2.1.2 & L'applicazione deve permettere di aggiornare una mappa presente sul proprio dispositivo che richieda un aggiornamento & UC3.1.1.2 \\ 
	\midrule 
	RDesF10.2.1.3 & L'applicazione deve permettere di rimuovere una mappa installata nel dispositivo & UC3.1.1.3 \\ 
	\midrule 
	RDesF10.2.1.4 & L'applicazione deve permettere di reperire i dettagli informativi di una mappa installata nel dispositivo & UC3.1.1.4 \\ 
	\midrule 
	RDesF10.2.1.4.1 & L'applicazione deve permettere l'accesso al nome di ogni singola mappa installata nel dispositivo & UC3.1.1.4 \\ 
	\midrule 
	RDesF10.2.1.4.2 & L'applicazione deve permettere l'accesso all'indirizzo dell'edificio associato ad ogni singola mappa installata nel dispositivo & UC3.1.1.4 \\ 
	\midrule 
	RDesF10.2.1.4.3 & L'applicazione deve permettere l'accesso alla descrizione dell'edificio associato ad ogni singola mappa installata nel dispositivo & UC3.1.1.4 \\ 
	\midrule 
	RDesF10.2.1.4.4 & L'applicazione deve permettere l'accesso alla dimensione in megabyte di ogni singola mappa installata nel dispositivo & UC3.1.1.4 \\ 
	\midrule 
	RDesF10.2.1.4.5 & L'applicazione deve permettere l'accesso alla versione di ogni singola mappa installata nel dispositivo & UC3.1.1.4 \\ 
	\midrule 
	RDesF10.2.2 & L'applicazione deve permettere di gestire le mappe non installate nel sistema & UC3.1.2 \\ 
	\midrule 
	RDesF10.2.2.1 & L'applicazione deve permettere di ricercare una mappa per nome tra le mappe disponibili online & UC3.1.2.1 \\ 
	\midrule 
	RDesF10.2.2.1.1 & L'applicazione deve permettere all'utente l'inserimento di un possibile nome di una mappa & UC3.1.2.1 \\ 
	\midrule 
	RDesF10.2.2.2 & L'applicazione deve permettere l'installazione di una nuova mappa disponibile online non precedentemente installata & UC3.1.2.2 \\ 
	\midrule 
	RDesF10.2.2.3 & L'applicazione deve permettere di reperire i dettagli informativi di una mappa non installata nel dispositivo & UC3.1.2.3 \\ 
	\midrule 
	RDesF10.2.2.3.1 & L'applicazione deve permettere l'accesso al nome di una mappa non installata nel dispositivo & UC3.1.2.3 \\ 
	\midrule 
	RDesF10.2.2.3.2 & L'applicazione deve permettere l'accesso all'indirizzo dell'edificio associato di una mappa non installata nel dispositivo & UC3.1.2.3 \\ 
	\midrule 
	RDesF10.2.2.3.3 & L'applicazione deve permettere l'accesso alla descrizione dell'edificio associato di una mappa non installata nel dispositivo & UC3.1.2.3 \\ 
	\midrule 
	RDesF10.2.2.3.4 & L'applicazione deve permettere l'accesso alla dimensione in megabyte di una mappa non installata nel dispositivo & UC3.1.2.3 \\ 
	\midrule 
	RDesF10.2.2.3.5 & L'applicazione deve permettere l'accesso alla versione di una mappa non installata nel dispositivo & UC3.1.2.3 \\ 
	\midrule 
	RDesF10.2.2.4 & L'applicazione deve segnalare che la mappa non è stata trovata nel caso in cui la ricerca per nome non abbia trovato corrispondenze & UC3.1.2.4 \\ 
	\midrule 
	RDesF10.2.3 & L'applicazione deve permettere di recuperare una mappa collegandosi ad un server & Riunione esterna(Verbale2016-01-18) \par UC3.1.2.2 \\ 
	\midrule 
	RObbF10.3 & L'applicazione deve permettere di sbloccare le funzionalità di sviluppatore & UC4 \\ 
	\midrule 
	RObbF10.3.1 & L'applicazione deve permettere di inserire un codice per sbloccare le funzionalità di sviluppatore & UC4.1 \\ 
	\midrule 
	RObbF10.3.2 & L'applicazione deve permettere di confermare il codice inserito & UC4.2 \\ 
	\midrule 
	RObbF10.3.3 & L'applicazione deve segnalare un messaggio d'errore esplicito nel caso in cui il codice inserito non sia corretto & UC4.3 \\ 
	\midrule 
	ROpzF11 & L'applicazione deve fornire una guida che spieghi l'utilizzo della stessa & UC5 \\ 
	\midrule 
	RObbF12 & L'applicazione deve permettere di  accedere alle informazioni dei beacon circostanti rilevati & UC6.1 \\ 
	\midrule 
	RObbF12.1 & L'applicazione deve permettere accedere all'identificativo UUID di un beacon & UC6.1.1 \\ 
	\midrule 
	RObbF12.2 & L'applicazione deve permettere di accedere al livello di potenza del segnale dei beacon rilevati & UC6.1.4 \\ 
	\midrule 
	RObbF12.3 & L'applicazione deve permettere di gestire i log delle informazioni riguardanti i beacon rilevati & UC6.2 \\ 
	\midrule 
	RDesF12.3.1 & L'applicazione deve permettere di interrompere un log & UC6.2.2 \\ 
	\midrule 
	RDesF12.3.2 & L'applicazione deve permettere di avviare un nuovo log & UC6.2.1 \\ 
	\midrule 
	RDesF12.3.3 & L'applicazione deve permettere di salvare un log & UC6.2.2 \\ 
	\midrule 
	RDesF12.3.4 & L'applicazione deve permettere di accedere ad un log salvato & UC6.2.4 \\ 
	\midrule 
	RDesF12.3.5 & L'applicazione deve permettere di eliminare un log salvato & UC6.2.3 \\ 
	\midrule 
	ROpzF12.4 & L'applicazione deve permettere di accedere al livello di batteria dei beacon rilevati & UC6.1.5 \\ 
	\midrule 
	RObbF12.5 & L'applicazione deve permettere di accedere alla distanza approssimativa dell'utente dai beacon rilevati & UC6.1.6 \\ 
	\midrule 
	RObbF12.6 & L'applicazione deve permettere di accedere al formato dei beacon rilevati & UC6.1.7 \\ 
	\midrule 
	RObbF12.7 & L'applicazione deve permettere di accedere alle indicazioni sull'area coperta dai beacon rilevati & UC6.1.8 \\ 
	\midrule 
	RObbF12.8 & L'applicazione deve permettere di accedere all'identificativo Major di un beacon & UC6.1.2 \\ 
	\midrule 
	RObbF12.9 & L'applicazione deve permettere di accedere all'identificativo Minor di un beacon & UC6.1.3 \\ 
	\midrule 
	ROpzQ13 & Devono essere utilizzate le issue di GitHub per la segnalazione dei bug & Riunione interna(Verbale2016-01-07) \\ 
	\midrule 
	RDesP14 & L'applicazione deve rilevare il beacon e le sue informazioni in un tempo minore di 15 secondi & Riunione interna(Verbale2016-01-13) \\ 
	\midrule 
	ROpzP15 & L'applicazione deve rilevare il beacon e le sue informazioni in un tempo minore di 5 secondi & Riunione interna(Verbale2016-01-13) \\ 
	\arrayrulecolor{black}
	\bottomrule
	\caption{Tabella Requisiti / Fonti} \\
\end{longtabu}


	\newpage
	\subsection{Tracciamento fonti-requisiti}
\begin{longtabu} to \textwidth {X X[2] X}
	\toprule
	\textbf{Codice} & \textbf{Dettaglio} & \textbf{Requisiti}\\
	\midrule
	\endhead
	\arrayrulecolor{gray}
	& Capitolato & RObbV1 \par RObbQ2 \par RObbQ2.1 \par RObbQ2.2 \par RObbQ2.3 \par RObbQ2.4 \par RObbQ2.5 \par RObbQ2.6 \par RObbV3 \par RObbV3.1 \par RObbV3.2 \par RObbV3.3 \par RObbV3.3.1 \par ROpzV3.3.1.1 \par RObbV4 \par RObbV4.1 \par RObbV4.1.1 \par RObbV4.1.2 \par RObbV4.2 \par RObbV4.2.1 \par RObbV4.3 \par RObbV4.3.1 \par RObbV4.3.2 \par RObbV4.3.3  RObbV4.4 \par RObbV4.5 \par RObbV4.6 RObbV5 \par RObbV5.1 \par RObbV5.1.1 \par RObbV5.1.2 \par RObbV5.1.3 \par RObbV5.2 \par RObbV5.2.1 \par RObbV5.2.2 \\
	 & Capitolato & RObbV5.3 \par RObbV5.3.1 \par RObbV5.3.2 \par RObbV5.3.3 \par RObbV5.4 \par RObbV5.4.1 \par RObbV5.4.2 \par RObbV5.4.3 \par RObbV5.4.4 \par RObbV5.5 \par RObbF8 \par RObbF8.1 \par RObbF8.1.1 \par RObbF9.1 \\ 
	\midrule 
	& Riunione interna(Verbale2016-01-07) & RObbV6 \par ROpzQ13 \\ 
	\midrule 
	& Riunione esterna(Verbale2016-01-18) & RObbV3.3.1.2 \par ROpzV3.3.1.3 \par RObbF7.4.1.3 \par RDesF10.2.3 \\ 
	\midrule 
	& Riunione interna(Verbale2016-01-13) & RDesP14 \par ROpzP15 \\ 
	\midrule 
	& Riunione interna(Verbale2016-03-17) & ROpzV3.3.1.4 \\ 
	\midrule 
	UCG & UCG - Utilizzo generale del prototipo &  \\ 
	\midrule 
	UC1 & UC1 - Navigazione & RObbF7 \par RObbF7.4 \par RObbF7.4.1 \par RObbF7.4.1.1 \par RObbF7.4.1.2 \par RObbF8 \par RObbF8.1 \par RObbF8.1.1 \\ 
	\midrule 
	UC1.1 & UC1.1 - Inserimento destinazione & RObbF7.1 \par RObbF9.1 \\ 
	\midrule 
	UC1.1.1 & UC1.1.1 - Scelta destinazione & RObbF7.1.3 \\ 
	\midrule 
	UC1.1.2 & UC1.1.2 - Conferma scelta & RObbF7.2 \\ 
	\midrule 
	UC1.1.3 & UC1.1.3 - Ricerca per nome & RDesF7.1.1 \par RDesF7.1.1.1 \\ 
	\midrule 
	UC1.1.4 & UC1.1.4 - Ricerca per categoria & RObbF7.1.2 \par RObbF7.1.2.1 \\ 
	\midrule 
	UC1.1.5 & UC1.1.5 - Visualizzazione errore destinazione sconosciuta & RObbF7.1.4 \\ 
	\midrule 
	UC1.2 & UC1.2 - Avvio navigazione & RObbF7.3 \par RDesF7.3.1 \par RObbF10.1.2.1 \par RDesF7.3.1.1 \par RDesF7.3.1.2 \par RDesF7.3.1.3 \\ 
	\midrule 
	UC1.2.1 & UC1.2.1 - Conferma avvio navigazione & RObbF7.4.4 \\ 
	\midrule 
	UC1.3 & UC1.3 - Interruzione navigazione & RObbF7.5 \\ 
	\midrule 
	UC1.4 & UC1.4 - Accesso a maggiori indicazioni & RDesF7.4.3 \\ 
	\midrule 
	UC1.4.1 & UC1.4.1 - Accesso foto della prossima area & RDesF7.4.3.1 \\ 
	\midrule 
	UC1.4.2 & UC1.4.2 - Accesso indicazione testuale estesa & ROpzF7.4.3.2 \\ 
	\midrule 
	UC1.4.3 & UC1.4.3 - Accesso lista indicazioni & ROpzF7.4.3.3 \\ 
	\midrule 
	UC1.4.4 & UC1.4.4 - Segnalazione errore connessione Internet assente & RDesF7.4.3.4 \\ 
	\midrule 
	UC1.5 & UC1.5 - Visualizzazione indicazioni & RObbF7.4.2 \par RObbF7.4.2.1 \par RDesF7.4.2.2 \par RDesF7.4.2.3 \par ROpzF7.4.2.4 \par ROpzF7.4.2.5 \par RObbF10.1.2.1 \\ 
	\midrule 
	UC1.6 & UC1.6 - Visualizzazione errore percorso & RDesF7.4.2.3 \\ 
	\midrule 
	UC1.7 & UC1.7 - Visualizzazione errore beacon & ROpzF7.4.2.6 \\ 
	\midrule 
	UC2 & UC2 - Accesso alle informazioni & RObbF8 \par RObbF8.1 \par RObbF8.1.1 \\ 
	\midrule 
	UC2.1 & UC2.1 - Accesso alle informazioni dell'edificio & RObbF9 \\ 
	\midrule 
	UC2.1.1 & UC2.1.1 - Accesso orari edificio & RObbF9.3 \\ 
	\midrule 
	UC2.1.2 & UC2.1.2 - Accesso indirizzo edificio & RObbF9.4 \\ 
	\midrule 
	UC2.1.3 & UC2.1.3 - Accesso nome edificio & RObbF9.5 \par RObbQ9.7 \\ 
	\midrule 
	UC2.1.4 & UC2.1.4 - Accesso descrizione edificio & RObbF9.6 \\ 
	\midrule 
	UC2.2 & UC2.2 - Accesso lista POI edificio & RObbF9.1 \par ROpzF9.2 \\ 
	\midrule 
	UC2.3 & UC2.3 - Accesso lista POI circostanti & RObbF8.1.2 \par ROpzF9.2.2 \\ 
	\midrule 
	UC2.4 & UC2.4 - Accesso informazione POI & RObbF8.1.2 \par ROpzF9.2 \par ROpzF9.2.1 \\ 
	\midrule 
	UC2.4.1 & UC2.4.1 - Accesso nome POI & RObbF9.2.3 \par RObbQ9.8 \\ 
	\midrule 
	UC2.4.2 & UC2.4.2 - Accesso descrizione POI & RObbF9.2.4 \\ 
	\midrule 
	UC3 & UC3 - Gestione dell'applicazione & ROpzF10 \\ 
	\midrule 
	UC3.1 & UC3.1 - Gestione mappe & RDesF10.2 \\ 
	\midrule 
	UC3.1.1 & UC3.1.1 - Gestione mappe installate & RDesF10.2.1 \\ 
	\midrule 
	UC3.1.1.1 & UC3.1.1.1 - Reperimento mappe installate & RDesF10.2.1.1 \\ 
	\midrule 
	UC3.1.1.2 & UC3.1.1.2 - Aggiornamento mappa & RDesF10.2.1.2 \\ 
	\midrule 
	UC3.1.1.3 & UC3.1.1.3 - Rimozione mappa & RDesF10.2.1.3 \\ 
	\midrule 
	UC3.1.1.4 & UC3.1.1.4 - Reperimento dettagli mappa installata & RDesF10.2.1.4 \par RDesF10.2.1.4.1 \par RDesF10.2.1.4.2 \par RDesF10.2.1.4.3 \par RDesF10.2.1.4.4 \par RDesF10.2.1.4.5 \\ 
	\midrule 
	UC3.1.2 & UC3.1.2 - Gestione mappe non installate & RDesF10.2.2 \\ 
	\midrule 
	UC3.1.2.1 & UC3.1.2.1 - Ricerca mappa per nome & RDesF10.2.2.1 \par RDesF10.2.2.1.1 \\ 
	\midrule 
	UC3.1.2.2 & UC3.1.2.2 - Installazione nuova mappa & RDesF10.2.3 \par RDesF10.2.2.2 \\ 
	\midrule 
	UC3.1.2.3 & UC3.1.2.3 - Reperimento dettagli mappa non installata & RDesF10.2.2.3.1 \par RDesF10.2.2.3.2 \par RDesF10.2.2.3.3 \par RDesF10.2.2.3.4 \par RDesF10.2.2.3.5 \par RDesF10.2.2.3 \\ 
	\midrule 
	UC3.1.2.4 & UC3.1.2.4 - Nessuna mappa trovata & RDesF10.2.2.4 \\ 
	\midrule 
	UC3.2 & UC3.2 - Gestione preferenze navigazione & ROpzF10.1 \\ 
	\midrule 
	UC3.2.1 & UC3.2.1 - Gestione fruizione indicazioni & RDesF10.1.2 \\ 
	\midrule 
	UC3.2.1.1 & UC3.2.1.1 - Attivazione indicazioni vocali & RDesF10.1.2.2 \\ 
	\midrule 
	UC3.2.1.2 & UC3.2.1.2 - Attivazione indicazioni sonore & RDesF10.1.2.3 \\ 
	\midrule 
	UC3.2.1.3 & UC3.2.1.3 - Disattivazione indicazioni vocali & RDesF10.1.2.4 \\ 
	\midrule 
	UC3.2.1.4 & UC3.2.1.4 - Disattivazione indicazioni sonore & RDesF10.1.2.5 \\ 
	\midrule 
	UC3.2.2 & UC3.2.2 - Gestione preferenze percorso & RDesF7.3.1 \par ROpzF10.1.1 \\ 
	\midrule 
	UC3.2.2.1 & UC3.2.2.1 - Scelta percorso più accessibile & ROpzF10.1.1.1 \par RDesF7.3.1.1 \\ 
	\midrule 
	UC3.2.2.2 & UC3.2.2.2 - Scelta evitare ascensori & ROpzF10.1.1.2 \par RDesF7.3.1.2 \\ 
	\midrule 
	UC3.2.2.3 & UC3.2.2.3 - Scelta percorso più veloce & RObbF10.1.1.3 \par RDesF7.3.1.3 \\ 
	\midrule 
	UC4 & UC4 - Attivazione funzionalità sviluppatore & RObbF10.3 \\ 
	\midrule 
	UC4.1 & UC4.1 - Inserimento codice sviluppatore & RObbF10.3.1 \\ 
	\midrule 
	UC4.2 & UC4.2 - Conferma codice & RObbF10.3.2 \\ 
	\midrule 
	UC4.3 & UC4.3 - Visualizzazione errore codice & RObbF10.3.3 \\ 
	\midrule 
	UC5 & UC5 - Accesso alla guida & ROpzF11 \\ 
	\midrule 
	UC6 & UC6 - Accesso funzionalità sviluppatore & RObbF8 \par RObbF8.1 \par RObbF8.2 \par RObbF8.3 \par RObbF8.4 \par RObbF8.5 \par RObbF8.6 \\ 
	\midrule 
	UC6.1 & UC6.1 - Accesso informazioni beacon circostanti & RObbF12 \\ 
	\midrule 
	UC6.1.1 & UC6.1.1 - Accesso UUID & RObbF12.1 \\ 
	\midrule 
	UC6.1.2 & UC6.1.2 - Accesso Major & RObbF12.8 \\ 
	\midrule 
	UC6.1.3 & UC6.1.3 - Accesso Minor & RObbF12.9 \\ 
	\midrule 
	UC6.1.4 & UC6.1.4 - Accesso livello di potenza del segnale & RObbF12.2 \\ 
	\midrule 
	UC6.1.5 & UC6.1.5 - Accesso livello batteria & ROpzF12.4 \\ 
	\midrule 
	UC6.1.6 & UC6.1.6 - Accesso distanza approssimativa & RObbF12.5 \\ 
	\midrule 
	UC6.1.7 & UC6.1.7 - Accesso formato del beacon & RObbF12.6 \\ 
	\midrule 
	UC6.1.8 & UC6.1.8 - Accesso area coperta da beacon & RObbF12.7 \\ 
	\midrule 
	UC6.2 & UC6.2 - Gestione log & RObbF12.3 \\ 
	\midrule 
	UC6.2.1 & UC6.2.1 - Avvio nuovo log & RDesF12.3.2 \\ 
	\midrule 
	UC6.2.2 & UC6.2.2 - Interruzione log & RDesF12.3.1 \par RDesF12.3.3 \\ 
	\midrule 
	UC6.2.3 & UC6.2.3 - Rimozione log & RDesF12.3.5 \\ 
	\midrule 
	UC6.2.4 & UC6.2.4 - Accesso log salvato & RDesF12.3.4 \\ 
	\midrule 
	UC7 & UC7 - Segnalazione errore nessuna connessione Internet & RObbF7.6 \par RObbF9.9 \\ 
	\midrule 
	UC8 & UC8 - Segnalazione errore mappa non aggiornata & RObbF7.7 \par RObbF9.10 \\ 
	\midrule 
	UC9 & UC9 - Segnalazione errore mappa non installata & RObbF7.8 \\ 
	\arrayrulecolor{black}
	\bottomrule
	\caption{Tabella Fonti / Requisiti} \\
\end{longtabu}
\newpage
	\subsection{Riepilogo}
\begin{longtabu} to \textwidth {X X X X}
	\toprule
	\textbf{Categoria} & \textbf{Obbligatorio} & \textbf{Opzionale} & \textbf{Desiderabile}\\
	\midrule
	\endhead
	\arrayrulecolor{gray}
	Funzionale & 54 & 15 & 44 \\ 
	\midrule 
	Qualità & 9 & 1 & 0 \\ 
	\midrule 
	Prestazionali & 0 & 1 & 1 \\ 
	\midrule 
	Vincolo & 39 & 3 & 0 \\ 
	\arrayrulecolor{black}
	\bottomrule
	\caption{Riepilogo requisiti} \\
\end{longtabu}

\end{document}