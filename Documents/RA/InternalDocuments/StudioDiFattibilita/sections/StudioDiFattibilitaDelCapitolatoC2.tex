\documentclass[../StudioDiFattibilita.tex]{subfiles}
\begin{document}
\section{Studio di fattibilità del capitolato C2}
	\subsection{Descrizione del capitolato}
	Il capitolato\g\ scelto è denominato \progetto, acronimo di "Communication \& Localization with Indoor Positioning Systems", ed è stato presentato da Miriade S.p.A.
	Il progetto ha come obiettivo la ricerca e la sperimentazione di nuovi scenari per l'implementazione della navigazione indoor\g\ applicata a più ambiti.
	In questo senso, il proponente non desidera esplorare uno scenario di proximity marketing\g, già largamente diffuso, ma è interessato all'esplorazione di nuove possibilità di interazione tra un ambiente "opportunamente cablato" e la popolazione di tale ambiente, attraverso un software\g\ installato sul sistema operativo Android\g\ o iOS\g\ (scelta lasciata al gruppo).
	\subsection{Studio del dominio}
	Per sviluppare il capitolato\g\ in esame occorre comprendere l'ambito in cui l'applicazione verrà utilizzata e le tecnologie che bisognerà utilizzare per realizzarla. Si descrivono di seguito il dominio applicativo e quello tecnologico.
		\subsubsection{Dominio applicativo}
		Il software\g\ permetterà all'utente di spostarsi all'interno di una struttura a lui sconosciuta senza problemi guidandolo fino alla sua destinazione, a patto che la suddetta struttura sia mappata da sensori beacon\g; ciò permetterà all'utente di interagire con un edificio in maniera totalmente innovativa e renderà l'edificio stesso più accessibile a chi non lo ha mai visitato in precedenza.
		\subsubsection{Dominio tecnologico}
		Per l'implementazione del prodotto\g\ richiesto, il gruppo andrà ad utilizzare le seguenti tecnologie:
			\begin{itemize}
				\item tecnologia Beacon\g;
				\item tecnologia BLE\g\ - Bluetooth\g\ Low Energy (definita nelle Specifiche Bluetooth\g\ 
4.0);
				\item il sistema operativo Android\g\ per la creazione di un'applicazione mobile;
				\item linguaggio Java\g.
			\end{itemize}
	\subsection{Valutazione del capitolato}
		\subsubsection{Motivi della scelta}
		Il capitolato\g\ C2 è stato scelto perché le tecnologie usate e il dominio di applicazione risultavano molto interessanti al gruppo.
		Inoltre, il team\g\ ha ritenuto positiva l'acquisizione di conoscenze riguardanti la tecnologia Beacon\g: tecnologia nata nel 2011, in continuo sviluppo e con molto potenziale a livello di mercato.
		\subsubsection{Potenziali criticità}
		Le criticità sono state rilevate soprattutto nel campo della navigazione, attualmente non implementata per problemi relativi alla tecnologia beacon\g\ ed al suo uso, e sono le seguenti:
			\begin{itemize}
				\item variabilità del segnale;
				\item i tempi di aggiornamento del beacon\g\ possono richiedere anche 30 secondi;
				\item interferenze tra beacon\g\ vicini;
				\item problemi derivati dalla struttura in cui sono posizionati (Esempio: muri troppo spessi, beacon\g\ nella stessa posizione ma su piani differenti che rischiano di essere confusi tra loro);
				\item problemi derivanti dal sovraffollamento del luogo in cui è posizionato il beacon\g;
				\item inaffidabilità della posizione segnalata;
				\item necessità di avere sul dispositivo il sensore di localizzazione\g\ e il sensore Bluetooth\g\ sempre accesi.
			\end{itemize}
		\subsubsection{Individuazione dei rischi}
		I rischi rilevati sono stati trattati nel \pianodiprogettov.
		\subsubsection{Aspetti di mercato}
		Il prodotto\g\ ha l'obiettivo di rivolgersi ad un ampio numero di utenti ed attualmente manca sul mercato, quindi ha alta probabilità di avere successo e di offrire un contributo importante alla società.
	\subsection{Stima di fattibilità}
	Il gruppo \leaf\ in base allo studio effettuato si prefigge l'obiettivo di portare a termine il prodotto\g\ entro le scadenze prefissate e i costi stimati.
	Inoltre, il gruppo non ha mai avuto l'opportunità di fare esperienza nel campo delle tecnologie trattate ma ritiene di possedere le conoscenze necessarie per riuscire a comprendere le principali problematiche ed intende approfondire ed ampliare le proprie conoscenze.
\end{document}