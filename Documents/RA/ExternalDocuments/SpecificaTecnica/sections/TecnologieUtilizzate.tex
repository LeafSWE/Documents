\documentclass[../SpecificaTecnica.tex]{subfiles}
\begin{document}
\section{Tecnologie utilizzate}
	In questa sezione vengono descritte le tecnologie sulle quali si basa lo sviluppo di BlueWhere seguite dai vantaggi e svantaggi riscontrati nel loro uso.
	
	\subsection{Android}
		\subsubsection{Descrizione}
			Android\g\ è un sistema operativo mobile sviluppato da Google\g\ e basato su kernel\g\ Linux\g. È stato progettato per essere eseguito principalmente su smartphone\g\ e tablet\g\, con interfacce utente specializzate per orologi e televisori. Le versioni di riferimento sono la 4.4 e superiori. L'utilizzo di questa tecnologia è stato richiesto dal proponente.
		\subsubsection{Vantaggi}
			\begin{itemize}
				\item possiede una vasta fetta di mercato mobile;
				\item disponibile su un vasto numero di dispositivi;
				\item quasi totalmente gratuito ed Open Source\g.
			\end{itemize}
		\subsubsection{Svantaggi}
			\begin{itemize}
				\item essendoci un vasto numero di produttori di smartphone e tablet che non aggiornano la versione di Android che rilasciano all'interno dei loro dispositivi, Android risulta essere estremamente frammentato;
				\item necessità di sviluppare applicazioni per dispositivi che possono differire per:
					\begin{itemize}
						\item prestazioni;
						\item risoluzione dello schermo;
						\item durata della batteria;
						\item sensori disponibili.
					\end{itemize}
			\end{itemize}
			
	\subsection{Java}
		Java è uno dei più famosi linguaggi di programmazione orientato agli oggetti supportato da una moltitudine di librerie e documentazione. Viene utilizzato per la scrittura e lo sviluppo dell'applicazione.
		\subsubsection{Vantaggi}
		\begin{itemize}
			\item linguaggio più conosciuto, diffuso e utilizzato nell'ambiente di sviluppo Android
			\item ampia documentazione disponibile;
			\item dispone di un gran numero di librerie;
			\item portabilità su diversi sistemi operativi.
		\end{itemize}
		\subsubsection{Svantaggi}
		\begin{itemize}
			\item linguaggio verboso.
		\end{itemize}
		
	\subsection{SQLite}
		Libreria che implementa un database SQL transazionale senza la necessità di un server. Viene utilizzata per gestire le mappe scaricate e installate nel dispositivo e il relativo contenuto. Il suo utilizzo è stato consigliato dal proponente.
		\subsubsection{Vantaggi}
		\begin{itemize}
			\item database transazionale leggerissimo e molto veloce;
			\item consigliato per i dispositivi mobile;
			\item supporta buona parte dello standard SQL92, già conosciuto dai membri del team;
			\item sistema diffuso e documentato;
			\item formato del database multipiattaforma;
		\end{itemize}
		\subsubsection{Svantaggi}
		\begin{itemize}
			\item non gestisce da sé la concorrenza;
			\item alcune funzionalità SQL sono limitate.
		\end{itemize}
		
	\subsection{AltBeacon}
		Libreria che permette ai sistemi operativi mobile di interfacciarsi ai Beacon, offrendo molteplici funzionalità. Viene utilizzata per permettere la comunicazione tra l'applicativo Android e i Beacon. Il suo utilizzo è stato consigliato dal proponente.
		\subsubsection{Vantaggi}
			\begin{itemize}
				\item pieno supporto Android;
				\item supporta le tipologie di beacon più popolari attualmente in commercio;
				\item supporta dispositivi Android con versione 4.3 o superiore;
				\item si propone come nuovo standard open source.
			\end{itemize}
		\subsubsection{Svantaggi}
			\begin{itemize}
				\item poca documentazione disponibile.
			\end{itemize}
			
	\subsection{JGraphT}
		JGraphT è una libreria Java che fornisce funzionalità matematiche per modellare grafi. Viene utilizzata per la rappresentazione e l'uso delle mappe.
		\subsubsection{Vantaggi}
			\begin{itemize}
				\item progettata per essere semplice e type-safe;
				\item fornisce la possibilità di visualizzare i grafi attraverso JGraph;
				\item ben documentata;
			\end{itemize}
			
	\subsection{PostgreSQL}
		PostgreSQL è un DBMS ad oggetti open source. Viene utilizzato per la gestione del database da cui scaricare le mappe attraverso l'applicazione. Il suo utilizzo è stato consigliato dal proponente. 
		\subsubsection{Vantaggi}
			\begin{itemize}
				\item molto più robusto di MySQL, più stabile e performante;
				\item gestisce la conversione delle informazioni dal mondo SQL a quello della programmazione orientata agli oggetti.
			\end{itemize}
		\subsubsection{Svantaggi}
			\begin{itemize}
				\item più complesso rispetto al classico MySQL.
			\end{itemize}
			
\end{document}