\documentclass[../DefinizioneDiProdotto.tex]{subfiles}
\begin{document}
\section{Introduzione}
	\subsection{Scopo del documento}
		Questo documento definisce nel dettaglio la struttura e le relazioni tra le parti del prodotto\g\ , approfondendo ulteriormente dove ritenuto necessario. In particolare vengono descritti in dettaglio i package, le classi e le interfacce, concludendo con il tracciamento tra le classi e i requisiti analizzati nell'\analisideirequisitiv\ .
	
	\subsection{Scopo del prodotto}
		Lo scopo del prodotto\g\ è implementare un metodo di navigazione indoor\g\ che sia funzionale alla tecnologia Bluetooth Low Energy (BLE\g\ ). Il prodotto\g\ comprenderà un prototipo software\g\ che permetta la navigazione all'interno di un'area predefinita, basandosi sui concetti di Indoor Positioning System (IPS\g\ ) e smart place\g\ .
	
	\subsection{Glossario} \label{sec:Glossario}
	Allo scopo di rendere più semplice e chiara la comprensione dei documenti viene allegato il \glossariov\ nel quale verranno raccolte le spiegazioni di  terminologia tecnica o  ambigua, abbreviazioni ed acronimi. Per evidenziare un termine presente in tale documento, esso verrà marcato con il pedice \g .
	
	\subsection{Riferimenti utili}
	
		\subsubsection{Riferimenti normativi}
		\begin{itemize}
			\item capitolato d'appalto C2: CLIPS\g\ : Comunication \& Localization with Indoor Positioning Systems:
			\frmURI{http://www.math.unipd.it/~tullio/IS-1/2015/Progetto/C2.pdf};
			\item \normediprogettov.
		\end{itemize}
		
		\subsubsection{Riferimenti informativi}
		\begin{itemize}
			\item Documentazione Android SDK: \frmURI{http://developer.android.com/guide/index.html};
			\item Documentazione AltBeacon Library: \frmURI{https://altbeacon.github.io/android-beacon-library/documentation.html};
			\item Documentazione SQLite: \frmURI{https://www.sqlite.org/docs.html};
			\item Documentazione JavaDoc JGraphT Library: \frmURI{http://jgrapht.org/javadoc/};
			\item Materiale di riferimento del corso di Ingegneria del Software\g\ - Diagrammi delle classi: \frmURI{http://www.math.unipd.it/~tullio/IS-1/2015/Dispense/E03.pdf};
			\item Materiale di riferimento del corso di Ingegneria del Software\g\ - Model View Presenter: \frmURI{http://www.math.unipd.it/~rcardin/sweb/Design\%20Pattern\%20Architetturali\%20-\%20Model\%20View\%20Controller_4x4.pdf};
			\item Materiale di riferimento del corso di Ingegneria del Software\g\ - Layer Architecture: \frmURI{http://www.math.unipd.it/~rcardin/sweb/Software\%20Architecture\%20Patterns_4x4.pdf}
			\item Design Pattern: elementi per il riuso di software ad oggetti - Gamma, Helm, Johnson, Vlissides - editore Pearson - 2002: Part 1, capitoli: 5 - System modeling, 6 - Architectural design \& 7 - Design and implementation;
			\item UML e ingegneria del software: dalla teoria alla pratica - Luca Vetti Tagliati - 2015: capitoli: 7 - Gli oggetti: una questione di classe \& 9 - Diagrammi di interazione.
		\end{itemize}
		
\end{document}