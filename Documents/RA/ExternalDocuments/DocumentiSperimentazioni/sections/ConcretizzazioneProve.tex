\documentclass[../SperimentazioniPratiche.tex]{subfiles}

\begin{document}

\definecolor{cadmiumgreen}{rgb}{0.0, 0.42, 0.24}
\newcommand{\ok}{\textcolor{cadmiumgreen}{OK}} % Risultati quivalenti agli attesi
\newcommand{\ns}{\textcolor{red}{N.S.}} % non supportato
\newcommand{\nd}{\textcolor{red}{N.D.}} % non disponibile

\section{Concretizzazione prove}
\label{sec:ConcretizzazioneProve}
			Ogni prova concettuale è effettuata attraverso una sua concretizzazione. Una concretizzazione rappresenta una prova reale effettuata sul campo, essa è identificata da un codice univoco:
			\begin{quote}
				\centering
				Prova \verb|[N][I].[T]|
			\end{quote}
			Dove:
			\begin{itemize}
				\item N è un carattere numerico che identifica la prova concettuale su cui la prova reale si basa;
				\item I è un carattere dell'alfabeto latino che identifica una impostazione dei valori delle variabili in input e output atteso della prova;
				\item T è un carattere numerico che identifica il numero di tentativo della prova.
			\end{itemize}
			
			Ogni prova reale è rappresentata all'interno di una \textbf{scheda} con le seguenti informazioni:
			\begin{itemize}
				\item Title: è il codice univoco che identifica la prova effettuata;
				\item Input: valori associati alle variabili in ingresso della prova. Se non richiesti è segnalato con 'N.R.' (Non Richiesti);
				\item Output attesi: valori associati alle variabili in uscita dalla prova secondo gli input definiti in precedenza. Le variabili di output sono sempre identificate dalla variabile \verb|$RESULT| con un suffisso numerico se gli output siano più di uno;
				\item Output riscontrati: valori associati alle variabili in uscita dalla prova effettuata per ogni dispositivo elencato. 
				\begin{itemize}
					\item Se i risultati sono equivalenti ai risultati attesi, le variabili output sono marcate con \ok;
					\item Se non è possibile reperire i risultati perché l'applicazione non li supporta, le variabili output sono marcate con \ns (Non Supportata);
					\item Se i risultati non corrispondono ai risultati attesi, vengono mostrate le informazioni errate in uscita dal dispositivo.
				\end{itemize}
			\end{itemize}

\end{document}