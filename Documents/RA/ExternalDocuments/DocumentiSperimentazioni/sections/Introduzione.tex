\documentclass[../Sperimentazioni.tex]{subfiles}

\begin{document}

\section{Introduzione}

	\subsection{Scopo del documento}
		Questo documento raccoglie le istruzioni, le tecniche e le procedure seguite per effettuare le prove sperimentali pratiche della navigazione indoor implementata nell'applicazione sviluppata dal gruppo \Leaf.
	
	\subsection{Scopo del prodotto}
		Lo scopo del prodotto\g\ è implementare un metodo di navigazione indoor\g\ che sia funzionale alla tecnologia Bluetooth Low Energy (BLE\g\ ). Il prodotto\g\ comprenderà un prototipo software\g\ che permetta la navigazione all'interno di un'area predefinita, basandosi sui concetti di Indoor Positioning System (IPS\g\ ) e smart place\g\ .
	
	\subsection{Glossario} \label{sec:Glossario}
	Allo scopo di rendere più semplice e chiara la comprensione dei documenti viene allegato il \glossariov\ nel quale verranno raccolte le spiegazioni di  terminologia tecnica o  ambigua, abbreviazioni ed acronimi. Per evidenziare un termine presente in tale documento, esso verrà marcato con il pedice \g .
	
	
	\subsection{Riferimenti utili}
	
		\subsubsection{Riferimenti normativi}
		\begin{itemize}
			\item capitolato d'appalto C2: CLIPS\g\ : Comunication \& Localization with Indoor Positioning Systems:
			\frmURI{http://www.math.unipd.it/~tullio/IS-1/2015/Progetto/C2.pdf};
			\item \normediprogettov.
		\end{itemize}
		
		\subsubsection{Riferimenti informativi}
		\begin{itemize}
			\item Manuale utente \textit{Clips}: ???;
			\item Manuale sviluppatore \textit{Clips}: ???.
		\end{itemize}
		
\end{document}