\documentclass[../Sperimentazione.tex]{subfiles}

\begin{document}
	
\section{Prove concettuali}
	Nella presente sezione si raccolgono i modelli concettuali di prove che verranno effettuate successivamente sul campo. Ogni modello di \textbf{prova} è strutturata nelle seguenti sottosezioni:
	\begin{itemize}
		\item \textbf{Prova} \verb|num_id|, dove \verb|num_id| rappresenta un numero intero positivo che identifica univocamente la prova;
		\item \textbf{Obiettivo}, in cui si descrive lo scopo della prova;
		\item \textbf{Test di sistema}, in cui si elencano i test di sistema associati a tale prova, se tale prova è soddisfatta anche i test di sistema associati sono considerati soddisfatti;
		\item \textbf{Procedura}, in cui si elencano i passaggi da eseguire in ordine per la corretta esecuzione della prova;
		\item \textbf{Output attesi}, in cui si elencano i risultati dell'esecuzione della \textbf{procedura}.
	\end{itemize}
	
\subsection{Variabili}
	Per rappresentare le prove come modelli concettuali si è fatto uso di variabili identificabili dal simbolo \verb|$| come prefisso, seguite da un nome identificativo in caratteri \textbf{maiuscolo} e, se necessario, da un suffisso numerico se esistono più variabili simili con stesso nome identificativo.

	L'uso delle variabili consente di effettuare la stessa prova con diverse configurazioni con diversi input, ossia diversi valori assegnati alle variabili di ingresso.
	
	Le variabili in \textbf{output} sono sempre identificate dalla variabile \verb|$RESULT| che può essere seguita da un suffisso numerico se i risultati sono più di uno.
	


\subsection{Precondizioni generali}
	La seguente lista contiene tutte le precondizioni da rispettare in ogni prova elencata successivamente. Queste precondizioni sono valide se e solo se all'interno della prova stessa non ne sono specificate altre che vanno in contraddizione.
	\begin{itemize}
		\item Il dispositivo utilizzato ha il Bluetooth BLE 4.0 attivo;
		\item Il dispositivo utilizzato ha il servizio di geolocalizzazione attivo;
		\item Il dispositivo utilizzato ha una connessione Internet (connessione Wi-fi o connessione dati);
		\item Il dispositivo utilizzato ha almeno 10 MB di spazio libero nella memoria principale;
		\item Il dispositivo utilizzato opera all'interno dell'edificio Torre Archimede (soddisfatti TS22, TS22.1 e TS22.2);
		\item Il dispositivo utilizzato è nelle vicinanze di un beacon che identifica una Region Of Interest (ROI) dell'edificio;
		\item L'area sviluppatore dell'applicazione \textit{Clips} è già sbloccata tramite l'inserimento dell'apposita password.
	\end{itemize}



\newpage
\subsection{Prova 1} % lista indicazioni e ricerca categorie
\label{subsec:Prova1}	
	
	\subsubsection{Obiettivo}
		Visualizzare dall'applicazione le indicazioni attese per raggiungere la destinazione \verb|$END|.
		
	\subsubsection{Test di sistema}
		TS1, TS1.8,
		TS1.14, TS1.14.1,
		TS14.1,
		TS1.1.3,
		TS22.1;
		
	\subsubsection{Procedura}
		\begin{enumerate}
		\item Avviare l'applicazione \textit{Clips};
		\item Seleziona la categoria  \verb|$CAT|;
		\item Seleziona la destinazione  \verb|$END|.
		\end{enumerate}
	
	\subsubsection{Output attesi}
		\begin{itemize}
		\item Al punto 2 l'applicazione deve mostrare la lista dei POI seguente:  \verb|$RESULT1|
		\item L'applicazione una volta selezionata la destinazione  \verb|$END| deve mostrare il percorso composto dai passi  \verb|$RESULT2|.
		\end{itemize}
		
	
\newpage		
\subsection{Prova 2} % preferenze
\label{subsec:Prova2}
	
	\subsubsection{Obiettivo}
		Visualizzare dall'applicazione le indicazioni attese per raggiungere la destinazione  \verb|$END| con impostate le preferenze  \verb|$PREF|
		
	\subsubsection{Test di sistema}
		TS1.1, TS1.1.1, TS1.1.2,
		TS14, TS14.6, TS14.7, TS14.8;
		
		
	\subsubsection{Procedura}
		\begin{enumerate}
		\item Posizionarsi nell'area \verb|$START|;
		\item Avviare l'applicazione \textit{Clips};
		\item Dal menu dell'applicazione accedere a preferenze;
		\item Impostare le preferenze  \verb|$PREF1|;
		\item Dalla schermata principale si seleziona la categoria  \verb|$CAT|;
		\item Si seleziona la destinazione  \verb|$END|.
		\end{enumerate}
		
	\subsubsection{Output attesi}
		\begin{itemize}
		\item L'applicazione una volta selezionata la destinazione  \verb|$END| deve mostrare il percorso composta dai passi  \verb|$RESULT| che prevedono l'uso delle \verb|$PREF| precedentemente impostate.
		\end{itemize}
		
	
	
\newpage
\subsection{Prova 3} % Ricalcolo percorso
\label{subsec:Prova3}
	
	\subsubsection{Obiettivo}
		Visualizzare l'avviso ricalcolo percorso quando intenzionalmente si procede per una direzione diversa da quella prevista dall'applicazione.
		
	\subsubsection{Test di sistema}
		TS1.4;
		
	\subsubsection{Procedura}
		\begin{enumerate}
		\item Posizionarsi in  \verb|$START|;
		\item Avviare l'applicazione \textit{Clips};
		\item Inserire nella search box la destinazione  \verb|$END| e premere \textit{Invio};
		\item Seguire le indicazioni fino al punto  \verb|$CHANGE|;
		\item Dal punto \verb|$CHANGE| prendere.
		\end{enumerate}
		
	\subsubsection{Output attesi}
		\begin{itemize}
		\item L'applicazione una volta presa la direzione opposta deve mostrare un avviso di ricalcolo percorso e mostrare le nuove indicazioni previste:  \verb|$RESULT|.
		\end{itemize}
		
	
	
\newpage		
\subsection{Prova 4} % indicazioni estese e immagini
\label{subsec:Prova4}
	
	\subsubsection{Obiettivo}
		Visualizzare le indicazioni testuali estese e le immagini di un'area da raggiungere per poi continuare verso la destinazione scelta.
		
	\subsubsection{Test di sistema}
		TS1.6, TS1.7;
		
	\subsubsection{Procedura}
		\begin{enumerate}
		\item Posizionarsi nell'area  \verb|$START|;
		\item Avviare l'applicazione \textit{Clips};
		\item Inserire la destinazione  \verb|$END| nella search box e premere \textit{Invio};
		\item Selezionare indicazione  \verb|$INST|.
		\end{enumerate}

	\subsubsection{Output attesi}
		\begin{itemize}
		\item L'applicazione una volta selezionata un'indicazione del percorso mostrato deve mostrare informazioni testuali  \verb|$INFO|;
		\item L'applicazione una volta selezionata un'indicazione del percorso mostrato deve mostrare le due immagini previste di tale area:  \verb|$IMG1| e  \verb|$IMG2|.
		\end{itemize}
		
	
	
\newpage		
\subsection{Prova 5} % annullare la navigazione
\label{subsec:Prova5}
	
	\subsubsection{Obiettivo}
		Annullare la navigazione avviata precedentemente.
		
	\subsubsection{Test di sistema}
		TS1.9;
		
	\subsubsection{Procedura}
		\begin{enumerate}
		\item Posizionarsi nell'area  \verb|$START|;
		\item Avviare l'applicazione \textit{Clips};
		\item Inserire la destinazione  \verb|$END| nella search box e premere 'Invio';
		\item Selezionare pulsante \textit{Back}.
		\end{enumerate}
		
	\subsubsection{Output attesi}
		\begin{itemize}
		\item L'applicazione in seguito alla pressione del pulsante \textit{Back} deve annullare la navigazione in corso e ritornare alla schermata principale.
		\end{itemize}	
		
	
	
\newpage
\subsection{Prova 6} % avviso mappa non aggiornata
\label{subsec:Prova6}
	
	\subsubsection{Obiettivo}
		Visualizzare avviso: \textit{"mappa non aggiornata"}.
		
	\subsubsection{Test di sistema}
		TS1.11;
		
	\subsubsection{Precondizioni}
		\begin{itemize}
			\item Assicurarsi di avere una mappa installata con versione inferiore rispetto quella disponibile nel database remoto
		\end{itemize}
	
	\subsubsection{Procedura}	
	
		\begin{enumerate}
		\item Avviare l'applicazione
		\end{enumerate}
		
	\subsubsection{Output attesi}
		\begin{itemize}
		\item Dopo l'avvio l'applicazione deve mostrare un messaggio di avviso se la mappa dell'edificio in cui il dispositivo si trova salvata in locale non ha la versione uguale a quella disponibile nel database remoto.
		\end{itemize}
	
	
	
\newpage	
\subsection{Prova 7} % avviso mappa non scaricata
\label{subsec:Prova7}	
	
	\subsubsection{Obiettivo}
		Visualizzare avviso: \textit{"mappa non scaricata"}.
		
	\subsubsection{Test di sistema}
		TS1.12;
		
	\subsubsection{Precondizioni}
		\begin{itemize}
			\item Qualsiasi dato salvato precedentemente dall'applicazione è eliminato.
		\end{itemize}
		
	\subsubsection{Procedura}
		\begin{enumerate}
		\item Avviare l'applicazione \textit{Clips}.
		\end{enumerate}
		
	\subsubsection{Output attesi}
		\begin{itemize}
		\item Dopo l'avvio l'applicazione deve mostrare un messaggio di avviso se la mappa dell'edificio in cui il dispositivo si trova non è salvata in locale.
		\end{itemize}
	
	
	

\newpage	
\subsection{Prova 8} % attivazione sensori
\label{subsec:Prova8}
	
	\subsubsection{Obiettivo}
		Visualizzare avvisi per l'attivazione dei sensori richiesti per il funzionamento dell'applicazione
		
	\subsubsection{Test di sistema}
		TS2, TS2.1, TS2.2, TS2.3;
		
	\subsubsection{Precondizioni}
		\begin{itemize}
			\item Il sensore bluetooth del device è spento;
			\item Il servizio di geolocalizzazione del device è spento;
			\item Il GPS del device è spento.
		\end{itemize}
		
	\subsubsection{Procedura}
		\begin{enumerate}
		\item Si avvia l'applicazione \textit{Clips}.
		\end{enumerate}
		
	\subsubsection{Output attesi}
		\begin{itemize}
		\item Dopo l'avvio l'applicazione richiede con un messaggio d'avviso di attivare il sensore bluetooth e in seguito il servizio di geolocalizzazione;
		\item Se il sistema operativo in uso nel device è la versione Lollipop 5.0 o superiore l'applicazione richiede con un ulteriore messaggio d'avviso di attivare il GPS del dispositivo.
		\end{itemize}
		

\newpage	
\subsection{Prova 9} % gestire un log
\label{subsec:Prova9}	
	
	\subsubsection{Obiettivo}
		Reperire e visualizzare UUID di beacon, major, minor, livello di potenza, livello di batteria, distanza approssimativa dal dispositivo, formato del beacon e area coperta dal beacon.
		
	\subsubsection{Test di sistema}
		TS18, TS18.1, TS18.2, TS18.3, TS18.4, TS18.5, TS18.6, TS18.7,
		TS18.9, TS18.9.1, TS18.9.2, TS18.9.3, TS18.9.5,
		TS3.2, TS3.4, TS3.5;
		
	\subsubsection{Precondizioni}
		\begin{itemize}
			\item Durante la rilevazione non esistono altri beacon all'infuori di beacon che compongono la mappatura dell'edificio.
		\end{itemize}				
		
	\subsubsection{Procedura}
		\begin{enumerate}
		\item Posizionarsi nell'area \verb|$POS|;
		\item Avviare l'applicazione;
		\item Dal menu accedere sezione area sviluppatore;
		\item Dalla schermata \textit{I tuoi log} selezionare il pulsante \textit{Nuovo log};
		\item Dopo 5 secondi selezionare il pulsante \textit{Stop}  che salverà il log in corso;
		\item Selezionare il log posizionato più in basso della lista.
		\end{enumerate}
		
	\subsubsection{Output attesi}
		\begin{itemize}
		\item \verb|$BEACONREAD|:
			\begin{itemize}
				\item \verb|$UUID|
			 	\item \verb|$Major|
			 	\item \verb|$Minor|
			 	\item \verb|$RSSI|
			 	\item \verb|$TXPOWER|
			 	\item \verb|$BATTERY|
			 	\item \verb|$DISTANCE|
			 	\item \verb|$BEACONTYPE|
			 	\item \verb|$BLUETOOTHADDRESS|
			\end{itemize}
			 
		\end{itemize}
		
\newpage
\subsection{Prova 10} % visualizzare i beacon nella planimetria
\label{subsec:Prova10}

	\subsubsection{Obiettivo}
		Visualizzare nella planimetria la circonferenza del beacon che identifica l'area \verb|$AREA|.
	
	\subsubsection{Test di sistema}
		TS3.6, TS18.8;
	
	\subsubsection{Procedura}
		\begin{enumerate}
		\item Avviare l'applicazione \textit{Clips};
		\item Dal menu dell'applicazione selezionare \textit{Area sviluppatore};
		\item Aprire il menu in alto a destra;
		\item Selezionare \textit{Beacon Power Adapter};
		\item Selezionare il pulsante \textit{play}.
		\end{enumerate}
		
	\subsubsection{Output attesi}
		\begin{itemize}
			\item L'applicazione deve mostrare una planimetria dell'edificio Torre Archimede in cui si evidenzia una circonferenza nel punto \verb|$RESULT|.
		\end{itemize}



\newpage	
\subsection{Prova 11} % search box
\label{subsec:Prova11}
	
	\subsubsection{Obiettivo}
		Ottenere l'avviso \textit{"Nessun risultato"} dopo avere inserito nella search box  \verb|$WRONGSTRING| e successivamente ottenere le informazioni per raggiungere la destinazione  \verb|$END| selezionata nel menu a tendina dopo aver inserito  \verb|$STRING| nella search box.
		
	\subsubsection{Test di sistema}
		TS1.13, TS1.13.1,
		TS19,
		TS1.15, TS1.16;
		
	\subsubsection{Precondizioni}
		\begin{itemize}
			\item Il dispositivo non deve essere mosso durante la prova.
		\end{itemize}
		
	\subsubsection{Procedura}
		\begin{enumerate}
		\item Posizionarsi in \verb|$START|;
		\item Avviare l'applicazione;
		\item Dalla schermata principale selezionare la search box in alto;
		\item Inserire la stringa  \verb|$WRONGSTRING|;
		\item Premere \textit{Invio};
		\item Selezionare pulsante \textit{Back};
		\item Selezionare nuovamente la search box in alto;
		\item Inserire la stringa  \verb|$STRING|;
		\item Premere \textit{Invio}; % Non dovrebbe accadere nulla
		\item Dal menu a tendina selezionare la destinazione  \verb|$END|;
		\end{enumerate}
		
	\subsubsection{Output attesi}
		\begin{itemize}
		\item Al punto 4 l'applicazione deve mostrare nel menu a tendina le alternative:  \verb|$RESULT1|;
		\item Al punto 5 l'applicazione deve mostrare una schermata con l'avviso \verb|$RESULT2|;
		\item Al punto 8 l'applicazione deve mostrare nel menu a tendina i seguenti risultati:  \verb|$RESULT3|;
		\item Al punto 10 l'applicazione deve mostrare la lista di indicazioni composta da tali passi:  \verb|$RESULT4|.
		\end{itemize}

	


\newpage	
\subsection{Prova 12} % tutti i POI edificio
\label{subsec:Prova12}
	
	\subsubsection{Obiettivo}
		Visualizzare una lista di tutti i POI all'interno dell'edificio e visualizzare i dettagli del POI \verb|$SELECT| contenuto in essa.
		
	\subsubsection{Test di sistema}
		TS9, TS9.1,
		TS10, TS10.1, TS10.2,
		TS3.1.1;

	\subsubsection{Procedura}
		\begin{enumerate}
		\item Avviare l'applicazione;
		\item Dalla schermata principale selezionare il pulsante identificato con un'icona di un edificio per accedere a tutta la lista dei POI;
		\item Selezionare il POI \verb|$SELECT|.
		\end{enumerate}
		
	\subsubsection{Output attesi}
		\begin{itemize}
		\item Al punto 2 la schermata fornisce la seguente lista di POI:  \verb|$RESULT1|;
		\item Al punto 3 è possibile accedere al nome e informazioni:  \verb|$RESULT2| del POI  \verb|$SELECT| selezionato.
		\end{itemize}
	

	
\newpage	
\subsection{Prova 13} % esplora POI circostanti
\label{subsec:Prova13}
	
	\subsubsection{Obiettivo}
		Visualizzare la lista di POI appartenenti alla ROI in cui si trova l'utente.
		
	\subsubsection{Test di sistema}
		TS11,
		TS3.1.2;
		
	\subsubsection{Procedura}
		\begin{enumerate}
		\item Posizionarsi nell'area \verb|$POS|;
		\item Avviare l'applicazione;
		\item Dalla schermata principale selezionare il pulsante \textit{Esplora} identificato dall'icona posizione. 
		\end{enumerate}
		
	\subsubsection{Output attesi}
		\begin{itemize}
		\item La lista dei POI mostrata deve contenere tali POI: \verb|$RESULT|.
		\end{itemize}
		
	
		
\newpage	
\subsection{Prova 14} % immagini non disponibili per assenza internet
\label{subsec:Prova14}
	
	\subsubsection{Obiettivo}
		Visualizzare un messaggio di avviso che la connessione Internet non è attiva ed è impossibile scaricare le immagini della istruzione  \verb|$INST| selezionata.
		
	\subsubsection{Test di sistema}
		TS1.17;
		
	\subsubsection{Precondizioni}
		\begin{itemize}
			\item Qualsiasi tipo di connessione a internet del dispositivo viene disattivata;
		\end{itemize}				
		
	\subsubsection{Procedura}
		\begin{enumerate}
		\item Posizionarsi nell'area \verb|$START|;
		\item Avviare l'applicazione \textit{Clips};
		\item Inserire nella search box la destinazine  \verb|$END| e premere \textit{Invio};
		\item Selezionare indicazione  \verb|$INST|;
		\item Selezionare un'immagine disponibile tra quelle disponibili.
		\end{enumerate}

	\subsubsection{Output attesi}
		\begin{itemize}
		\item L'applicazione una volta selezionata un'immagine della ROI da attraversare mostra un avviso \textit{"Connessione a Internet assente, impossibile scaricare l'immagine"}.
		\end{itemize}

%-----------------
% Navigazione 2.0
%-----------------				
		
\newpage	
\subsection{Prova 15} % navigazione guidata dinamicamente a passo veloce
\label{subsec:Prova15}	
	
	\subsubsection{Obiettivo}
		Verificare che l'applicazione guidi istruzione per istruzione l'utente evidenziando le prossime istruzioni da seguire.
		
	\subsubsection{Test di sistema}
		Nessuno.
		
	\subsubsection{Procedura}
		\begin{enumerate}
		\item Posizionare il dispositivo nella ROI  \verb|$START|;
		\item Avviare l'applicazione \textit{Clips};
		\item Inserire nella search box la destinazione  \verb|$END| e premere \textit{Invio};
		\item A \verb|$SPEED|, seguire tutte le istruzioni fino a raggiungere la destinazione scelta.
		\end{enumerate}
		
	\subsubsection{Output attesi}
		\begin{itemize}
		\item All'attraversamento della ROI  \verb|$RESULT| l'istruzione relativa al suo raggiungimento viene evidenziata.
		\end{itemize}

	
		
\newpage	
\subsection{Prova 16} % indicazioni della bussola
\label{subsec:Prova16}	
	
	\subsubsection{Obiettivo}
		Visualizzare la corretta direzione da seguire associata ad ogni istruzione.
		
	\subsubsection{Test di sistema}
		Nessuno.
		
	\subsubsection{Procedura}
		\begin{enumerate}
		\item Posizionare il dispositivo nella ROI \verb|$START| rivolgendolo verso la prossima ROI  \verb|$NEXT| da raggiungere;
		\item Avviare l'applicazione \textit{Clips};
		\item Inserire nella search box la destinazione  \verb|$END| e premere \textit{Invio};
		\item Girare il dispositivo di  \verb|$GRADE| gradi.
		\end{enumerate}
		
	\subsubsection{Output attesi}
		\begin{itemize}
			\item La prima istruzione contiene l'indicazione di direzione  \verb|$RESULT|.
		\end{itemize}

	
	% Tabella prove totali
	
		
	
\end{document}