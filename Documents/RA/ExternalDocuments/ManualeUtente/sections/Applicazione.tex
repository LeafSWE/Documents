\documentclass[../ClipsManualeUtente.tex]{subfiles}

\begin{document}

\section{Altre funzionalità dell'applicazione}
	Nella sezione seguente vengono raccolte tutte le istruzioni necessarie per poter usufruire pienamente di tutte le altre funzionalità a supporto della navigazione offerte dall'applicazione CLIPS.
	
	%\newpage
	\subsection{Esplora luoghi nelle vicinanze}
		L'applicazione è in grado di individuare la posizione approssimata dello smartphone all'interno dell'edificio.
		È possibile visualizzare una lista delle aree di interesse attorno alla posizione in cui si è, per visualizzare ciò:
		\begin{enumerate}
		\item attiva il \textbf{bluetooth};
		\item attiva una \textbf{connessione dati} oppure una \textbf{connessione Wi-fi} e assicurati che lo smartphone sia connesso ad Internet;
		\item \textbf{avvia l'applicazione} selezionando l'icona \includegraphics[scale=0.4]{img/LogoApp};
		\item assicurati di essere in un \textbf{edifico supportato} dall'applicazione. Per assicurarsi di ciò basta accertarsi che dalla schermata principale dell'applicazione siano visibili le informazioni dell'edificio;
		
		\begin{framed}
			\textbf{Nota:} solo per i dispositivi con \textbf{Android `Marshmallow' 6.0} - il \textbf{GPS} dello smartphone deve essere \textbf{attivato} prima di avviare l'applicazione.
		\end{framed}
		
		\item selezione il pulsante segnalato in figura \ref{fig:PulsanteEsplora}.
		\end{enumerate}
		
		\begin{figure} [h]
			\centering
			\subfloat[][Pulsante per accedere alla lista di aree d'interesse vicine]
			{\includegraphics[width=0.40\textwidth]{img2/Esplora-PulsanteEsplora}
			\label{fig:PulsanteEsplora}} \quad
			\hspace{1.5cm}
			\subfloat[][Lista di possibili aree d'interesse ottenuta dopo la procedura]
			{\includegraphics[width=0.33\textwidth]{img/ListaNearbyPoi}
			\label{fig:ListaNearbyPoi}} \\
			\caption{Utilizzare la funzionalità esplora area}
		\end{figure}
	
	% -------------------------------------------------	
	
	\newpage
	\subsection{Esplora luoghi edificio}
		L'applicazione offre anche la possibilità di visualizzare la lista di tutti i luoghi accessibili presenti nell'edificio in cui ci si trova.
		Per accedere a tale funzionalità:
		\begin{enumerate}
			\item attiva il \textbf{bluetooth};
			\item attiva una \textbf{connessione dati} oppure una \textbf{connessione Wi-fi} e assicurati che lo smartphone sia connesso ad Internet;
			\item \textbf{avvia l'applicazione} selezionando l'icona \includegraphics[scale=0.4]{img/LogoApp};
			\item assicurati di essere in un \textbf{edifico supportato} dall'applicazione. Per assicurarsi di ciò basta accertarsi che dalla schermata principale dell'applicazione siano visibili le informazioni dell'edificio;
			\item seleziona il pulsante segnalato in figura \ref{fig:PulsanteTuttiLuoghi} per visualizzare una lista di tutti i luoghi accessibili dell'edificio;
			
		\begin{figure} [h]
			\centering
			\includegraphics[width=0.50\textwidth]{img2/Tutti-PulsanteTuttiLuoghi}
			\caption{Pulsante per accedere alla lista di aree d'interesse dell'edificio}
			\label{fig:PulsanteTuttiLuoghi}
		\end{figure}					
			
			\item seleziona un luogo dalla lista in figura \ref{fig:ListaLuoghi} per accedere alle sue informazioni dettagliate (figura \ref{fig:DettaglioLuogo}).
		\end{enumerate}
		
		\begin{figure} [p]
			\centering
			\subfloat[][Lista di possibili aree d'interesse ottenuta dopo la procedura]
			{\includegraphics[width=0.33\textwidth]{img2/Tutti-ListaLuoghi}
			\label{fig:ListaLuoghi}} \quad
			\hspace{1.5cm}
			\subfloat[][Dettaglio luogo precedentemente selezionato]
			{\includegraphics[width=0.33\textwidth]{img2/Tutti-DettaglioLuogo}
			\label{fig:DettaglioLuogo}} \\
			\caption{Utilizzare la funzionalità esplora luoghi edificio}
		\end{figure}
	
	% --------------------------------------------------
		
	\newpage
	\subsection{Preferenze}
		È possibile configurare la navigazione attraverso l'impostazione di determinate preferenze. Tali preferenze impatteranno nel calcolo del percorso nella navigazione.
		Le preferenze disponibili sono:
			\begin{itemize}
				\item \textit{Percorso veloce} (impostazione predefinita);
				\item \textit{Percorso accessibile}: il percorso calcolato sarà privo di barriere architettoniche come ad esempio le scale;
				\item \textit{Percorso senza ascensori}: il percorso calcolato eviterà l'uso degli ascensori.
			\end{itemize}
		
		Per impostare le proprie preferenze:
		\begin{enumerate}
			\item apri il menu dell'applicazione selezionando l'icona $\equiv$ (figura: \ref{fig:MenuApp});
			\item seleziona \textit{Preferenze};
			\item seleziona \textit{Preferenze di percorso} (figura \ref{fig:ListaPreferenze});
			
	\begin{framed}
	\textbf{Nota:} attualmente le uniche preferenze disponibili sono le \textit{Preferenze di percorso}. Nuove versioni dell'applicazione potrebbero aggiungerne altre.
	\end{framed}			
			
			\item scegli la preferenza oppure seleziona \textit{Annulla} per lasciare la preferenza già impostata (figura \ref{fig:SceltaPercorso}).
		\end{enumerate}

	\begin{framed} 
	\textbf{Nota:} una volta impostate le preferenze, esse saranno attive solo dalla prossima navigazione avviata.
	\end{framed}
	
	\begin{figure} [p]
			\centering
			\subfloat[][Menu dell'applicazione]
			{\includegraphics[width=0.33\textwidth]{img2/MenuApp}
			\label{fig:MenuApp}} \quad
			\hspace{1.5cm}
			\subfloat[][Lista preferenze dell'applicazione]
			{\includegraphics[width=0.33\textwidth]{img2/Preferenze-ListaPreferenze}
			\label{fig:ListaPreferenze}} \\
			
			\subfloat[][Scelta delle preferenze di percorso]
			{\includegraphics[width=0.33\textwidth]{img2/Preferenze-SceltaPercorso}
			\label{fig:SceltaPercorso}} \\
			\caption{Accedere all'area Preferenze e impostare le proprie preferenze di percorso}
	\end{figure}

	% -----------------------------------------

	\newpage
	\subsection{Gestire le mappe}
		L'applicazione offre la possibilità di scaricare, aggiornare o rimuovere o semplicemente reperire informazioni le mappe supportate dall'applicazione.
		Per accedere a tale funzionalità:
		\begin{enumerate}
			\item apri il menu dell'applicazione selezionando l'icona $\equiv$;
			\item seleziona \textit{Le tue mappe} come mostra la figura \ref{fig:GestioneMappe-Menu};
			\item la schermata ora aperta mostra le mappe scaricate e installate nello smartphone (figura \ref{fig:GestioneMappe-ListaMappeScaricate}). Ogni mappa mostra le seguenti informazioni (partendo dall'alto da destra a sinistra):
			\begin{itemize}
				\item l'indirizzo dell'edificio;
				\item la versione della mappa installata (maggiore è la versione più la mappa ha subito modifiche e cambiamenti per migliorarla);
				\item il nome dell'edificio;
				\item una breve descrizione dell'edificio;
				\item la dimensione in KB della mappa installata;
				\item l'avviso se la mappa è già aggiornata o richiede un aggiornamento, nell'ultimo caso sarà segnalato con il testo in rosso (figura \ref{fig:GestioneMappe-MappaDaAggiornare});
				\item due pulsanti per eliminarla e richiedere l'aggiornamento.
			\end{itemize}

		\begin{framed}
			\textbf{Nota:} nella schermata \ref{fig:GestioneMappe-ListaMappeScaricate} sono elencate anche le mappe scaricate automaticamente per permettere la navigazione all'interno di edifici rilevati e identificati dall'applicazione stessa. Tuttavia prima di effettuare il download di tali mappe l'applicazione ti avvisa con un messaggio per procedere.
		\end{framed}
		
			\item se desideri eliminare definitivamente la mappa dal tuo dispositivo premi il pulsante \includegraphics[scale=0.5]{img2/GestioneMappe-EliminaMappaPulsante} sotto alla descrizione dell'edificio interessato;
			\item se desideri aggiornare una mappa installata nel tuo dispositivo premi il pulsante \includegraphics[scale=0.5]{img2/GestioneMappe-AggiornaMappaPulsante} sotto alla descrizione dell'edificio interessato, se la mappa è già aggiornata ti verrà segnalato con un avviso;
			\item premi il pulsante \includegraphics[scale=0.3]{img2/GestioneMappe-DownloadMappePulsante} per accedere alla lista di mappe degli edifici supportati dall'applicazione (figura \ref{fig:GestioneMappe-DownloadMappe}).
		\end{enumerate}
		

			\begin{figure} [h]
				\centering
				\subfloat[][Menu dell'applicazione con la selezione di \textit{Le tue mappe} per accedere a tale area]
				{\includegraphics[width=0.33\textwidth]{img2/GestioneMappe-Menu}
				\label{fig:GestioneMappe-Menu}} \quad
				\hspace{1.5cm}
				\subfloat[][Lista delle mappe scaricate e installate nel dispositivo accompagnate da descrizione]
				{\includegraphics[width=0.33\textwidth]{img2/GestioneMappe-ListaMappeScaricate}
				\label{fig:GestioneMappe-ListaMappeScaricate}} \\
				\caption{Gestire le mappe}
			\end{figure}		
		
		\subsubsection{Scaricare una mappa}
		Se desideri scaricare una mappa da Internet:
		\begin{enumerate}
			\item assicurati di avere una connessione a Internet attiva;
			\item accedi alla sezione \textit{Le tue mappe} seguendo l'elenco precedente a questo;
			\item premi il pulsante \includegraphics[scale=0.3]{img2/GestioneMappe-DownloadMappePulsante} per accedere alla lista di mappe degli edifici supportati dall'applicazione (figura \ref{fig:GestioneMappe-DownloadMappe});
			\item nella schermata con la lista delle mappe seleziona il pulsante \includegraphics[scale=0.5]{img2/GestioneMappe-ScaricaMappaPulsante}. Se la mappa è già stata scaricata ti verrà mostrato l'avviso come in figura \ref{fig:GestioneMappe-MappaGiaScaricata}.
		\end{enumerate}		
		
		
		\begin{figure} [p]
				\centering
				\subfloat[][Lista delle mappe installate con la segnalazione tramite testo rosso di aggiornare una mappa]
				{\includegraphics[width=0.33\textwidth]{img2/GestioneMappe-MappaDaAggiornare}
				\label{fig:GestioneMappe-MappaDaAggiornare}} \quad
				\hspace{1.5cm}
				\subfloat[][Lista delle mappe supportate e disponibili per il download]
				{\includegraphics[width=0.33\textwidth]{img2/GestioneMappe-DownloadMappe} 
				\label{fig:GestioneMappe-DownloadMappe}} \\
				\subfloat[][Avviso se si tenta il download di una mappa già installata]
				{\includegraphics[width=0.33\textwidth]{img2/GestioneMappe-MappaGiaScaricata}
				\label{fig:GestioneMappe-MappaGiaScaricata}} \\
				\caption{Scaricare e installare una mappa}
			\end{figure}
		
	% --------------------------------------
		
	\newpage
	\subsection{Area sviluppatore}
		L'applicazione rende disponibile a chi è in possesso della \textbf{password sviluppatore} un'area per poter accedere ai file log creati dall'applicazione durante il suo utilizzo.
		
		\begin{framed}
			\textbf{Nota:} questa funzionalità è creata specificatamente per chi dovrà effettuare dei test sul campo dell'applicazione e raccogliere dati sull'applicazione. Il normale utilizzo dell'applicazione \textbf{non richiede} l'uso di questa funzionalità.
		\end{framed}
		
		Tale materiale ha lo scopo di fornire tutte le informazioni ritenute utili sugli ultimi utilizzi della  funzionalità di navigazione. Questo può essere d'aiuto per espandere il suo potenziale o migliorare la mappatura dell'edificio con i dispositivi \gls{beacon}.
		
		Per accedere all'area sviluppatore:
		\begin{enumerate}
			\item apri il menu dell'applicazione selezionando l'icona $\equiv$;
			\item seleziona \textbf{Area sviluppatore}; %figure inline 
			\item inserisci la \textbf{password sviluppatore} (figura \ref{fig:AreaSviluppatore-InserisciPassword}), ti verrà mostrata la lista dei file log in ordine di data e ora (figura \ref{fig:AreaSviluppatore-ListaLog});
			
		\begin{framed}
			\textbf{Nota:} la password sviluppatore verrà chiesta soltanto la prima volta, gli accessi all'area successivi si salterà tale passaggio. Ad ogni reinstallazione dell'applicazione o cancellazione dei dati la password verrà richiesta nuovamente.
		\end{framed}
					
			\item seleziona il log di interesse per visualizzarlo in dettaglio (figura \ref{fig:AreaSviluppatore-DettaglioLog}) e il pulsante in basso a destra se si desidera eliminarlo;
			
			\item seleziona il pulsante in basso a destra per avviare la registrazione di nuovi dati in un nuovo file log. Dalla nuova schermata seleziona nuovamente il pulsante per fermare la registrazione di dati e salvare il file log (figura \ref{fig:AreaSviluppatore-NuovoLog}).
		\end{enumerate}
		
			\begin{figure} [p]
				\centering
				\subfloat[][Schermata in cui si richiede la password per sbloccare l'area sviluppatore]
				{\includegraphics[width=0.33\textwidth]{img/AreaSviluppatore-InserisciPassword}
				\label{fig:AreaSviluppatore-InserisciPassword}} \quad
				\hspace{1.5cm}
				\subfloat[][Schermata principale dell'area sviluppatore in cui si visualizzano tutti i log salvati]
				{\includegraphics[width=0.33\textwidth]{img/AreaSviluppatore-ListaLog}
				\label{fig:AreaSviluppatore-ListaLog}} \\
				\subfloat[][Schermata di dettaglio di un log selezionato]
				{\includegraphics[width=0.33\textwidth]{img/AreaSviluppatore-DettaglioLog} 
				\label{fig:AreaSviluppatore-DettaglioLog}} \quad
				\hspace{1.5cm}
				\subfloat[][Schermata di dettaglio nuovo log avviato]
				{\includegraphics[width=0.33\textwidth]{img/AreaSviluppatore-NuovoLog}
				\label{fig:AreaSviluppatore-NuovoLog}} \\
				\caption{Area sviluppatore}
			\end{figure}
			

\end{document}