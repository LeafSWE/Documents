\documentclass[../PianoProgetto.tex]{subfiles}

\begin{document}

\section{Modello di sviluppo}

	Il modello di sviluppo scelto per il prodotto\g\ è il modello incrementale\g : il progetto viene suddiviso in fasi ed il completamento di ogni fase\g\ è indicato da una milestone\g .
	Il proponente, al termine di ogni fase\g, può valutare il sistema prodotto fino a quel momento e fornire un feedback prezioso.
	Per agevolare il coinvolgimento del proponente, il progetto sarà suddiviso in fasi di breve durata.
	\begin{description}
	
	\item[Fase A - Analisi:] questa fase\g\ prevede quattro attività:
		\begin{itemize}
		\item individuazione degli strumenti necessari al lavoro collaborativo;
		\item individuazione degli strumenti adatti alla redazione della documentazione;
		\item individuazione del progetto da sviluppare;
		\item analisi dei requisiti del progetto che si intende sviluppare.
		\end{itemize}
		Questa fase\g\ si conclude con la \revisionedeirequisiti\ che consente di avere un riscontro sulle intenzioni del proponente.


	\item[Fase AD - Analisi di Dettaglio:]
	in questa fase\g\ si procede al consolidamento dei requisiti, individuati nella fase A, attraverso una nuova analisi.
	Eventuali requisiti individuati dagli analisti in questa fase\g\ andranno ad aggiungersi ai requisiti individuati precedentemente. 
	Verranno apportate delle modifiche ai documenti che non rispecchiano le richieste del proponente, mentre agli altri verrà apportato un incremento.

	\item[Fase PA - Progettazione Architetturale]
		fase\g\ che segue l'incontro con il proponente previsto nella fase\g\ AD. Durante questa fase si procederà alla progettazione dell'architettura logica del sistema. Verranno incrementati i documenti delle fasi precedenti e verrà prodotta la \specificatecnica . Al termine di questa fase si organizzerà un incontro con il proponente per avere un responso sull'architettura prodotta.

	\item[Fase PDROB - Progettazione di Dettaglio e codifica dei] \ \\
		\textbf{ Requisiti Obbligatori:}
		questa fase\g\ termina con una milestone\g\ rappresentata dall'approvazione, da parte del proponente, di un software\g\ che soddisfi i requisiti obbligatori.
		Verrà apportato un incremento ai documenti prodotti nelle fasi precedenti.
		Alla \revisionediprogettazione\ si prevede la consegna del documento \definizionediprodotto .
  
  
	\item[Fase PDRD - Progettazione di Dettaglio e codifica dei Requisiti] \ \\
		\textbf{Desiderabili:}
		fase\g\ che segue immediatamente la fase\g\ 
PDROB. Questa fase termina con una milestone\g\ rappresentata dall'approvazione, da parte del proponente, di un software\g\ che soddisfi i requisiti obbligatori e i requisiti desiderabili.
		Verrà apportato un incremento ai documenti prodotti nelle fasi precedenti.


	\item[Fase PDROP - Progettazione di Dettaglio e codifica dei ] \ \\
		\textbf{Requisiti Opzionali:}
		Segue immediatamente la fase\g\ PDRD. Questa fase termina con la \revisionediqualifica , nella quale verrà presentato un software\g\ che soddisfi i requisiti obbligatori, i requisiti desiderabili e i requisiti opzionali definiti dagli \analisti.
		Verrà apportato un incremento ai documenti prodotti nelle fasi precedenti.


	\item[Fase V - Validazione:]
	segue immediatamente la Fase PDROP e in questa fase\g\ il progetto si conclude. Viene eseguita la validazione del software\g\ e, successivamente, il collaudo dello stesso.
		Questa fase\g\ termina con la \revisionediaccettazione . 
		
	\end{description}
	
		Nel caso in cui il soddisfacimento dei requisiti obbligatori richieda più tempo del previsto, la fase\g\ PDRD e la fase\g\ 
PDROP verranno ridimensionate ed, eventualmente, non avviate.
		Le fasi saranno facilmente suddivise in sottofasi meno onerose, questo permetterà un maggior controllo sull'avanzamento del progetto e dà la possibilità di applicare il modello del miglioramento continuo PDCA\g\ più frequentemente.

\end{document}
