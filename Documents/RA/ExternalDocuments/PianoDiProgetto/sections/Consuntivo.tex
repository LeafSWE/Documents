\documentclass[../PianoProgetto.tex]{subfiles}

\begin{document}

\section{Consuntivo}


\subsection{Consuntivi di fase}

\label{sec:consuntivo}

	Verranno indicate di seguito le spese effettivamente sostenute, sia per ruolo che per persona.
	 
	Il bilancio risultante potrà essere: 
	\begin{itemize}
		\item \textbf{positivo}: se il preventivo supera il consuntivo;
		\item \textbf{in pari}: se consuntivo e preventivo sono equivalenti;
		\item \textbf{negativo}: se il consuntivo supera il preventivo;
	\end{itemize}

	\subsubsection{Fase A}
		\subsubsection{Consuntivo}
		Le ore di lavoro sostenute in questa fase sono da considerarsi come ore di approfondimento personale, in quanto il gruppo \leaf{} non è ancora stato scelto come fornitore ufficiale per il progetto \progetto.
		
		Tali dati riguardano quindi le ore non rendicontate.

		
\begin{table}[h]
		\centering
		\begin{tabular}{l * {2}{c}}
			\toprule
			\textbf{Ruolo} & \textbf{Ore} & \textbf{Costo (\euro{})} \\
			\midrule
			Responsabile &	33 (+7) & 990,00 (+210,00)\\
			%\midrule
			Amministratore & 87 (+12) & 1.740,00 (+240,00)\\
			%\midrule
			Progettista & 0 & 0,00 \\
			%\midrule
			Analista & 86 (+3) & 2.150,00 (+75,00)\\
			%\midrule
			Programmatore & 0 & 0,00 \\
			%\midrule
			Verificatore & 74 (-14) & 1.110,00 (-210,00)\\
			\midrule
			\textbf{Totale Preventivo} & 280
 & 5.990,00
 \\		
			\textbf{Totale Consuntivo} & 288 & 6.305,00
 \\
			\midrule
			\textbf{Differenza} & +8 & +315,00 \\
			\bottomrule
		\end{tabular}
		
		\caption{Fase A - Consuntivo}
		\label{tab:consuntivoA}
		
	\end{table}		
		
		\subsubsection{Conclusioni}	
		Come si evince dalla tabella \ref{tab:consuntivoA}, che presenta i dati relativi al consuntivo della fase A, è stato necessario investire più tempo del previsto nei ruoli di \responsabilediprogetto{}, \amministratore{} e \analista, di conseguenza il bilancio risultante è \textbf{negativo}.
		
		L'attività del \responsabilediprogetto{} ha richiesto più tempo del previsto a causa dell'inesperienza nell'ambito della pianificazione e della mancanza di progetti conosciuti sui quali basare la preventivazione dei costi.		
		
		L'attività degli \amministratori{} ha richiesto più tempo del previsto in quanto è stato necessario apportare modifiche non banali al software adottato per il tracciamento dei requisiti.
		
		L'attività degli \analisti{} ha richiesto più tempo del previsto in quanto il capitolato scelto richiede una buona dose di innovazione e ricerca che, in questa fase, ha impattato sulla specifica dei casi d'uso e dei requisiti.

	\subsubsection{Fase AD}
	\subsubsection{Consuntivo}
	Il gruppo dopo aver affrontato la \revisionedeirequisiti\ è diventato fornitore ufficiale. Le ore prese in considerazione sono ore rendicontate. 
	
	\begin{table}[h]
		\centering
		\begin{tabular}{l * {2}{c}}
			\toprule
			\textbf{Ruolo} & \textbf{Ore} & \textbf{Costo (\euro{})} \\
			\midrule
			Responsabile &	9 (+1) & 270,00 (+30,00)\\
			%\midrule
			Amministratore &  14 (+6) & 280,00 (+120,00)\\
			%\midrule
			Progettista & 0 & 0,00 \\
			%\midrule
			Analista & 20 (+9) & 500,00 (+225,00)\\
			%\midrule
			Programmatore & 0 & 0,00 \\
			%\midrule
			Verificatore & 38 (-11) & 570,00 (-165,00)\\
			\midrule
			\textbf{Totale Preventivo} & 81
			& 1620,00
			\\		
			\textbf{Totale Consuntivo} & 86 & 1830,00 
			\\
			\midrule
			\textbf{Differenza} & +5 & +210,00 \\
			\bottomrule
		\end{tabular}
		
		\caption{Fase AD - Consuntivo}
		\label{tab:consuntivoAD}
		
	\end{table}		
	
	\subsubsection{Conclusioni}	
     Anche in questa fase il consuntivo ha avuto esito \textbf{negativo}.
     Le ore spese in più dal gruppo derivano da una pianificazione non particolarmente precisa che non ha tenuto conto degli imprevisti presentati nei primi cinque giorni del periodo.
     Come da consuntivo si notano che sono state spese ore non previste nei ruoli di \amministratore\ e \analista. Queste ore sono state impiegate per effettuare le correzioni comunicate dal committente e nel caso degli \amministratori\ c'è stato il bisogno di rivedere il \pianodiqualifica. 
	
    \subsubsection{Fase PA}
	\subsubsection{Consuntivo}
	Il gruppo dopo aver superato la fase AD è passato nella fase nella quale ha dovuto effettuare la progettazione architetturale del software\g. Le ore prese in considerazione sono ore rendicontate. 
	
	\begin{table}[h]
		\centering
		\begin{tabular}{l * {2}{c}}
			\toprule
			\textbf{Ruolo} & \textbf{Ore} & \textbf{Costo (\euro{})} \\
			\midrule
			Responsabile &		10 (-1) & 300,00  (-30,00) \\
			%\midrule
			Amministratore &	19 (-2) & 380,00  (-40,00) \\
			%\midrule
			Progettista & 		45 (+7) & 990,00  (+154,00)\\
			%\midrule
			Analista & 			75	(-10)	& 1875,00   (-250,00)       \\
			%\midrule
			Programmatore & 	0		& 0,00 				\\
			%\midrule
			Verificatore & 		18 (-4) & 270,00 (-60,00)	\\
			\midrule
			\textbf{Totale Preventivo} & 167
			& 3815,00
			\\		
			\textbf{Totale Consuntivo} & 157 & 3589,00 
			\\
			\midrule
			\textbf{Differenza} & -10 & -226,00 \\
			\bottomrule
		\end{tabular}
		
		\caption{Fase PA - Consuntivo}
		\label{tab:consuntivoPA}
		
	\end{table}		
	
	\subsubsection{Conclusioni}	
     In questa fase, l'esito del consuntivo è \textbf{positivo}. 
     Ciò è dovuto principalmente ad una ripianificazione delle ore effettuata nella fase AD e ad una sovrastima delle ore da \analista. D'altro canto, le ore del \progettista\ (ruolo principale durante questa fase), sono state sottostimate.
     
	\subsubsection{Fase PDROB}
	\subsubsection{Consuntivo}
	Il gruppo, dopo aver terminato la fase PA, è passato nella fase dedicata alla progettazione di dettaglio del software\g. Le ore prese in considerazione sono ore rendicontate. 
	
	\begin{table}[h]
		\centering
		\begin{tabular}{l * {2}{c}}
			\toprule
			\textbf{Ruolo} & \textbf{Ore} & \textbf{Costo (\euro{})} \\
			\midrule
			Responsabile &		20 (-8) & 600,00  (-240,00) \\
			%\midrule
			Amministratore &	5 (-1) & 100,00  (-20,00) \\
			%\midrule
			Progettista & 		62 (+36) & 1364,00  (+792,00)\\
			%\midrule
			Analista & 			28	(-15)	& 700,00   (-375,00)       \\
			%\midrule
			Programmatore & 	44	(-20)	& 660,00 	(-300,00)			\\
			%\midrule
			Verificatore & 		20 (-2) & 300,00 (-30,00)	\\
			\midrule
			\textbf{Totale Preventivo} & 179
			& 3724,00
			\\		
			\textbf{Totale Consuntivo} & 169 & 3551 
			\\
			\midrule
			\textbf{Differenza} & -10 & -173,00 \\
			\bottomrule
		\end{tabular}
		
		\caption{Fase PDROB - Consuntivo}
		\label{tab:consuntivoPDROB}
		
	\end{table}	
	
	\subsubsection{Conclusioni}	
     In questa fase, l'esito del consuntivo è \textbf{positivo}. 
     Ciò è dovuto a una sovrastima delle ore di determinati ruoli, come il \responsabilediprogetto\ e l'\analista\, e ad un ritardo nelle attività che ha concentrato principalmente i membri del team come \progettisti, togliendo molte ore al ruolo di \programmatore, andando a diminuire il monte ore totale.
	
	\subsubsection{Fase PDRD}
	\subsubsection{Consuntivo}
	Il gruppo, dopo aver terminato la fase PDROB, è passato nella fase dedicata alla codifica del software\g. Le ore prese in considerazione sono ore rendicontate. 
	
	\begin{table}[h]
		\centering
		\begin{tabular}{l * {2}{c}}
			\toprule
			\textbf{Ruolo} & \textbf{Ore} & \textbf{Costo (\euro{})} \\
			\midrule
			Responsabile &		5 & 150,00  \\
			%\midrule
			Amministratore &	9 (-4) & 180,00  (-80,00) \\
			%\midrule
			Progettista & 		10 (-2) & 220,00  (-44,00)\\
			%\midrule
			Analista & 			13	(-4)	& 325,00   (-100,00)       \\
			%\midrule
			Programmatore & 	20	(+40)	& 300,00 	(+600,00)			\\
			%\midrule
			Verificatore & 		47 (-6) & 705,00 (-90,00)	\\
			\midrule
			\textbf{Totale Preventivo} & 104
			& 1880,00
			\\		
			\textbf{Totale Consuntivo} & 128 & 2166,00
			\\
			\midrule
			\textbf{Differenza} & +24 & +286,00 \\
			\bottomrule
		\end{tabular}
		
		\caption{Fase PDRD - Consuntivo}
		\label{tab:consuntivoPDRD}
		
	\end{table}			
	
	\subsubsection{Conclusioni}	
     In questa fase, l'esito del consuntivo è \textbf{negativo}. 
     Ciò è dovuto ad un ritardo accumulato nella fase precedente nella codifica del software\g. Nel complesso, ciò non influisce negativamente sul consuntivo, ma ha richiesto da parte dei membri del gruppo un maggior impegno.
     
     
     \subsubsection{Fase PDROP}
	\subsubsection{Consuntivo}
	Il gruppo, dopo aver terminato la fase PDRD, prosegue nella codifica dei requisiti desiderabili e opzionali se possibile. Le ore prese in considerazione sono ore rendicontate.
	
	\newpage
	\begin{table}[h]
		\centering
		\begin{tabular}{l * {2}{c}}
			\toprule
			\textbf{Ruolo} & \textbf{Ore} & \textbf{Costo (\euro{})} \\
			\midrule
			Responsabile &		9 (-2) & 270,00 (-60,00) \\
			%\midrule
			Amministratore &	21 (-5) & 420,00  (-100,00) \\
			%\midrule
			Progettista & 		16 (-6) & 352,00  (-132,00)\\
			%\midrule
			Analista & 			20	(-10)	& 500,00   (-250,00)       \\
			%\midrule
			Programmatore & 	8	(+15)	& 120,00 	(+225,00)			\\
			%\midrule
			Verificatore & 		21 (-1) & 315,00 (-15,00)	\\
			\midrule
			\textbf{Totale Preventivo} & 95
			& 1977,00
			\\		
			\textbf{Totale Consuntivo} & 86 & 1.645,00
			\\
			\midrule
			\textbf{Differenza} & -9 & -332,00 \\
			\bottomrule
		\end{tabular}
		
		\caption{Fase PDROP - Consuntivo}
		\label{tab:consuntivoPDROP}
		
	\end{table}			
	
	\subsubsection{Conclusioni}	
     In questa fase, l'esito del consuntivo è \textbf{positivo}. 
     Ciò è dovuto agli impegni di quasi metà del gruppo che durante la fase ha avuto scarse disponibilità. Nonostante questo i membri pienamente operativi sono riusciti ad eseguire gran parte delle attività pianificate sopperendo l'assenza degli altri membri. Le ore da Analista sono state sovrastimate mentre sono state sottostimate quelle da programmatore, ciò è dovuto per un ritardo sulla codifica dei requisiti funzionali desiderabili della fase precedente. Il risultato positivo sarà investito nella successiva fase.
     
     
     \subsubsection{Fase V}
	\subsubsection{Consuntivo}
	Il gruppo, dopo aver terminato la fase PDROP, è passato nella fase dedicata alla verifica e validazione del prodotto, inoltre ha eseguito la stesura del documento delle prove pratiche del prodotto richieste dal proponente. Le ore prese in considerazione sono ore rendicontate.
	
	\newpage
	\begin{table}[h]
		\centering
		\begin{tabular}{l * {2}{c}}
			\toprule
			\textbf{Ruolo} & \textbf{Ore} & \textbf{Costo (\euro{})} \\
			\midrule
			Responsabile &		10 (-3) & 300,00 (-90,00) \\
			%\midrule
			Amministratore &	15 (-5) & 300,00  (-100,00) \\
			%\midrule
			Progettista & 		7 (-4) & 154,00  (-88,00)\\
			%\midrule
			Analista & 			0	(+5)	& 0,00   (+125,00)       \\
			%\midrule
			Programmatore & 	22	(+7)	& 330,00 	(+105,00)			\\
			%\midrule
			Verificatore & 		55 (0) & 825,00 (0,00)	\\
			\midrule
			\textbf{Totale Preventivo} & 109
			& 1909,00
			\\		
			\textbf{Totale Consuntivo} & 109 & 1.861,00
			\\
			\midrule
			\textbf{Differenza} & 0 & -48,00 \\
			\bottomrule
		\end{tabular}
		
		\caption{Fase V - Consuntivo}
		\label{tab:consuntivoV}
		
	\end{table}			
	
	\subsubsection{Conclusioni}	
     In questa fase, l'esito del consuntivo è \textbf{positivo}. 
     Ciò è dovuto ad una distribuzione errata delle ore nel preventivo, le ore da programmatore sono aumentate come previsto alla conclusione della precedente fase. Inoltre si sono rendicontate ore da analista poiché, in accordo con il proponente, a seguito di una riunione interna si è deciso di modificare le priorità di alcuni requisiti opzionali e desiderabili.
    
    
    
\newpage
	\subsection{Consuntivo finale}
		Viene di seguito riportato, in formato tabellare, il consuntivo finale del progetto, indicante le spese effettivamente sopportate, sia per ruolo che per persona.

	\begin{table}[H]
		\centering
		\begin{tabular}{l * {2}{c}}
			\toprule
			\textbf{Ruolo} & \textbf{Ore} & \textbf{Costo (\euro{})} \\
			\midrule
			Responsabile & 50    & 1550,00 \\
			Amministratore  & 72    & 1.440,00 \\
			Progettista  & 171    & 3.762,00 \\
			Analista & 131    & 3.275,00 \\
			Programmatore  & 136    & 2040,00 \\
			Verificatore  & 175    & 2.625,00 \\
			\toprule
			\textbf{Totale Preventivo}  & 735   & 14.925,00 \\
			\textbf{Totale Consuntivo} & 735 	& 14.642,00 \\
			\midrule
			\textbf{Differenza} & 0 & -283,00 \\
			\bottomrule
		\end{tabular}
		\caption{Consuntivo finale - Costi e ore per ruolo}
		\label{tab:consuntivo_finale_costi}
	\end{table}


	\begin{table}[H]
		%\centering
		\begin{tabularx}{\textwidth}{l  * {6}{C}  c}
			\toprule
			\textbf{Nominativo} & \textbf{Rp} & \textbf{Am} & \textbf{Pt} 
						& \textbf{An} & \textbf{Pm} & \textbf{Ve} & \textbf{Ore totali} \\
			\midrule
			Andrighetto Cristian & 10 & 7 & 22 & 20 & 11 & 35 &	105 \\
			%\midrule
			Bicego Eduard & 7 & 17 & 21 & 19 & 20 & 21 & 105 \\
			%\midrule
			Castello Davide & 5 & 5 & 33 & 18 & 24 & 20 & 105 \\
			%\midrule
			Conti Oscar Elia & 7 & 16 & 21 & 20 & 20 & 21 & 105 \\
			%\midrule
			Tavella Federico &	5 & 9 & 22 & 14 & 22 & 33 & 105 \\
			%\midrule
			Tombolato Andrea & 7 & 8 & 26 & 19 & 20 & 25 & 105 \\
			%\midrule
			Zanella Marco & 9 & 10 & 26 & 21 & 19 & 20 & 105 \\
			\toprule			
			\textbf{Ore Totali Ruolo Preventivo} & 63 & 83 & 140 & 156 & 94 & 199 & 735 \\
			\midrule
			\textbf{Ore Totali Ruolo Consuntivo} & 50 & 72 & 171 & 131 & 136 & 175 & 735 \\
			\bottomrule
		\end{tabularx}
		\caption{Consuntivo finale suddivisione delle ore di lavoro}
		\label{tab:consuntivo_finale_rendicontate_ore}
	\end{table}     
     
     \subsubsection{Conclusioni}
     	Come illustrano le tabelle, i ruoli da progettista e programmatore sono stati fortemente sottostimati. L'incremento delle ore nel primo caso deriva principalmente dalla poca esperienza del mondo Android e dalle numerose complicazioni riscontrate utilizzando tale tecnologia. Tali problemi sono poi ricaduti nell'attività di codifica incrementando di fatto il numero delle ore da programmatore. Nonostante queste difficoltà incontrate il risparmio rispetto a quanto preventivato ammonta a 283,00 \euro{}. Il prezzo finale è quindi di \textbf{14.642,00} \euro{}.
     
\end{document}
