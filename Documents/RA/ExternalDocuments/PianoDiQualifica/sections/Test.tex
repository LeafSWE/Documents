\documentclass[../PianoDiQualifica.tex]{subfiles}

\begin{document}
\begin{appendices}

\section{Test}
	\subsection{Test di accettazione}
	Il test di accettazione serve ad accertare il soddisfacimento dei \textbf{requisiti utente}. Viene effettuato in presenza del proponente che può, in questo modo, avere un primo approccio con il prodotto software\g\ terminato. Nel caso in cui il test avesse esito positivo, si può procedere al rilascio ufficiale del prodotto\g\ 
sviluppato.\\
	Di seguito vengono riportati i test di accettazione definiti dal gruppo \leaf.
	
\begin{longtabu} to \textwidth{X[0.9] X[1.5] X[2.5] X}
\toprule
\textbf{Codice} & \textbf{Requisito} & \textbf{Descrizione} & \textbf{Stato}\\
\midrule
\endhead
\arrayrulecolor{gray}
TA1 & L'utente deve poter verificare che sia possibile navigare all'interno di un edificio utilizzando l'applicazione. & All'utente è chiesto di: \begin{itemize} \item attivare il bluetooth; \item accedere alla sezione preposta alla navigazione; \item scegliere la destinazione all'interno dell'edificio; \item confermare la destinazione scelta; \item verificare che venga data la possibilità di avviare la navigazione. \end{itemize} & N.I. \\ 
\midrule 
TA1.1 & L'utente deve poter verificare che sia possibile ricercare una destinazione per la navigazione. & All'utente è chiesto di: \begin{itemize} \item scegliere di ricercare la destinazione per nome; \item scegliere di ricercare la destinazione per categoria. \end{itemize} & N.I. \\ 
\midrule 
TA1.1.1 & L'utente deve poter verificare che sia possibile ricercare una destinazione per nome. & All'utente è chiesto di: \begin{itemize} \item inserire il nome di una destinazione; \item verificare che sia possibile confermare l'inserimento fatto. \end{itemize} & N.I. \\ 
\midrule 
TA1.1.1.1 & L'utente deve poter verificare che sia possibile inserire il nome di una destinazione. & All'utente è chiesto di: \begin{itemize} \item inserire il nome di una destinazione; \item verificare che la destinazione voluta sia stata inserita. \end{itemize} & N.I. \\ 
\midrule 
TA1.1.1.2 & L'utente deve poter verificare che venga segnalato un errore qualore venga inserita una destinazione non prevista dal sistema. & All'utente è chiesto di: \begin{itemize} \item inserire una destinazione non prevista dal sistema; \item verificare che venga visualizzato un errore che spieghi che la destinazione inserita non è presente tra quelle disponibili; \item verificare che venga data la possibilità di inserire un'altra destinazione. \end{itemize} & N.I. \\ 
\midrule 
TA1.1.2 & L'utente deve poter verificare che sia possibile ricercare una destinazione per categoria. & All'utente è chiesto di: \begin{itemize} \item scegliere una delle categorie proposte \item scegliere una delle destinazioni presenti all'interno della categoria scelta; \item verificare che sia possibile confermare la scelta fatta. \end{itemize} & N.I. \\ 
\midrule 
TA1.1.2.1 & L'utente deve verificare che sia possibile scegliere una categoria tra quelle proposte. & All'utente è chiesto di: \begin{itemize} \item verificare che l'applicazione fornisca una o più categorie di destinazioni; \item verificare che sia possibile scegliere una di queste categorie. \end{itemize} & N.I. \\ 
\midrule 
TA1.1.2.2 & L'utente deve verificare che sia possibile scegliere una destinazione tra i risultati di una ricerca. & All'utente è chiesto di: \begin{itemize} \item verificare che all'interno di una categoria siano proposte una o più destinazioni; \item verificare che sia possibile scegliere una di queste destinazioni; \item effettuare la ricerca di una destinazione (prevista dal sistema) per nome; \item verificare che sia possibile scegliere una delle destinazioni restituite dalla ricerca. \end{itemize} & N.I. \\ 
\midrule 
TA1.1.3 & L'utente deve poter verificare che sia possibile confermare una destinazione. & All'utente è chiesto di: \begin{itemize} \item confermare la destinazione scelta; \item verificare che venga data la possibilità di avviare la navigazione verso la destinazione scelta. \end{itemize} & N.I. \\ 
\midrule 
TA1.2 & L'utente deve poter verificare che sia possibile avviare la navigazione. & All'utente è chiesto di: \begin{itemize} \item confermare l'avvio della navigazione; \item verificare che venga fornita un'indicazione testuale per raggiungere la destinazione scelta. \end{itemize} & N.I. \\ 
\midrule 
TA1.2.1 & L'utente deve poter verificare che l'indicazione sia fornita in forma testuale. & All'utente è chiesto di: \begin{itemize} \item verificare che l'indicazione fornita sia un testo scritto. \end{itemize} & N.I. \\ 
\midrule 
TA1.2.2 & L'utente deve poter verificare che l'indicazione testuale fornita per raggiungere la destinazione scelta, quando è stata avviata la navigazione, sia corretta. & All'utente è chiesto di: \begin{itemize} \item seguire l'indicazione testuale data; \item verificare di essere arrivato alla destinazione scelta. \end{itemize} & N.I. \\ 
\midrule 
TA1.2.3 & L'utente deve poter verificare che l'indicazione testuale fornita dall'applicazione abbia come punto di partenza il POI in cui l'utente si trova. & All'utente è chiesto di: \begin{itemize} \item avviare la navigazione; \item verificare che l'indicazione testuale fornita dall'applicazione abbia come punto di partenza il POI in cui l'utente si trova. \end{itemize} & N.I. \\ 
\midrule 
TA1.2.4 & L'utente deve poter verificare che sia possibile confermare l'avvio della navigazione. & All'utente viene chiesto di: \begin{itemize} \item impostare una destinazione; \item confermare la destinazione scelta; \item verificare che la navigazione sia stata avviata. \end{itemize} & N.I. \\ 
\midrule 
TA1.3 & L'utente deve verificare che sia possibile interrompere la navigazione. & All'utente è chiesto di: \begin{itemize} \item scegliere di interrompere la navigazione; \item verificare che la navigazione si interrompa. \end{itemize} & N.I. \\ 
\midrule 
TA1.4 & L'utente deve verificare che sia possibile accedere a delle informazioni più dettagliate riguardanti il percorso da seguire per raggiungere la destinazione. & All'utente è chiesto di: \begin{itemize} \item scegliere di visualizzare le fotografie della prossima area; \item scegliere di ricevere delle indicazioni testuali estese per raggiungere la prossima area; \item scegliere di accedere alla lista completa delle indicazioni per raggiungere la destinazione scelta. \end{itemize} & N.I. \\ 
\midrule 
TA1.4.1 & L'utente deve verificare che sia possibile visualizzare le fotografie della prossima area da raggiungere. & All'utente è chiesto di: \begin{itemize} \item verificare che siano visualizzate le fotografie che ritraggono la prossima area da raggiungere. \end{itemize} & N.I. \\ 
\midrule 
TA1.4.2 & L'utente deve verificare che sia possibile visualizzare delle indicazioni testuali estese dettagliate riguardanti le azioni da compiere per raggiungere la prossima area. & All'utente è chiesto di: \begin{itemize} \item verificare che sia visualizzata una descrizione tesuale estesa che descriva in modo dettagliato le azioni da compiere per raggiungere la prossima area. \end{itemize} & N.I. \\ 
\midrule 
TA1.4.3 & L'utente deve verificare che sia possibile visualizzare la lista completa delle indicazioni da seguire per raggiungere la destinazione scelta. & All'utente è chiesto di: \begin{itemize} \item verificare che sia visualizzata la lista completa delle indicazioni da seguire per raggiungere la destinazione scelta. \end{itemize} & N.I. \\ 
\midrule 
TA1.4.4 & L'utente deve verificare che venga visualizzato un errore nel caso in cui acceda alla foto del prossimo POI con connessione Internet non attiva. & All'utente è chiesto di: \begin{itemize} \item disattivare la connessione Internet \item accedere alle fotografie del prossimo POI \item verificare che venga segnalato un errore che spieghi che il dispositivo non ha la connessione Internet attiva. \end{itemize} & N.I. \\ 
\midrule 
TA1.5 & L'utente deve poter verificare che venga segnalato un errore qualora segua un percorso differente da quello calcolato dall'applicazione. & All'utente è chiesto di: \begin{itemize} \item avviare la navigazione; \item seguire un percorso differente da quello proposto dall'applicazione; \item verificare che venga segnalato un errore che spieghi che il percorso che si sta seguendo non è quello previsto. \end{itemize} & N.I. \\ 
\midrule 
TA1.6 & L'utente deve poter verificare che venga segnalato un errore nel caso in cui voglia avviare la navigazione all'interno di un'area dove non è presente il segnale di alcun beacon. & All'utente è chiesto di: \begin{itemize} \item posizionarsi in un'area dove non è presente il segnale di alcun beacon; \item avviare la navigazione; \item verificare che venga segnalato un errore che spieghi che in quell'area non è stato rilevato il segnale di alcun beacon. \end{itemize} & N.I. \\ 
\midrule 
TA1.7 & L'utente deve poter verificare che venga segnalato un errore nel caso in cui voglia avviare la navigazione con la connessione Internet del proprio dispositivo non attiva. & All'utente è chiesto di: \begin{itemize} \item disattivare la connessione internet; \item avviare la navigazione; \item verificare che venga segnalato un errore che spieghi che il dispositivo non ha la connessione Internet attiva. \end{itemize} & N.I. \\ 
\midrule 
TA1.8 & L'utente deve poter verificare che venga segnalato un errore nel caso in cui voglia avviare la navigazione e la mappa installata nel proprio dispositivo differisce dall'ultima versione online della mappa. & All'utente è chiesto di: \begin{itemize} \item non aggiornare una mappa che richieda un aggiornamento; \item avviare la navigazione; \item verificare che venga segnalato un errore che spieghi che la mappa presente nel dispositivo non è l'ultima versione della mappa per quell'edificio. \end{itemize} & N.I. \\ 
\midrule 
TA1.9 & L'utente deve poter verificare che venga segnalato un errore nel caso in cui si rilevi un beacon all'interno di un edificio mappato e non sia installata la mappa per quell'edificio. & All'utente è chiesto di: \begin{itemize} \item entrare in un edificio di cui non dispone della mappa; \item avviare l'applicazione; \item verificare che venga segnalato un errore che spieghi che non è presente nel dispositivo una mappa per quell'edificio. \end{itemize} & N.I. \\ 
\midrule 
TA2 & L'utente deve poter verificare che sia possibile accedere alle informazioni dell'edificio in cui ci si trova. & All'utente è chiesto di: \begin{itemize} \item scegliere di accedere alle informazioni generali sull'edificio in cui ci si trova; \item scegliere di accedere alla lista completa di tutti i POI presenti nell'edificio in cui si trova; \item scegliere di accedere alla lista dei POI associati ai beacon rilevati alla posizione dell’utente. \end{itemize} & N.I. \\ 
\midrule 
TA2.1 & L'utente deve poter verificare che sia possibile accedere al nome dell'edificio. & All'utente è chiesto di: \begin{itemize} \item verificare che sia presente un nome per l'edificio. \end{itemize} & N.I. \\ 
\midrule 
TA2.2 & L'utente deve poter verificare che sia possibile accedere alla descrizione dell'edificio dell'edificio. & All'utente è chiesto di: \begin{itemize} \item verificare che sia presente una descrizione per l'edificio. \end{itemize} & N.I. \\ 
\midrule 
TA2.3 & L'utente deve verificare che sia possibile accedere all'indirizzo dell'edificio. & All'utente è chiesto di: \begin{itemize} \item verificare che sia presente l'indirizzo per l'edificio. \end{itemize} & N.I. \\ 
\midrule 
TA2.4 & L'utente deve verificare che sia possibile accedere aglio orari dell'edificio. & All'utente è chiesto di: \begin{itemize} \item verificare che siano presenti gli orari di apertura dell'edificio. \end{itemize} & N.I. \\ 
\midrule 
TA2.5 & L'utente deve poter verificare che sia possibile accedere alla lista completa di tutti i POI presenti nell'edificio in cui si trova. & All'utente è chiesto di: \begin{itemize} \item verificare che venga visualizzata la lista completa di tutti i POI presenti nell'edificio. \end{itemize} & N.I. \\ 
\midrule 
TA2.6 & L'utente deve poter verificare che sia possibile accedere alla lista dei POI associati ai beacon rilevati alla posizione dell’utente. & All'utente è chiesto di: \begin{itemize} \item verificare che venga visualizzata la lista dei POI associati ai beacon rilevati alla posizione dell’utente; \item verificare che sia possibile accedere alle informazioni riguardanti uno specifico POI nella lista. \end{itemize} & N.I. \\ 
\midrule 
TA2.6.1 & L'utente deve poter verificare che sia possibile accedere alle informazioni riguardanti uno specifico POI. & All'utente è chiesto di: \begin{itemize} \item verificare che sia possibile accedere all'identificativo del POI; \item verificare che sia possibile accedere alla descrizione del POI. \end{itemize} & N.I. \\ 
\midrule 
TA2.6.1.1 & L'utente deve poter verificare che sia possibile accedere all'identificativo di uno specifico POI. & All'utente è chiesto di: \begin{itemize} \item verificare che sia presente un identificativo per il POI. \end{itemize} & N.I. \\ 
\midrule 
TA2.6.1.2 & L'utente deve poter verificare che sia possibile accedere alla descrizione di uno specifico POI. & All'utente è chiesto di: \begin{itemize} \item verificare che sia presente una descrizione per il POI. \end{itemize} & N.I. \\ 
\midrule 
TA2.7 & L'utente deve verificare che venga visualizzato un errore nel caso in cui acceda alle informazioni di un edificio con connessione internet non attiva. & All'utente è chiesto di: \begin{itemize} \item disattivare la connessione internet; \item accedere alle informazioni di un edificio; \item verificare che venga segnalato un errore che spieghi che il dispositivo non ha la connessione Internet attiva. \end{itemize} & N.I. \\ 
\midrule 
TA2.8 & L'utente deve verificare che venga visualizzato un errore nel caso in cui acceda alle informazioni dell'edificio e la versione della mappa non coincida con l'ultima versione disponibile. & All'utente è chiesto di: \begin{itemize} \item non aggiornare una mappa che richieda un aggiornamento; \item accedere alle informazioni di un edificio; \item verificare che venga segnalato un errore che spieghi che il dispositivo non è presenta l'ultima versione di mappa disponibile. \end{itemize} & N.I. \\ 
\midrule 
TA3 & L'utente deve poter verificare che sia possibile gestire gli aspetti relativi all'applicazione. & All'utente è chiesto di: \begin{itemize} \item gestire le mappe dell'applicazione; \item gestire le preferenze di navigazione. \end{itemize} & N.I. \\ 
\midrule 
TA3.1 & L'utente deve poter verificare che sia possibile gestire le mappe dall'applicazione. & All'utente è chiesto di: \begin{itemize} \item scegliere di gestire le mappe installate sul proprio dispositivo; \item scegliere di gestire le mappe non presenti sul proprio dispositivo. \end{itemize} & N.I. \\ 
\midrule 
TA3.1.1 & l'utente deve poter verificare che sia possibile gestire le mappe presenti sul proprio dispositivo. & All'utente è chiesto di: \begin{itemize} \item scegliere di accedere alle mappe installate; \item scegliere di aggiornare una mappa installata; \item scegliere di rimuovere una mappa installata; \item scegliere di accedere alle informazioni riguardanti una mappa. \end{itemize} & N.I. \\ 
\midrule 
TA3.1.1.1 & L'utente deve poter verificare che sia possibile accedere alle mappe installate sul proprio dispositivo. & All'utente è chiesto di: \begin{itemize} \item accedere alle mappe installate; \item se l'utente non ha installato alcuna mappa in precedenza verificare che la sezione sia vuota, in caso contrario verificare che la sezione contenga le mappe installate in precedenza. \end{itemize} & N.I. \\ 
\midrule 
TA3.1.1.2 & L'utente deve poter verificare che sia possibile aggiornare una mappa (che richieda un aggiornamento) presente sul proprio dispositivo. & All'utente è chiesto di: \begin{itemize} \item scegliere una mappa (che richieda un aggiornamento) presente sul proprio dispositivo; \item aggiornare tale mappa; \item verificare che sia possibile avviare la navigazione all'interno dell'edificio di cui è stata aggiornata la mappa. \end{itemize} & N.I. \\ 
\midrule 
TA3.1.1.3 & L'utente deve poter verificare che sia possibile rimuovere una mappa dal proprio dispositivo. & All'utente è chiesto di: \begin{itemize} \item scegliere una mappa tra quelle presenti sul proprio dispositivo; \item rimuovere la mappa scelta; \item verificare che la mappa rimossa non sia più presente sul proprio dispositivo. \end{itemize} & N.I. \\ 
\midrule 
TA3.1.1.4 & L'utente deve poter verificare che sia possibile accedere alle informazioni riguardanti una mappa presente sul proprio dispositivo. & All'utente è chiesto di: \begin{itemize} \item scegliere una mappa presente sul proprio dispositivo; \item scegliere di accedere al nome di una mappa; \item scegliere di accedere alla foto associata ad una mappa; \item scegliere di accedere all'indirizzo dell'edificio; \item scegliere di accedere alla descrizione dell'edificio; \item scegliere di accedere alla dimensione in megabyte della mappa; \item scegliere di accedere alla versione della mappa. \end{itemize} & N.I. \\ 
\midrule 
TA3.1.1.4.1 & L'utente deve poter verificare che sia possibile accedere al nome di una mappa presente sul proprio dispositivo. & All'utente è chiesto di: \begin{itemize} \item verificare che sia possibile accedere al nome di una mappa. \end{itemize} & N.I. \\ 
\midrule 
TA3.1.1.4.2 & L'utente deve poter verificare che sia possibile accedere all'indirizzo dell'edificio dalla mappa presente sul proprio dispositivo. & All'utente è chiesto di: \begin{itemize} \item verificare che sia possibile accedere all'indirizzo dell'edificio (dalla mappa). \end{itemize} & N.I. \\ 
\midrule 
TA3.1.1.4.3 & L'utente deve poter verificare che sia possibile accedere alla descrizione dell'edificio dalla mappa presente sul proprio dispositivo. & All'utente è chiesto di: \begin{itemize} \item verificare che sia possibile accedere alla descrizione dell'edificio (dalla mappa). \end{itemize} & N.I. \\ 
\midrule 
TA3.1.1.4.4 & L'utente deve poter verificare che sia possibile accedere alla dimensione in megabyte della mappa di un edificio presente sul proprio dispositivo. & All'utente è chiesto di: \begin{itemize} \item verificare che sia possibile accedere alla dimensione in megabyte della mappa di un edificio. \end{itemize} & N.I. \\ 
\midrule 
TA3.1.1.4.5 & L'utente deve poter verificare che sia possibile accedere alla versione della mappa di un edificio presente sul proprio dispositivo. & All'utente è chiesto di: \begin{itemize} \item verificare che sia possibile accedere alla versione della mappa di un edificio. \end{itemize} & N.I. \\ 
\midrule 
TA3.1.2 & L'utente deve poter verificare che sia possibile gestire le mappe non presenti sul proprio dispositivo. & All'utente è chiesto di: \begin{itemize} \item scegliere di ricercare una mappa non presente sul proprio dispositivo; \item scegliere di installare una mappa non presente sul proprio dispositivo; \item scegliere di accedere alle informazioni riguardanti una mappa non presente sul proprio dispositivo. \end{itemize} & N.I. \\ 
\midrule 
TA3.1.2.1 & L'utente deve poter verificare che sia possibile ricercare per nome (dell'edificio) una mappa non presente sul proprio dispositivo. & All'utente è chiesto di: \begin{itemize} \item inserire il nome dell'edificio di cui cerca la mappa; \item scegliere la mappa tra quelle proposte come risultati della ricerca. \end{itemize} & N.I. \\ 
\midrule 
TA3.1.2.1.1 & L'utente deve poter verificare che venga segnalato un messaggio di errore nel caso in cui l'utente voglia scaricare una mappa non prevista. & All'utente è chiesto di: \begin{itemize} \item inserire il nome di una mappa non prevista dal sistema; \item verificare che venga visualizzato un messaggio di errore che spieghi che tale mappa non è prevista. \end{itemize} & N.I. \\ 
\midrule 
TA3.1.2.1.2 & L'utente deve poter verificare che sia possibile inserire il possibile nome di una mappa. & All'utente è chiesto di: \begin{itemize} \item inserire il possibile nome di una mappa; \item verificare che il nome voluto sia stato inserito. \end{itemize} & N.I. \\ 
\midrule 
TA3.1.2.2 & L'utente deve poter verificare che sia possibile installare una nuova mappa. & All'utente è chiesto di: \begin{itemize} \item ricercare una mappa; \item scegliere una mappa tra quelle proposte nei risultati della ricerca; \item eseguire il download della mappa; \item verificare che la mappa sia presente tra quelle disponibili nel dispositivo; \item verificare che sia possibile avviare la navigazione all'interno dell'edificio di cui è stato eseguito il download della mappa. \end{itemize} & N.I. \\ 
\midrule 
TA3.1.2.3 & L'utente deve poter verificare che sia possibile accedere alle informazioni riguardanti una mappa non ancora scaricata. & All'utente è chiesto di: \begin{itemize} \item effettuare la ricerca di una mappa; \item scegliere una mappa tra i risultati della ricerca; \item scegliere di accedere al nome dell'edificio; \item scegliere di accedere alle foto riguardanti l'edificio; \item scegliere di accedere all'indirizzo dell'edificio; \item scegliere di accedere alla descrizione dell'edificio; \item scegliere di accedere alla dimensione in megabyte della mappa; \item scegliere di accedere alla versione della mappa. \end{itemize} & N.I. \\ 
\midrule 
TA3.1.2.3.1 & L'utente deve poter verificare che sia possibile accedere al nome dell'edificio dalla mappa non presente sul proprio dispositivo. & All'utente è chiesto di: \begin{itemize} \item verificare che sia possibile accedere al nome dell'edificio. \end{itemize} & N.I. \\ 
\midrule 
TA3.1.2.3.2 & L'utente deve poter verificare che sia possibile accedere all'indirizzo dell'edificio dalla mappa non presente sul proprio dispositivo. & All'utente è chiesto di: \begin{itemize} \item verificare che sia possibile accedere all'indirizzo dell'edificio. \end{itemize} & N.I. \\ 
\midrule 
TA3.1.2.3.3 & L'utente deve poter verificare che sia possibile accedere alla descrizione dell'edificio dalla mappa non presente sul proprio dispositivo. & All'utente è chiesto di: \begin{itemize} \item verificare che sia possibile accedere alla descrizione dell'edificio. \end{itemize} & N.I. \\ 
\midrule 
TA3.1.2.3.4 & L'utente deve poter verificare che sia possibile accedere alla dimensione in megabyte della mappa di un edificio non presente sul proprio dispositivo. & All'utente è chiesto di: \begin{itemize} \item verificare che sia possibile accedere alla dimensione in megabyte della mappa di un edificio. \end{itemize} & N.I. \\ 
\midrule 
TA3.1.2.3.5 & L'utente deve poter verificare che sia possibile accedere alla versione della mappa di un edificio non presente sul proprio dispositivo. & All'utente è chiesto di: \begin{itemize} \item verificare che sia possibile accedere alla versione della mappa di un edificio. \end{itemize} & N.I. \\ 
\midrule 
TA3.2 & L'utente deve poter verificare che sia possibile gestire le preferenze di navigazione. & All'utente è chiesto di: \begin{itemize} \item modificare le preferenze riguardanti la modalità di fruizione delle indicazioni; \item modificare le preferenze riguardanti il percorso. \end{itemize} & N.I. \\ 
\midrule 
TA3.2.1 & L'utente deve poter verificare che sia possibile gestire le preferenze riguardanti la modalità di fruizione delle indicazioni. & All'utente è chiesto di: \begin{itemize} \item modificare le impostazioni riguardanti le indicazioni vocali; \item modificare le impostazioni riguardanti le inidicazioni sonore. \end{itemize} & N.I. \\ 
\midrule 
TA3.2.1.1 & L'utente deve poter verificare che sia possibile attivare le indicazioni vocali, se queste sono disattivate. & All'utente è chiesto di: \begin{itemize} \item attivare le indicazioni vocali; \item verificare che all'avvio della navigazione vengano fornite le indicazioni vocali per raggiungere la destinazione scelta. \end{itemize} & N.I. \\ 
\midrule 
TA3.2.1.2 & L'utente deve poter verificare che sia possibile disattivare le indicazioni vocali, se queste sono attivate. & All'utente è chiesto di: \begin{itemize} \item disattivare le indicazioni vocali; \item verificare che all'avvio della navigazione non vengano fornite le indicazioni vocali per raggiungere la destinazione scelta. \end{itemize} & N.I. \\ 
\midrule 
TA3.2.1.3 & L'utente deve poter verificare che sia possibile attivare le inidcazioni sonore, se queste sono disattivate. & All'utente è chiesto di: \begin{itemize} \item attivare le indicazioni sonore; \item verificare che all'avvio della navigazione vengano fornite le indicazioni sonore per raggiungere la destinazione scelta. \end{itemize} & N.I. \\ 
\midrule 
TA3.2.1.4 & L'utente deve poter verificare che sia possibile disattivare le indicazioni sonore, se queste sono attivate. & All'utente è chiesto di: \begin{itemize} \item disattivare le indicazioni sonore; \item verificare che all'avvio della navigazione non vengano fornite le indicazioni sonore per raggiungere la destinazione scelta. \end{itemize} & N.I. \\ 
\midrule 
TA3.2.2 & L'utente deve poter verificare che sia possibile gestire le preferenze riguardanti il percorso da seguire. & All'utente viene chiesto di: \begin{itemize} \item modificare le impostazioni riguardanti il percorso più accessibile; \item modificare le impostazioni riguardanti il percorso con meno ascensori. \end{itemize} & N.I. \\ 
\midrule 
TA3.2.2.1 & L'utente deve poter verificare che sia possibile scegliere di seguire il percorso più accessibile per arrivare alla destinazione desiderata. & All'utente viene chiesto di: \begin{itemize} \item attivare l'impostazione riguardante il percorso più accessibile; \item verificare che all'avvio della navigazione l'applicazione fornisca un percorso che prediliga gli ascensori rispetto altre soluzioni per raggiungere la destinazione scelta. \end{itemize} & N.I. \\ 
\midrule 
TA3.2.2.2 & L'utente deve poter verificare che sia possibile scegliere di seguire il percorso con il minor numero di ascensori possibile. & All'utente viene chiesto di: \begin{itemize} \item attivare l'impostazione riguardante il percorso con il minor numero di ascensori possibile; \item verificare che all'avvio della navigazione l'applicazione fornisca un percorso che prediliga soluzioni alternative rispetto gli ascensori per raggiungere la destinazione scelta. \end{itemize} & N.I. \\ 
\midrule 
TA3.2.2.3 & L'utente deve poter verificare che sia possibile scegliere di seguire il percorso più veloce in assoluto. & All'utente viene chiesto di: \begin{itemize} \item attivare l'impostazione riguardante il percorso che è ritenuto più veloce; \item verificare che all'avvio della navigazione l'applicazione fornisca un percorso che prediliga soluzioni alternative rispetto al percorso più veloce per raggiungere la destinazione scelta. \end{itemize} & N.I. \\ 
\midrule 
TA4 & L'utente deve poter verificare che sia possibile accedere alla guida. & All'utente viene chiesto di: \begin{itemize} \item verificare che sia possibile accedere alla guida; \item verificare che la guida spieghi il funzionamento dell'applicazione. \end{itemize} & N.I. \\ 
\midrule 
TA5 & L'utente non sviluppatore deve poter verificare che sia possibile attivare le funzionalità sviluppatore. & All'utente non sviluppatore viene chiesto di: \begin{itemize} \item inserire un codice sviluppatore valido; \item confermare il codice inserito; \item verificare che siano state attivate le funzionalità sviluppatore. \end{itemize} & N.I. \\ 
\midrule 
TA5.1 & L'utente non sviluppatore deve poter verificare che venga segnalato un errore nel caso in cui venga inserito un codice sviluppatore non valido. & All'utente non sviluppatore viene chiesto di: \begin{itemize} \item inserire un codice sviluppatore non valido; \item confermare il codice inserito; \item verificare che venga visualizzato un errore che spieghi che il codice inserito non è valido; \item verificare che non siano state attivate le funzionalità di sviluppatore. \end{itemize} & N.I. \\ 
\midrule 
TA5.2 & L'utente non sviluppatore deve poter verificare che sia possibile inserire un codice sviluppatore. & All'utente non sviluppatore viene chiesto di: \begin{itemize} \item inserire un codice sviluppatore; \item verificare che il codice voluto sia stato inserito. \end{itemize} & N.I. \\ 
\midrule 
TA5.3 & L'utente non sviluppatore deve poter verificare che sia possibile confermare il codice inserito. & All'utente non sviluppatore viene chiesto di: \begin{itemize} \item inserire un codice sviluppatore; \item confermare il codice inserito; \item verificare che, se il codice inserito è valido, sono ora attive le funzionalità sviluppatore, altrimenti se non è valido viene segnalato un errore. \end{itemize} & N.I. \\ 
\midrule 
TA6 & Lo sviluppatore deve verificare che sia possibile accedere alle informazioni riguardanti i beacon rilevati. & Allo sviluppatore viene chiesto di: \begin{itemize} \item accedere all'UUID di un beacon rilevato; \item accedere al Major di un beacon rilevato; \item accedere al Minor di un beacon rilevato; \item accedere al livello di potenza del segnale di un beacon rilevato; \item accedere al livello di batteria di un beacon rilevato; \item accedere alla distanza approssimativa dal dispositivo utilizzato al beacon rilevato; \item accedere al formato di un beacon rilevato; \item accedere all'area coperta da un beacon rilevato. \end{itemize} & N.I. \\ 
\midrule 
TA6.1 & Lo sviluppatore deve verificare che sia possibile accedere all'UUID di un beacon rilevato. & Allo sviluppatore viene chiesto di: \begin{itemize} \item accedere all'UUID di un beacon rilevato; \item verificare che l'UUID rilevato corrisponda al valore corretto. \end{itemize} & N.I. \\ 
\midrule 
TA6.2 & Lo sviluppatore deve verificare che sia possibile accedere al Major di un beacon rilevato. & Allo sviluppatore viene chiesto di: \begin{itemize} \item accedere al Major di un beacon rilevato; \item verificare che il Major rilevato corrisponda al valore corretto. \end{itemize} & N.I. \\ 
\midrule 
TA6.3 & Lo sviluppatore deve verificare che sia possibile accedere al Minor di un beacon rilevato. & Allo sviluppatore viene chiesto di: \begin{itemize} \item accedere al Minor di un beacon rilevato; \item verificare che il Minor rilevato corrisponda al valore corretto. \end{itemize} & N.I. \\ 
\midrule 
TA6.4 & Lo sviluppatore deve verificare che sia possibile accedere al formato di un beacon rilevato. & Allo sviluppatore viene chiesto di: \begin{itemize} \item accedere al formato di un beacon rilevato; \item verificare che il formato rilevato corrisponda al valore corretto. \end{itemize} & N.I. \\ 
\midrule 
TA6.5 & Lo sviluppatore deve verificare che sia possibile accedere al livello di potenza del segnale di un beacon rilevato. & Allo sviluppatore viene chiesto di: \begin{itemize} \item accedere al livello di potenza del segnale di un beacon rilevato. \end{itemize} & N.I. \\ 
\midrule 
TA6.6 & Lo sviluppatore deve verificare che sia possibile accedere al livello di batteria di un beacon rilevato. & Allo sviluppatore viene chiesto di: \begin{itemize} \item accedere al livello di batteria di un beacon rilevato. \end{itemize} & N.I. \\ 
\midrule 
TA6.7 & Lo sviluppatore deve verificare che sia possibile accedere alla distanza approssimativa dal dispositivo utilizzato al beacon rilevato. & Allo sviluppatore viene chiesto di: \begin{itemize} \item accedere alla distanza approssimativa dal dispositivo utilizzato al beacon rilevato. \end{itemize} & N.I. \\ 
\midrule 
TA6.8 & Lo sviluppatore deve verificare che sia possibile accedere all'area coperta da beacon rilevato. & Allo sviluppatore viene chiesto di: \begin{itemize} \item accedere all'area coperta da beacon rilevato. \end{itemize} & N.I. \\ 
\midrule 
TA7 & Lo sviluppatore deve poter verificare che sia possibile gestire i log. & Allo sviluppatore viene chiesto di: \begin{itemize} \item Avviare un nuovo log; \item Interrompere un log precedentemente avviato; \item Accedere ad un log salvato in precedenza; \item rimuovere un log salvato in precedenza; \item salvare un log appena interrotto. \end{itemize} & N.I. \\ 
\midrule 
TA7.1 & Lo sviluppatore deve poter verificare che sia possibile avviare un nuovo log. & Allo sviluppatore viene chiesto di: \begin{itemize} \item avviare un nuovo log; \item verificare che il log sia stato avviato. \end{itemize} & N.I. \\ 
\midrule 
TA7.2 & Lo sviluppatore deve poter verificare che sia possibile interrompere precedentemente avviato. & Allo sviluppatore viene chiesto di: \begin{itemize} \item scegliere di interrompere un log precedentemente avviato; \item verificare che il log non sia più avviato. \end{itemize} & N.I. \\ 
\midrule 
TA7.3 & Lo sviluppatore deve poter verificare che sia possibile accedere ad un log salvato in precedenza. & Allo sviluppatore viene chiesto di: \begin{itemize} \item accedere ad un log salvato in precedenza; \item verificare che riesca a accedere al contenuto del log scelto. \end{itemize} & N.I. \\ 
\midrule 
TA7.4 & Lo sviluppatore deve poter verificare che sia possibile rimuovere un log salvato in precedenza. & Allo sviluppatore viene chiesto di: \begin{itemize} \item rimuovere un log salvato in precedenza; \item verificare che il log rimosso non sia più presente nella lista dei log salvati. \end{itemize} & N.I. \\ 
\midrule 
TA7.5 & Lo sviluppatore deve poter verificare che sia possibile salvare un log appena interrotto. & Allo sviluppatore viene chiesto di: \begin{itemize} \item avviare un nuovo log; \item interrompere il log precedentemente avviato; \item salvare il log appena interrotto; \item verificare che sia possibile accedere al log appena salvato. \end{itemize} & N.I. \\ 
\midrule 
TA8 & L'utente deve poter verificare che sia possibile accedere alle informazioni dei punti d'interesse dell'edificio in cui si trova. & All'utente è chiesto di: \begin{itemize} \item attivare il bluetooth; \item accedere alla sezione preposta all'esplorazione; \item avviare la scansione; \item verificare che sia possibile accedere alle informazioni dell'area circostante. \end{itemize} & N.I. \\ 
\arrayrulecolor{black}
\bottomrule
\caption{Tabella test di accettazione} \\
\end{longtabu}
	
	\subsection{Test di sistema}
	Il test di sistema verifica il comportamento dinamico del sistema completo al fine di verificare il soddisfacimento dei \textbf{requisiti software}. La maggior parte degli errori dovrebbe essere già stata identificata durante i test di unità e di integrazione. Il test di sistema viene di solito considerato appropriato per verificare il sistema anche rispetto ai requisiti non funzionali, come quelli prestazionali, di qualità e di vincolo. A questo livello, viene effettuata anche una serie di test in una struttura opportunamente mappata da beacon\g\ per verificare il corretto funzionamento del software\g\ ed evidenziare eventuali bug\g\ o mancanze a livello di performance e precisione.\\
	Di seguito vengono riportati i test di sistema definiti dal gruppo \leaf.
	
	\begin{longtabu} to \textwidth {X[0.7] X[2] X[1.3] X}
\toprule
\textbf{Test} & \textbf{Descrizione} & \textbf{Requisito} & \textbf{Stato}\\
\midrule
\endhead
\arrayrulecolor{gray}
TS1 & Viene verificato che il sistema calcoli un percorso per navigare da un POI A ad un POI B. & RObbF8.3 & Non \par implementato \\ 
\midrule 
TS1.1 & Viene verificato che il sistema calcoli un percorso per navigare da un POI A ad un POI B secondo le preferenze dell'utente. & RDesF8.3.1 & Non \par implementato \\ 
\midrule 
TS1.1.1 & Viene verificato che il sistema calcoli un percorso per navigare da un POI A ad un POI B scegliendo il percorso con meno barriere architettoniche. & RDesF8.3.1.1 & Non \par implementato \\ 
\midrule 
TS1.1.2 & Viene verificato che il sistema calcoli un percorso per navigare da un POI A ad un POI B scegliendo il percorso con meno ascensori. & RDesF8.3.1.2 & Non \par implementato \\ 
\midrule 
TS1.1.3 & Viene verificato che il sistema calcoli un percorso per navigare da un POI A ad un POI B scegliendo il percorso più veloce. & RDesF8.3.1.3 & Non \par implementato \\ 
\midrule 
TS1.2 & Viene verificato che il sistema fornisca le indicazioni per raggiungere il prossimo POI. & RDesF8.4.2.4 & Non \par implementato \\ 
\midrule 
TS1.3 & Viene verificato che il sistema fornisce una lista contenente le indicazioni utili per raggiungere la destinazione scelta percorrendo tutti i POI che compongono il percorso previsto. & RDesF8.4.2.2 & Non \par implementato \\ 
\midrule 
TS1.4 & Viene verificato che il sistema avvisi l'utente qualora rilevi un beacon differente da quelli previsti dal percorso calcolato. & RDesF8.4.2.3 & Non \par implementato \\ 
\midrule 
TS1.5 & Viene verificato che il sistema avvisi l'utente qualora si trovi in un'area in cui non viene rilevato alcun beacon. & RDesF8.4.2.6 & Non \par implementato \\ 
\midrule 
TS1.6 & Viene verificato che il sistema fornisca delle informazioni testuali estese. & ROpzF8.4.3.2 & Non \par implementato \\ 
\midrule 
TS1.7 & Viene verificato che il sistema fornisca le fotografie del prossimo POI da raggiungere. & RDesF8.4.3.1 & Non \par implementato \\ 
\midrule 
TS1.8 & Viene verificato che il sistema fornisca la lista di tutte le prossime indicazioni da seguire per raggiungere la destinazione scelta. & ROpzF8.4.3.3 & Non \par implementato \\ 
\midrule 
TS1.9 & Viene verificato che il sistema permetta di interrompere la navigazione in corso. & RObbF8.5 & Non \par implementato \\ 
\midrule 
TS1.9.1 & Viene verificato che il sistema richieda l'attivazione della geolocalizzazione. & RObbF8.4.1.1 & Non \par implementato \\ 
\midrule 
TS1.9.2 & Viene verificato che il sistema richieda l'attivazione del Bluetooth. & RObbF8.4.1.2 & Non \par implementato \\ 
\midrule 
TS1.9.3 & Viene verificato che il sistema richieda l'attivazione del GPS se il dispositivo ha una versione del sistema operativo uguale o superiore a 6.0. & RObbF8.4.1.3 & Non \par implementato \\ 
\midrule 
TS1.10 & Viene verificato che il sistema avverta l'utente qualora volesse avviare la navigazione in mancanza di una connessione internet attiva. & RObbF8.6 & Non \par implementato \\ 
\midrule 
TS1.11 & Viene verificato che il sistema avverta l'utente qualora volesse avviare la navigazione e la mappa installata sul suo dispositivo differisce dall'ultima versione disponibile per quell'edificio. & RObbF8.7 & Non \par implementato \\ 
\midrule 
TS1.12 & Viene verificato che il sistema avverta l'utente qualora rilevasse un beacon all'interno di un edificio e la mappa dell’edificio non fosse installata nel dispositivo. & RObbF8.8 & Non \par implementato \\ 
\midrule 
TS1.13 & Viene verificato che il sistema fornisca la possibilità di ricercare una destinazione per nome. & RDesF8.1.1 & Non \par implementato \\ 
\midrule 
TS1.13.1 & Viene verificato che il sistema fornisca la possibilità di inserire il nome di una destinazione. & RDesF8.1.1.1 & Non \par implementato \\ 
\midrule 
TS1.14 & Viene verificato che il sistema fornisca la possibilità di ricercare una destinazione per categoria. & RObbF8.1.2 & Non \par implementato \\ 
\midrule 
TS1.14.1 & Viene verificato che il sistema permetta di accedere ad una categoria tra quelle disponibili per il dato edificio, accedendo ai POI in essa contenuti. & RObbF8.1.2.1 & Non \par implementato \\ 
\midrule 
TS1.15 & Viene verificato che il sistema permetta di selezionare il risultato di una ricerca. & RObbF8.1.3 & Non \par implementato \\ 
\midrule 
TS1.16 & Viene verificato che il sistema permetta di confermare la scelta di una destinazione. & RObbF8.2 & Non \par implementato \\ 
\midrule 
TS1.17 & Viene verificato che il sistema avverta l'utente qualora volesse accedere alla foto del prossimo POI e la connessione Internet non fosse attiva sul proprio dispositivo. & RDesF8.4.3.4 & Non \par implementato \\ 
\midrule 
TS2 & Viene verificato che il sistema richieda l'attivazione dei sensori. & RObbF8.4.1 & Non \par implementato \\ 
\midrule 
TS3 & Viene verificato che il sistema interagisca con i beacon. & RObbF9 & Non \par implementato \\ 
\midrule 
TS3.1 & Viene verificato che il sistema rilevi gli identificativi (UUID, Major, Minor) di un beacon rilevato dall'applicazione. & RObbF9.1 & Non \par implementato \\ 
\midrule 
TS3.1.1 & Viene verificato che, rilevato l'identificativo di un beacon, il sistema riesca a reperire informazioni riguardanti il POI a cui è associato quel beacon. & RObbF9.1.1 & Non \par implementato \\ 
\midrule 
TS3.1.2 & Viene verificato che, rilevato l'identificativo di un beacon, il sistema riesca a reperire informazioni riguardanti i POI circostanti quel beacon. & RObbF9.1.2 & Non \par implementato \\ 
\midrule 
TS3.2 & Viene verificato che il sistema rilevi il livello di potenza del segnale di un beacon rilevato. & RObbF9.2 & Non \par implementato \\ 
\midrule 
TS3.3 & Viene verificato che il sistema rilevi il livello di batteria di un beacon rilevato. & RObbF9.3 & Non \par implementato \\ 
\midrule 
TS3.4 & Viene verificato che il sistema rilevi la distanza approssimativa di un beacon rilevato dal dispositivo utilizzato. & RObbF9.4 & Non \par implementato \\ 
\midrule 
TS3.5 & Viene verificato che il sistema rilevi il formato di un beacon rilevato. & RObbF9.5 & Non \par implementato \\ 
\midrule 
TS3.6 & Viene verificato che il sistema rilevi l'area coperta dal segnale di un beacon rilevato. & RObbF9.6 & Non \par implementato \\ 
\midrule 
TS4 & Viene verificato che il sistema permette di recuperare una mappa collegandosi ad un server. & RDesF11.2.3 & Non \par implementato \\ 
\midrule 
TS5 & Viene verificato che il sistema permetta di accedere al nome dell'edificio in cui si trova l'utente. & RObbF10.5 & Non \par implementato \\ 
\midrule 
TS6 & Viene verificato che il sistema permetta di accedere alla descrizione dell'edificio in cui si trova l'utente. & RObbF10.6 & Non \par implementato \\ 
\midrule 
TS7 & Viene verificato che il sistema permetta di accedere all'orario dell'edificio in cui si trova l'utente. & RObbF10.3 & Non \par implementato \\ 
\midrule 
TS8 & Viene verificato che il sistema permetta di accedere all'indirizzo dell'edificio in cui si trova l'utente. & RObbF10.4 & Non \par implementato \\ 
\midrule 
TS9 & Viene verificato che il sistema permetta di accedere alla lista di POI di un edificio. & RObbF10.1 & Non \par implementato \\ 
\midrule 
TS9.1 & Viene verificato che il sistema permetta di accedere alle informazioni su tutti i luoghi interni all'edificio in cui si trova l'utente. & RDesF10.2 & Non \par implementato \\ 
\midrule 
TS10 & Viene verificato che il sistema permetta di accedere alle informazioni relative ad uno specifico POI. & RDesF10.2.1 & Non \par implementato \\ 
\midrule 
TS10.1 & Viene verificato che il sistema permetta di accedere al nome di un POI. & RObbF10.2.3 & Non \par implementato \\ 
\midrule 
TS10.2 & Viene verificato che il sistema permetta di accedere alla descrizione di un POI. & RObbF10.2.4 & Non \par implementato \\ 
\midrule 
TS11 & Viene verificato che il sistema permetta di accedere ad un elenco dei POI appartenenti all’edificio in cui si trova l’utente e rilevati alla posizione dell’utente. & RDesF10.2.2 & Non \par implementato \\ 
\midrule 
TS12 & Viene verificato che il sistema avverta l'utente qualora volesse accedere alle informazioni dell'edificio in cui si trova e la connessione Internet non fosse attiva sul proprio dispositivo. & RObbF10.9 & Non \par implementato \\ 
\midrule 
TS13 & Viene verificato che il sistema avverta l'utente qualora volesse accedere alle informazioni dell'edificio in cui si trova e la versione della mappa presente sul dispositivo non coincidesse con l'ultima versione della mappa disponibile. & RObbF10.10 & Non \par implementato \\ 
\midrule 
TS14 & Viene verificato che il sistema permetta di impostare le preferenze di navigazione. & RDesF11.1 & Non \par implementato \\ 
\midrule 
TS14.1 & Viene verificato che il sistema permetta di fornire le indicazioni in forma testuale. & RObbF11.1.2.1 & Non \par implementato \\ 
\midrule 
TS14.2 & Viene verificato che il sistema permetta di attivare le indicazioni sonore. & ROpzF11.1.2.3 & Non \par implementato \\ 
\midrule 
TS14.3 & Viene verificato che il sistema permetta di attivare le indicazioni vocali. & ROpzF11.1.2.2 & Non \par implementato \\ 
\midrule 
TS14.4 & Viene verificato che il sistema permetta di disattivare le indicazioni sonore. & ROpzF11.1.2.5 & Non \par implementato \\ 
\midrule 
TS14.5 & Viene verificato che il sistema permetta di disattivare le indicazioni vocali. & ROpzF11.1.2.4 & Non \par implementato \\ 
\midrule 
TS14.6 & Viene verificato che il sistema permetta di scegliere il percorso più accessibile. & RDesF11.1.1.1 & Non \par implementato \\ 
\midrule 
TS14.7 & Viene verificato che il sistema permetta di scegliere il percorso con il minor numero di ascensori. & RDesF11.1.1.2 & Non \par implementato \\ 
\midrule 
TS14.8 & Viene verificato che il sistema permetta di scegliere il percorso più veloce. & RObbF11.1.1.3 & Non \par implementato \\ 
\midrule 
TS15 & Viene verificato che il sistema permetta la gestione delle mappe. & RDesF11.2 & Non \par implementato \\ 
\midrule 
TS15.1 & Viene verificato che il sistema permetta di accedere alle mappe installate nel proprio dispositivo. & RDesF11.2.1.1 & Non \par implementato \\ 
\midrule 
TS15.2 & Viene verificato che il sistema permetta di installare una mappa disponibile online non precedentemente installata. & RDesF11.2.2.2 & Non \par implementato \\ 
\midrule 
TS15.3 & Viene verificato che il sistema permetta di ricercare una mappa. & ROpzF11.2.2.1 & Non \par implementato \\ 
\midrule 
TS15.4 & Viene verificato che il sistema permetta di rimuovere una mappa. & RDesF11.2.1.3 & Non \par implementato \\ 
\midrule 
TS15.5 & Viene verificato che il sistema permetta di aggiornare una mappa. & RDesF11.2.1.2 & Non \par implementato \\ 
\midrule 
TS15.6 & Viene verificato che il sistema permetta di accedere al nome di una mappa presente sul dispositivo. & RDesF11.2.1.4.1 & Non \par implementato \\ 
\midrule 
TS15.7 & Viene verificato che il sistema permetta di accedere all'indirizzo dell'edificio a cui si riferisce una mappa presente sul proprio dispositivo. & RDesF11.2.1.4.2 & Non \par implementato \\ 
\midrule 
TS15.8 & Viene verificato che il sistema permetta di accedere alla descrizione dell'edificio a cui si riferisce una mappa presente sul proprio dispositivo. & RDesF11.2.1.4.3 & Non \par implementato \\ 
\midrule 
TS15.9 & Viene verificato che il sistema permetta di accedere alla dimensione in megabyte di una mappa presente sul proprio dispositivo. & RDesF11.2.1.4.4 & Non \par implementato \\ 
\midrule 
TS15.10 & Viene verificato che il sistema permetta di accedere alla versione di una mappa presente sul proprio dispositivo. & RDesF11.2.1.4.5 & Non \par implementato \\ 
\midrule 
TS15.11 & Viene verificato che il sistema permetta di accedere al nome di una mappa non presente sul dispositivo. & RDesF11.2.2.3.1 & Non \par implementato \\ 
\midrule 
TS15.12 & Viene verificato che il sistema permetta di accedere all'indirizzo dell'edificio a cui si riferisce una mappa non presente sul proprio dispositivo. & RDesF11.2.2.3.2 & Non \par implementato \\ 
\midrule 
TS15.13 & Viene verificato che il sistema permetta di accedere alla descrizione dell'edificio a cui si riferisce una mappa non presente sul proprio dispositivo. & RDesF11.2.2.3.3 & Non \par implementato \\ 
\midrule 
TS15.14 & Viene verificato che il sistema permetta di accedere alla dimensione in megabyte di una mappa non presente sul proprio dispositivo. & RDesF11.2.2.3.4 & Non \par implementato \\ 
\midrule 
TS15.15 & Viene verificato che il sistema permetta di accedere alla versione di una mappa non presente sul proprio dispositivo. & RDesF11.2.2.3.5 & Non \par implementato \\ 
\midrule 
TS15.16 & Viene verificato che il sistema segnali all'utente qualora la ricerca per nome non abbia trovato corrispondenza tra le mappe disponibili online.. & RDesF11.2.2.4 & Non \par implementato \\ 
\midrule 
TS16 & Viene verificato che il sistema permetta di inserire il codice sviluppatore. & RObbF11.3.1 & Non \par implementato \\ 
\midrule 
TS16.1 & Viene verificato che il sistema permetta di confermare il codice sviluppatore. & RObbF11.3.2 & Non \par implementato \\ 
\midrule 
TS17 & Viene verificato che il sistema metta a disposizione una sezione per la guida. & RDesF12 & Non \par implementato \\ 
\midrule 
TS18 & Viene verificato che il sistema permetta di accedere alle informazioni di un beacon rilevato. & RObbF13 & Non \par implementato \\ 
\midrule 
TS18.1 & Viene verificato che il sistema permetta di accedere al UUID di un beacon rilevato. & RObbF13.1 & Non \par implementato \\ 
\midrule 
TS18.2 & Viene verificato che il sistema permetta di accedere al Major di un beacon rilevato. & RObbF13.8 & Non \par implementato \\ 
\midrule 
TS18.3 & Viene verificato che il sistema permetta di accedere al Minor di un beacon rilevato. & RObbF13.9 & Non \par implementato \\ 
\midrule 
TS18.4 & Viene verificato che il sistema permetta di accedere al livello di potenza del segnale di un beacon rilevato. & RObbF13.2 & Non \par implementato \\ 
\midrule 
TS18.5 & Viene verificato che il sistema permetta di accedere al livello di batteria di un beacon rilevato. & RDesF13.4 & Non \par implementato \\ 
\midrule 
TS18.6 & Viene verificato che il sistema permetta di accedere alla distanza approssimativa di un beacon rilevato dal dispositivo utilizzato. & RObbF13.5 & Non \par implementato \\ 
\midrule 
TS18.7 & Viene verificato che il sistema permetta di accedere al formato di un beacon rilevato. & RObbF13.6 & Non \par implementato \\ 
\midrule 
TS18.8 & Viene verificato che il sistema permetta di accedere all'area coperta da un beacon rilevato. & RObbF13.7 & Non \par implementato \\ 
\midrule 
TS18.9 & Viene verificato che il sistema permetta di gestire un log. & RObbF13.3 & Non \par implementato \\ 
\midrule 
TS18.9.1 & Viene verificato che il sistema permetta di avviare un log. & RDesF13.3.2 & Non \par implementato \\ 
\midrule 
TS18.9.2 & Viene verificato che il sistema permetta di interrompere un log. & RDesF13.3.1 & Non \par implementato \\ 
\midrule 
TS18.9.3 & Viene verificato che il sistema permetta di salvare un log. & RDesF13.3.3 & Non \par implementato \\ 
\midrule 
TS18.9.4 & Viene verificato che il sistema permetta di rimuovere un log. & RDesF13.3.5 & Non \par implementato \\ 
\midrule 
TS18.9.5 & Viene verificato che il sistema permetta di accedere ad un log salvato. & RDesF13.3.4 & Non \par implementato \\ 
\midrule 
TS19 & Viene verificato che il sistema avverta l'utente qualora venga inserita una destinazione non prevista dal sistema. & RObbF8.1.4 & Non \par implementato \\ 
\midrule 
TS20 & Viene verificato che il sistema avverta l'utente qualora il codice inserito per sbloccare le funzionalità sviluppatore non sia corretto. & RObbF11.3.3 & Non \par implementato \\ 
\midrule 
TS21 & Viene verificato che il sistema fornisca la possibilità di inserire il possibile nome di una mappa. & ROpzF11.2.2.1.1 & Non \par implementato \\ 
\midrule 
TS22 & Viene verificato che il prototipo operi all'interno dell'area indoor scelta. & RObbF3 & Non \par implementato \\ 
\midrule 
TS22.1 & Viene verificato che il prototipo dia un'indicazione approssimativa di un utente all'interno dell'edificio. & RObbF3.1 & Non \par implementato \\ 
\midrule 
TS22.2 & Viene verificato che il prototipo permetta di fornire informazioni all'utente relative all'area mappata dal beacon (smart places). & RObbF3.2 & Non \par implementato \\ 
\arrayrulecolor{black}
\bottomrule
\caption{Tabella di tracciamento test di sistema / requisiti} \\

\end{longtabu}	
	\subsection{Test di integrazione}
		I test di integrazione servono per verificare il corretto funzionamento di più moduli assemblati insieme. Per una descrizione completa della sintassi utilizzata nella descrizione di tali test si consulti il documento \normediprogettov.
		\subsubsection{Test-componenti}
		\begin{longtabu}to \textwidth{X[0.5] X[2] X[2] X}
\toprule
\textbf{Test} & \textbf{Descrizione} & \textbf{Componenti} & \textbf{Stato}\\
\midrule
\endhead
\arrayrulecolor{gray}
TI1 & Test di integrazione finale tra tutte le componenti per verificare il corretto comportamento del sistema nel suo complesso &  & N.I. \\ 
\midrule 
TI2 & Viene verificato che il sistema gestisca correttamente le componenti relative al package model. In particolare che la navigazione si comporti secondo le esigenze dell'utente, utilizzi correttamente i sensori, si interfacci correttamente con la tecnologia Beacon e vengano recuperate le informazioni dal database & model & N.I. \\ 
\midrule 
TI3 & Viene verificato che il sistema gestisca correttamente le componenti relative al package beacon. In particolare che si interfacci correttamente con la libreria AltBeacon e con la tecnologia Beacon & model::\-beacon & N.I. \\ 
\midrule 
TI4 & Viene verificato che il sistema gestisca correttamente le componenti relative al package usersetting. In particolare che gestisca correttamente le preferenze dell'utente e le renda persistenti sul dispositivo & model::\-usersetting & N.I. \\ 
\midrule 
TI5 & Viene verificato che il sistema gestisca correttamente le componenti relative al package compass. In particolare che sia possibile avviare e fermare la bussola & model::\-compass & N.I. \\ 
\midrule 
TI6 & Viene verificato che il sistema gestisca correttamente le componenti relative al package navigator. In particolare che fornisca le funzionalità di navigazione e calcoli il percorso secondo le esigenze dell'utente & model::\-navigator & N.I. \\ 
\midrule 
TI7 & Viene verificato che il sistema gestisca correttamente le componenti relative al package algorithm. In particolare che, dati un grafo pesato e due nodi, calcoli un percorso dal nodo di partenza al nodo di arrivo & model::\-navigator::\-algorithm & N.I. \\ 
\midrule 
TI8 & Viene verificato che il sistema gestisca correttamente le componenti relative al package graph. In particolare che gestisca la struttura di un grafo e associ delle informazioni ad un PointOfInterest o ad un Edge & model::\-navigator::\-graph & N.I. \\ 
\midrule 
TI9 & Viene verificato che il sistema gestisca correttamente le componenti relative al package vertex. In particolare che rappresenti un nodo di un grafo. & model::\-navigator::\-graph::\-vertex & N.I. \\ 
\midrule 
TI10 & Viene verificato che il sistema gestisca correttamente le componenti relative al package area. In particolare che rappresentino la struttura di un edificio come PointOfInterest e RegionOfInterest & model::\-navigator::\-graph::\-area & N.I. \\ 
\midrule 
TI11 & Viene verificato che il sistema gestisca correttamente le componenti relative al package navigationInformation. In particolare rappresenti i diversi tipi di istruzione a seconda delle preferenze dell'utente & model::\-navigator::\-graph::\-navigationinformation & N.I. \\ 
\midrule 
TI12 & Viene verificato che il sistema gestisca correttamente le componenti relative al package edge. In particolare che rappresenti i diversi tipi di archi e le informazioni ad essi associate & model::\-navigator::\-graph::\-edge & N.I. \\ 
\midrule 
TI13 & Viene verificato che il sistema gestisca correttamente le componenti relative al package dataaccess. In particolare che si interfacci in maniera corretta col database remoto e permetta di gestire le informazioni nel database locale & model::\-dataaccess & N.I. \\ 
\midrule 
TI14 & Viene verificato che il sistema gestisca correttamente le componenti relative al package service. In particolare che faccia da tramite per l'accesso al database per il resto del model e costruisca oggetti della Business Logic a partire dagli oggetti che rappresentano le tabelle del database & model::\-dataaccess::\-service & N.I. \\ 
\midrule 
TI15 & Viene verificato che il sistema gestisca correttamente le componenti relative al package dao. In particolare che sia possibile effettuare le operazioni CRUD sul database locale e tradurre la struttura del database in oggetti, che rappresentano le tabelle, per accedere alle informazioni & model::\-dataaccess::\-dao & N.I. \\ 
\midrule 
TI16 & Viene verificato che il sistema gestisca correttamente le componenti relative al package view. In particolare che riesca a recuperare le informazioni dal package presenter e che le esponga correttamente all'utente & view & N.I. \\ 
\midrule 
TI17 & Viene verificato che il sistema gestisca correttamente le componenti relative al package presenter. In particolare che gestisca correttamente l'interazione coi componenti del package model e del package view & presenter & N.I. \\ 
\midrule 
TI18 & Viene verificato che il sistema gestisca correttamente le componenti relative al package di. In particolare che permetta al presenter di risolvere le dipendenze verso i componenti del package model & di & N.I. \\ 
\midrule 
TI19 & Viene verificato che il sistema gestisca correttamente le componenti relative al package di::component. In particolare che fornisca le interfacce per permettere di eseguire la dependecy injection & di::component & N.I. \\ 
\midrule 
TI20 & Viene verificato che il sistema gestisca correttamente le componenti relative al package di::module. In particolare che risolva le dipendenze tra gli oggetti presenti nell'applicazionee definisca la cardinalità delle istanze di un oggetto & di::module & N.I. \\ 
\arrayrulecolor{black}
\bottomrule
\caption{Tabella test di integrazione} \\
\end{longtabu}

\subsubsection{Componente - test}

\begin{longtabu}to \textwidth{X[3] X[0.5]}
\toprule
\textbf{Componente} & \textbf{Test}\\
\midrule
\endhead
\arrayrulecolor{gray}
di & T18 \\ 
\midrule 
di::\-component & T19 \\ 
\midrule 
di::\-module & T20 \\ 
\midrule 
model & TI2 \\ 
\midrule 
model::\-beacon & TI3 \\ 
\midrule 
model::\-compass & TI5 \\ 
\midrule 
model::\-dataaccess & TI13 \\ 
\midrule 
model::\-dataaccess::\-dao & TI15 \\ 
\midrule 
model::\-dataaccess::\-service & TI14 \\ 
\midrule 
model::\-navigator & TI6 \\ 
\midrule 
model::\-navigator::\-algorithm & TI7 \\ 
\midrule 
model::\-navigator::\-graph & TI8 \\ 
\midrule 
model::\-navigator::\-graph::\-area & TI10 \\ 
\midrule 
model::\-navigator::\-graph::\-edge & TI12 \\ 
\midrule 
model::\-navigator::\-graph::\-navigationinformation & TI11 \\ 
\midrule 
model::\-navigator::\-graph::\-vertex & TI9 \\ 
\midrule 
model::\-usersetting & TI4 \\ 
\midrule 
presenter & TI17 \\ 
\midrule 
view & TI16 \\ 
\arrayrulecolor{black}
\bottomrule
\caption{Tabella componente / test di integrazione} \\
\end{longtabu}

\subsection{Test di unità}
Il test di unità serve per accertare il corretto funzionamento delle singole unità, ovvero le classi. Per una descrizione completa della sintassi utilizzata nella descrizione di tali test si consulti il documento \normediprogettov.
	
\begin{longtabu}to \textwidth {X[0.6] X[2] X[3.2] X[0.8]}
\toprule
\textbf{Test} & \textbf{Descrizione} & \textbf{Metodi} & \textbf{Stato}\\
\midrule
\endhead
\arrayrulecolor{gray}
TU1 & Viene testato che tramite un oggetto SettingImp sia possibile salvare e recuperare le informazioni relative alle preferenze dell'utente & model::\-usersetting::\-SettingImp::\-getPathPreference() \par model::\-usersetting::\-SettingImp::\-getInstructionPreference() \par model::\-usersetting::\-SettingImp::\-setPathPreference() \par model::\-usersetting::\-SettingImp::\-setInstructionPreference() & Superato \\ 
\midrule 
TU2 & Viene testato che tramite la classe DeveloperCodeManager sia possibile riconoscere un codice sviluppatore valida da uno non valido & model::\-usersetting::\-DeveloperCodeManager::\-isValid() & Superato \\ 
\midrule 
TU3 & Viene testato che tramite un oggetto della classe SettimgImp sia possibile recuperare e modificare le informazioni riguardanti un eventuale codice sviluppatore inserito. In particolare viene testato se vengono salvate le informazioni relative al fatto che un utente sia o meno sviluppatore & model::\-usersetting::\-SettingImp::\-isDeveloper() \par model::\-usersetting::\-SettingImp::\-unlockDeveloper() & Superato \\ 
\midrule 
TU4 & Viene testato che sia possibile, utilizzando un oggetto BuildingInformation, accedere alle informazioni relative ad un edificio & model::\-navigator::\-BuildingInformation::\-getName() \par model::\-navigator::\-BuildingInformation::\-getDescription() \par model::\-navigator::\-BuildingInformation::\-getOpeningHours() \par model::\-navigator::\-BuildingInformation::\-getAddress() \par model::\-navigator::\-BuildingInformation::\-toString() & Superato \\ 
\midrule 
TU5 & Viene testato che utilizzando un oggetto BuildingMapImp sia possibile accedere alle informazioni dell'edificio, alla versione della mappa della mappa e al suo id all'interno del database, e alle collezioni di PointOfInterest, RegionOfInterest e EnrichedEdge che contiene & model::\-navigator::\-BuildingMapImp::\-getAddress() \par model::\-navigator::\-BuildingMapImp::\-getAllBuildingInformation() \par model::\-navigator::\-BuildingMapImp::\-getAllEdges() \par model::\-navigator::\-BuildingMapImp::\-getAllPOIs() \par model::\-navigator::\-BuildingMapImp::\-getAllROIs() \par model::\-navigator::\-BuildingMapImp::\-getDescription() \par model::\-navigator::\-BuildingMapImp::\-getId() \par model::\-navigator::\-BuildingMapImp::\-getName() \par model::\-navigator::\-BuildingMapImp::\-getOpeningHours() \par model::\-navigator::\-BuildingMapImp::\-getVersion() \par model::\-navigator::\-BuildingMapImp::\-getSize() & Superato \\ 
\midrule 
TU6 & Viene testato che utilizzando un oggetto BuildingMapImp sia possibile accedere alla collezione di PointOfInterest associati ad alla RegionOfInterest che contiene il beacon passato & model::\-navigator::\-BuildingMapImp::\-getNearbyPOIs() & Superato \\ 
\midrule 
TU7 & Viene testato che sia possibile accedere a tutte le informazioni contenute in un oggetto ProcessedInformationImp & model::\-navigator::\-ProcessedInformationImp::\-getDetailedInstruction() \par model::\-navigator::\-ProcessedInformationImp::\-getPhotoInstruction() \par model::\-navigator::\-ProcessedInformationImp::\-getProcessedBasicInstruction() \par model::\-navigator::\-ProcessedInformationImp::\-getDirection() & Superato \\ 
\midrule 
TU8 & Viene testato che sia possibile accedere alle informazioni contenute in un oggetto VertexImp & model::\-navigator::\-graph::\-vertex::\-VertexImp::\-getId() & Superato \\ 
\midrule 
TU9 & Viene testato che sia possibile accedere alle informazioni contenute in un oggetto BasicInformation & model::\-navigator::\-graph::\-navigationinformation::\-BasicInformation::\-getBasicInformation() & Superato \\ 
\midrule 
TU10 & Viene testato che sia possibile accedere alle informazioni contenute in un oggetto DetailedInformation & model::\-navigator::\-graph::\-navigationinformation::\-DetailedInformation::\-getDetailedInformation() & Superato \\ 
\midrule 
TU11 & Viene testato che sia possible accedere alle informazioni contenute in un oggetto PhotoRef & model::\-navigator::\-graph::\-navigationinformation::\-PhotoRef::\-getPhotoUri() \par model::\-navigator::\-graph::\-navigationinformation::\-PhotoRef::\-getId() & Superato \\ 
\midrule 
TU12 & Viene testato che sia possibile accedere alla collezione di oggetti PhotoRef contenuta in un oggetto PhotoInformation & model::\-navigator::\-graph::\-navigationinformation::\-PhotoInformation::\-getPhotoInformation() & Superato \\ 
\midrule 
TU13 & Viene testato che sia possibile accedere alle informazioni contenute in un oggetto NavigationInformationImp & model::\-navigator::\-graph::\-navigationinformation::\-NavigationInformation::\-getBasicInformation() \par model::\-navigator::\-graph::\-navigationinformation::\-NavigationInformation::\-getDetailedInformation() \par model::\-navigator::\-graph::\-navigationinformation::\-NavigationInformation::\-getPhotoInformation() & Superato \\ 
\midrule 
TU14 & Viene testato che sia possibile accedere alle informazioni relative punto di inizio, punto di fine, distanza tra i due punti, l'angolo, rispetto al nord, che c'è tra il primo e il secondo punto, collezione di PhotoRef, id di un oggetto che ha tipo statico AbsEnrichedEdge e tipo dinamico un sottotipo di AbsEnrichedEdge & model::\-navigator::\-graph::\-edge::\-AbsEnrichedEdge::\-getStarterPoint() \par model::\-navigator::\-graph::\-edge::\-AbsEnrichedEdge::\-getEndPoint() \par model::\-navigator::\-graph::\-edge::\-AbsEnrichedEdge::\-getPhotoInformation() \par model::\-navigator::\-graph::\-edge::\-AbsEnrichedEdge::\-getCoordinate() \par model::\-navigator::\-graph::\-edge::\-AbsEnrichedEdge::\-getId() & Superato \\ 
\midrule 
TU15 & Viene testato che sia possibile impostare le preferenze relative agli archi da attraversare & model::\-navigator::\-graph::\-edge::\-AbsEnrichedEdge::\-setUserPreference() & Superato \\ 
\midrule 
TU16 & Viene testato che sia possibile accedere alle informazioni di navigazione all'interno di una sottoclasse di AbsEnrichedEdge & model::\-navigator::\-graph::\-edge::\-AbsEnrichedEdge::\-getNavigationInformation() & Superato \\ 
\midrule 
TU17 & Viene testato che sia possibile accedere alle informazioni di base per superare tale arco e quelle dettagliate in un oggetto DefaultEdge & model::\-navigator::\-graph::\-edge::\-DefaultEdge::\-getBasicInformation() \par model::\-navigator::\-graph::\-edge::\-DefaultEdge::\-getDetailedInformation() & Superato \\ 
\midrule 
TU18 & Viene testato che sia possibile accedere alle informazioni di base per superare tale arco e quelle dettagliate in un oggetto StairEdge & model::\-navigator::\-graph::\-edge::\-StairEdge::\-getBasicInformation() \par model::\-navigator::\-graph::\-edge::\-StairEdge::\-getDetailedInformation() & Superato \\ 
\midrule 
TU19 & Viene testato che sia possibile accedere alle informazioni di base per superare tale arco e quelle dettagliate in un oggetto ElevatorEdge & model::\-navigator::\-graph::\-edge::\-ElevatorEdge::\-getBasicInformation() \par model::\-navigator::\-graph::\-edge::\-ElevatorEdge::\-getDetailedInformation() & Superato \\ 
\midrule 
TU20 & Viene testato che il peso dell'arco venga calcolato in base alle preferenze impostate tramite il metodo AbsEnrichedEdge.setUserPreference in un oggetto StairEdge & model::\-navigator::\-graph::\-edge::\-StairEdge::\-getWeight() & Superato \\ 
\midrule 
TU21 & Viene testato che il peso dell'arco venga calcolato in base alle preferenze impostate tramite il metodo AbsEnrichedEdge.setUserPreference in un oggetto ElevatorEdge & model::\-navigator::\-graph::\-edge::\-ElevatorEdge::\-getWeight() & Superato \\ 
\midrule 
TU22 & Viene testato che il peso dell'arco venga calcolato in base alle preferenze impostate tramite il metodo AbsEnrichedEdge.setUserPreference in un oggetto DefaultEdge & model::\-navigator::\-graph::\-edge::\-DefaultEdge::\-getWeight() & Superato \\ 
\midrule 
TU23 & Viene testato che sia possibile accedere a tutte le informazioni contenute in un oggetto PointOfInterestInformation & model::\-navigator::\-graph::\-area::\-PointOfInterestInformation::\-getName() \par model::\-navigator::\-graph::\-area::\-PointOfInterestInformation::\-getDescription() \par model::\-navigator::\-graph::\-area::\-PointOfInterestInformation::\-getCategory() & Superato \\ 
\midrule 
TU24 & Viene testato che sia possibile accedere a tutte le informazioni riguardanti il POI e all'id del POI relativo al database in un oggetto PointOfInterestImp & model::\-navigator::\-graph::\-area::\-PointOfInterestImp::\-getName() \par model::\-navigator::\-graph::\-area::\-PointOfInterestImp::\-getDescription() \par model::\-navigator::\-graph::\-area::\-PointOfInterestImp::\-getCategory() \par model::\-navigator::\-graph::\-area::\-PointOfInterestImp::\-getId() & Superato \\ 
\midrule 
TU25 & Viene testato che sia possibile settare e accedere a tutti i ROI in cui è contenuto il POI rappresentato da un oggetto PointOfInterest & model::\-navigator::\-graph::\-area::\-PointOfInterestImp::\-getAllBelongingROIs() \par model::\-navigator::\-graph::\-area::\-PointOfInterestImp::\-setBelongingROIs() & Superato \\ 
\midrule 
TU26 & Viene testato che sia possibile accedere alle informazioni alle informazioni relative al beacon che è contenuto in una determinata ROI tramite un oggetto RegionOfInterestImp & model::\-navigator::\-graph::\-area::\-RegionOfInterestImp::\-getUUID() \par model::\-navigator::\-graph::\-area::\-RegionOfInterestImp::\-getMajor() \par model::\-navigator::\-graph::\-area::\-RegionOfInterestImp::\-getMinor() & Superato \\ 
\midrule 
TU27 & Viene testato che sia possibile ricavare il piano di appartenenza di un oggetto RegionOfInterestImp ricavandolo dal minor & model::\-navigator::\-graph::\-area::\-RegionOfInterestImp::\-getFloor() & Superato \\ 
\midrule 
TU28 & Viene testato che sia possibile verificare tramite la classe RegionOfInterestImp è possibile verificare se un beacon è contenuto o meno in una certa ROI & model::\-navigator::\-graph::\-area::\-RegionOfInterestImp::\-contains() & Superato \\ 
\midrule 
TU29 & Viene testato che sia possibile settare e accedere a tutti i POI contenuti nel ROI rappresentato da un oggetto RegionOfInterest & model::\-navigator::\-graph::\-area::\-RegionOfInterestImp::\-getAllNearbyPOIs() \par model::\-navigator::\-graph::\-area::\-RegionOfInterestImp::\-setNearbyPOIs() & Superato \\ 
\midrule 
TU30 & Viene testato che sia possibile aggiungere EnrichedEdge e RegionOfInterest ad un oggetto MapGraph & model::\-navigator::\-graph::\-MapGraph::\-addAllRegions() \par model::\-navigator::\-graph::\-MapGraph::\-addEdge() \par model::\-navigator::\-graph::\-MapGraph::\-addAllEdges() & Superato \\ 
\midrule 
TU31 & Viene testato che un oggetto MapGraph possa ritornare un grafo & model::\-navigator::\-graph::\-MapGraph::\-getGraph() & Superato \\ 
\midrule 
TU32 & Viene testato che sia possibile calcolare un persorso formato da una lista di Edges utilizzando un oggetto DjikstraPathFinder & model::\-navigator::\-algorithm::\-DijkstraPathFinder::\-calculatePath() & Superato \\ 
\midrule 
TU33 & Viene testato che sia possibile settare ad un oggetto Navigator il grafo su cui si vuole effettuare la navigazione e calcolare un percorso da un certo punto ad un altro. In particolare deve essere testato che venga lanciata l'eccezione NoGraphSetException nel caso in cui venga richiesto di calcolare un percorso e non sia stato settato alcun grafo, mentre deve essere lanciata l'eccezione NoNavigationInformationException nel caso in cui si richieda un percorso e quest'ultimo non è ancora stato calcolato & model::\-navigator::\-NavigatorImp::\-calculatePath() \par model::\-navigator::\-NavigatorImp::\-setGraph() \par model::\-navigator::\-NavigatorImp::\-getPath() & Superato \\ 
\midrule 
TU34 & Viene testato che sia possibile, settato un grafo e calcolato un percorso, ottenere tutte le istruzioni di navigazione. In particolare deve essere lanciata un'eccezione di tipo NoNavigationInformationException nel caso in cui si richiedano le informazioni riguardanti un percorso ma queste non siano disponibili poichè non è stato settato un grafo o non è ancora stato calcolato un percorso & model::\-navigator::\-NavigatorImp::\-getAllInstructions() & Superato \\ 
\midrule 
TU35 & Viene testato che sia possibile, settato un grafo e calcolato un percorso, ottenere le informazioni di navifìgazione una di seguito all'altra. In particolare deve essere lanciata un'eccezione di tipo NoNavigationInformationException nel caso in cui si richiedano le informazioni riguardanti un percorso ma queste non siano disponibili poichè non è stato settato un grafo o non è ancora stato calcolato un percorso. Inoltre viene lanciata un'eccezione PathException nel caso in cui il beacon più potente rilevato non faccia parte del percorso previsto & model::\-navigator::\-NavigatorImp::\-toNextRegion() & Superato \\ 
\midrule 
TU36 & Viene testato che sia possibile accedere a tutte le informazioni relative ad un oggetto MyBeacon & model::\-beacon::\-MyBeacon::\-getUUID() \par model::\-beacon::\-MyBeacon::\-getMajor() \par model::\-beacon::\-MyBeacon::\-getBluetoothAddress() \par model::\-beacon::\-MyBeacon::\-getDistance() \par model::\-beacon::\-MyBeacon::\-getTxPower() \par model::\-beacon::\-MyBeacon::\-getRssi() \par model::\-beacon::\-MyBeacon::\-getMinor() \par model::\-beacon::\-MyBeacon::\-getBatteryLevel() \par model::\-beacon::\-MyBeacon::\-getBeaconTypeCode() & Superato \\ 
\midrule 
TU37 & Viene testato che sia possibile aggiungere ad un oggetto Log le informazioni di un beacon & model::\-beacon::\-Logger::\-add() & Non superato \\ 
\midrule 
TU38 & Viene testato che sia possibile salvare un oggetto Log & model::\-beacon::\-Logger::\-save() & Non superato \\ 
\midrule 
TU39 & Viene testato che sia possibile mettere un oggetto BeaconManagerAdapter in background mode & model::\-beacon::\-BeaconManagerAdapter::\-setBackgroundMode() & Superato \\ 
\midrule 
TU40 & Viene testato che sia possibile modificare il periodo di scansione di un oggetto BeaconManagerAdapter & model::\-beacon::\-BeaconManagerAdapter::\-modifyScanPeriod() & Superato \\ 
\midrule 
TU41 & Viene testato che sia possibile accedere alle informazioni associate ad un oggetto InformationManagerImp. In particolare nel caso in cui nessuna lista di beacon sia disponibile oppure non sia stato ancora visto un beacon e quindi non sia possibile accedere alla mappa dell'edificio deve essere lanciata un'eccezione di tipo NoBeaconSeenException & model::\-InformationManagerImp::\-getBuildingMap() \par model::\-InformationManagerImp::\-getNearbyPOIs() \par model::\-InformationManagerImp::\-getAllVisibleBeacons() \par model::\-InformationManagerImp::\-getDatabaseService() & Superato \\ 
\midrule 
TU42 & Viene testato che sia possibile aggiungere e rimuovere un listener ad un oggetto NavigationManagerImp. In particolare viene anche testato che nel caso in cui sia registrato almeno un listener venga avvertito nel caso in cui venga settata una nuova lista di beacon & model::\-NavigationManagerImp::\-addListener() \par model::\-NavigationManagerImp::\-removeListener() & Superato \\ 
\midrule 
TU43 & Viene testato che sia possibile registrare e salvare informazioni sui beacon rilevati utilizzando un oggetto InformationManagerImp & model::\-InformationManagerImp::\-startRecordingBeacons() \par model::\-InformationManagerImp::\-saveRecordedBeaconInformation() & Superato \\ 
\midrule 
TU44 & Viene testato che sia possibile gestire la navigazione utilizzando un oggetto della classe NavigationManagerImp & model::\-NavigationManagerImp::\-startNavigation() \par model::\-NavigationManagerImp::\-getAllNavigationInstruction() \par model::\-NavigationManagerImp::\-getNextInstruction() \par model::\-NavigationManagerImp::\-stopNavigation() & Superato \\ 
\midrule 
TU45 & Viene testato che sia possibile creare e ritornare un oggetto BuildingService.  & model::\-dataaccess::\-service::\-ServiceHelper::\-getService() & Superato \\ 
\midrule 
TU46 & Viene testato che sia possibile eliminare una foto dal database locale, recuperarne una o tutte quelle riguardanti un Edge. & model::\-dataaccess::\-service::\-PhotoService::\-deletePhoto() \par model::\-dataaccess::\-service::\-PhotoService::\-findPhoto() \par model::\-dataaccess::\-service::\-PhotoService::\-findAllPhotosOfEdge() & Superato \\ 
\midrule 
TU47 & Viene testato che, dato un oggetto JsonObject che possiede gli stessi valori di un oggetto PhotoTable, sia possibile costruire un oggetto PhotoTable e inserirlo nel database locale. & model::\-dataaccess::\-service::\-PhotoService::\-convertAndInsert() & Superato \\ 
\midrule 
TU48 & Viene testato che sia possibile eliminare una RegionOfInterest dal database locale, recuperarne una o tutte quelle riguardanti un edificio, dato il major del suddetto edificio. & model::\-dataaccess::\-service::\-RegionOfInterestService::\-deleteRegionOfInterest() \par model::\-dataaccess::\-service::\-RegionOfInterestService::\-findRegionOfInterest() \par model::\-dataaccess::\-service::\-RegionOfInterestService::\-findAllRegionsWithMajor() & Superato \\ 
\midrule 
TU49 & Viene testato che, dato un oggetto JsonObject che possiede gli stessi valori di un oggetto RegionOfInterestTable, sia possibile costruire un oggetto RegionOfInterestTable e inserirlo nel database locale. & model::\-dataaccess::\-service::\-RegionOfInterestService::\-convertAndInsert() & Superato \\ 
\midrule 
TU50 & Viene testato che sia possibile eliminare un Edge dal database locale, recuperarne uno o tutti quelli riguardanti un edificio, dato il major del suddetto edificio. & model::\-dataaccess::\-service::\-EdgeService::\-deleteEdge() \par model::\-dataaccess::\-service::\-EdgeService::\-findEdge() \par model::\-dataaccess::\-service::\-EdgeService::\-findAllEdgesOfBuilding() & Superato \\ 
\midrule 
TU51 & Viene testato che, dato un oggetto JsonObject che possiede gli stessi valori di un oggetto EdgeTable, sia possibile costruire un oggetto EdgeTable e inserirlo nel database locale. & model::\-dataaccess::\-service::\-EdgeService::\-convertAndInsert() & Superato \\ 
\midrule 
TU52 & Viene testato che, dato un oggetto JsonObject che possiede gli stessi valori di un oggetto EdgeTypeTable, sia possibile costruire un oggetto EdgeTypeTable e inserirlo nel database locale. & model::\-dataaccess::\-service::\-EdgeService::\-convertAndInsertEdgeType() & Superato \\ 
\midrule 
TU53 & Viene testato che sia possibile eliminare un PointOfInterest dal database locale, recuperarne uno o tutti quelli riguardanti un edificio, dato il major del suddetto edificio. & model::\-dataaccess::\-service::\-PointOfInterestService::\-deletePointOfInterest() \par model::\-dataaccess::\-service::\-PointOfInterestService::\-findPointOfInterest() \par model::\-dataaccess::\-service::\-PointOfInterestService::\-findAllPointsWithMajor() & Superato \\ 
\midrule 
TU54 & Viene testato che, dato un oggetto JsonObject che possiede gli stessi valori di un oggetto PointOfInterestTable, sia possibile costruire un oggetto PointOfInterestTable e inserirlo nel database locale. & model::\-dataaccess::\-service::\-PointOfInterestService::\-convertAndInsert() & Superato \\ 
\midrule 
TU55 & Viene testato che, dato un oggetto JsonObject che possiede gli stessi valori di un oggetto CategoryTable, sia possibile costruire un oggetto CategoryTable e inserirlo nel database locale. & model::\-dataaccess::\-service::\-PointOfInterestService::\-convertAndInsertCategory() & Superato \\ 
\midrule 
TU56 & Viene testato che, dato un oggetto JsonObject che possiede gli stessi valori di un oggetto RoiPoiTable, sia possibile costruire un oggetto RoiPoiTable e inserirlo nel database locale. & model::\-dataaccess::\-service::\-PointOfInterestService::\-convertAndInsertRoiPoi() & Superato \\ 
\midrule 
TU57 & Viene testato che sia possibile eliminare una BuildingMap dal database locale, recuperarne una o tutte quelle presenti nel database locale. & model::\-dataaccess::\-service::\-BuildingService::\-deleteBuilding() \par model::\-dataaccess::\-service::\-BuildingService::\-findBuildingByMajor() \par model::\-dataaccess::\-service::\-BuildingService::\-findAllBuildings() & Superato \\ 
\midrule 
TU58 & Viene testato che sia possibile recuperare una BuildingMap dal database remoto o le informazioni di tutte quelle presenti nel database remoto. & model::\-dataaccess::\-service::\-BuildingService::\-findAllRemoteBuildings() \par model::\-dataaccess::\-service::\-BuildingService::\-findRemoteBuildingByMajor() & Superato \\ 
\midrule 
TU59 & Viene testato che, dato un oggetto JsonObject che possiede gli stessi valori di un oggetto BuildingTable, sia possibile costruire un oggetto BuildingTable e inserirlo nel database locale. & model::\-dataaccess::\-service::\-BuildingService::\-convertAndInsert() & Superato \\ 
\midrule 
TU60 & Viene testato che, dato il major di un edificio, sia possibile verificare la presenza della BuildingMap nel database locale, verificare se è aggiornata all'ultima versione disponibile e aggiornarla. & model::\-dataaccess::\-service::\-BuildingService::\-isBuildingMapPresent() \par model::\-dataaccess::\-service::\-BuildingService::\-isBuildingMapUpdated() \par model::\-dataaccess::\-service::\-BuildingService::\-updateBuildingMap() & Superato \\ 
\midrule 
TU61 & Viene testato che sia possibile creare e ritornare un oggetto SQLiteDaoFactory dato un oggetto SQLiteDatabase. Viene inoltre testato che sia possibile creare e ritornare un oggetto RemoteDaoFactory.  & model::\-dataaccess::\-dao::\-DaoFactoryHelper::\-getSQLiteDaoFactory() \par model::\-dataaccess::\-dao::\-DaoFactoryHelper::\-getRemoteDaoFactory() & Superato \\ 
\midrule 
TU62 & Viene testato che sia possibile creare e ritornare: un oggetto RemoteBuildingDao, un oggetto RemotePointOfInterestDao, un oggetto RemoteRegionOfInterestDao, un oggetto RemoteRoiPoiDao, un oggetto RemoteEdgeDao, un oggetto RemoteEdgeTypeDao, un oggetto RemoteCategoryDao, un oggetto RemotePhotoDao.  & model::\-dataaccess::\-dao::\-RemoteDaoFactory::\-getBuildingDao() \par model::\-dataaccess::\-dao::\-RemoteDaoFactory::\-getPointOfInterestDao() \par model::\-dataaccess::\-dao::\-RemoteDaoFactory::\-getRoiPoiDao() \par model::\-dataaccess::\-dao::\-RemoteDaoFactory::\-getEdgeDao() \par model::\-dataaccess::\-dao::\-RemoteDaoFactory::\-getCategoryDao() \par model::\-dataaccess::\-dao::\-RemoteDaoFactory::\-getEdgeTypeDao() \par model::\-dataaccess::\-dao::\-RemoteDaoFactory::\-getPhotoDao() \par model::\-dataaccess::\-dao::\-RemoteDaoFactory::\-getRegionOfInterestDao() & Superato \\ 
\midrule 
TU63 & Viene testato che, dato un oggetto JsonObject che possiede gli stessi valori di un oggetto BuildingTable, sia possibile costruire un oggetto BuildingTable e ritornarlo. & model::\-dataaccess::\-dao::\-RemoteBuildingDao::\-fromJSONToTable() & Superato \\ 
\midrule 
TU64 & Viene testato che, dato un oggetto JsonObject che possiede gli stessi valori di un oggetto PointOfInterestTable, sia possibile costruire un oggetto PointOfInterestTable e ritornarlo. & model::\-dataaccess::\-dao::\-RemotePointOfInterestDao::\-fromJSONToTable() & Superato \\ 
\midrule 
TU65 & Viene testato che, dato un oggetto JsonObject che possiede gli stessi valori di un oggetto RegionOfInterestTable, sia possibile costruire un oggetto RegionOfInterestTable e ritornarlo. & model::\-dataaccess::\-dao::\-RemoteRegionOfInterestDao::\-fromJSONToTable() & Superato \\ 
\midrule 
TU66 & Viene testato che, dato un oggetto JsonObject che possiede gli stessi valori di un oggetto RoiPoiTable, sia possibile costruire un oggetto RoiPoiTable e ritornarlo. & model::\-dataaccess::\-dao::\-RemoteRoiPoiDao::\-fromJSONToTable() & Superato \\ 
\midrule 
TU67 & Viene testato che, dato un oggetto JsonObject che possiede gli stessi valori di un oggetto EdgeTable, sia possibile costruire un oggetto EdgeTable e ritornarlo. & model::\-dataaccess::\-dao::\-RemoteEdgeDao::\-fromJSONToTable() & Superato \\ 
\midrule 
TU68 & Viene testato che, dato un oggetto JsonObject che possiede gli stessi valori di un oggetto EdgeTypeTable, sia possibile costruire un oggetto EdgeTypeTable e ritornarlo. & model::\-dataaccess::\-dao::\-RemoteEdgeTypeDao::\-fromJSONToTable() & Superato \\ 
\midrule 
TU69 & Viene testato che, dato un oggetto JsonObject che possiede gli stessi valori di un oggetto CategoryTable, sia possibile costruire un oggetto CategoryTable e ritornarlo. & model::\-dataaccess::\-dao::\-RemoteCategoryDao::\-fromJSONToTable() & Superato \\ 
\midrule 
TU70 & Viene testato che, dato un oggetto JsonObject che possiede gli stessi valori di un oggetto PhotoTable, sia possibile costruire un oggetto PhotoTable e ritornarlo. & model::\-dataaccess::\-dao::\-RemotePhotoDao::\-fromJSONToTable() & Superato \\ 
\midrule 
TU71 & Viene testato che sia possibile accedere a tutte le informazioni relative ad un oggetto BuildingTable. & model::\-dataaccess::\-dao::\-BuildingTable::\-getId() \par model::\-dataaccess::\-dao::\-BuildingTable::\-getUUID() \par model::\-dataaccess::\-dao::\-BuildingTable::\-getMajor() \par model::\-dataaccess::\-dao::\-BuildingTable::\-getVersion() \par model::\-dataaccess::\-dao::\-BuildingTable::\-getName() \par model::\-dataaccess::\-dao::\-BuildingTable::\-getDescription() \par model::\-dataaccess::\-dao::\-BuildingTable::\-getOpeningHours() \par model::\-dataaccess::\-dao::\-BuildingTable::\-getAddress() \par model::\-dataaccess::\-dao::\-BuildingTable::\-getSize() & Superato \\ 
\midrule 
TU72 & Viene testato che sia possibile accedere a tutte le informazioni relative ad un oggetto PointOfInterestTable. & model::\-dataaccess::\-dao::\-PointOfInterestTable::\-getId() \par model::\-dataaccess::\-dao::\-PointOfInterestTable::\-getName() \par model::\-dataaccess::\-dao::\-PointOfInterestTable::\-getDescription() \par model::\-dataaccess::\-dao::\-PointOfInterestTable::\-getCategoryId() & Superato \\ 
\midrule 
TU73 & Viene testato che sia possibile accedere a tutte le informazioni relative ad un oggetto RegionOfInterestTable. & model::\-dataaccess::\-dao::\-RegionOfInterestTable::\-getId() \par model::\-dataaccess::\-dao::\-RegionOfInterestTable::\-getUUID() \par model::\-dataaccess::\-dao::\-RegionOfInterestTable::\-getMajor() \par model::\-dataaccess::\-dao::\-RegionOfInterestTable::\-getMinor() & Superato \\ 
\midrule 
TU74 & Viene testato che sia possibile accedere a tutte le informazioni relative ad un oggetto RoiPoiTable. & model::\-dataaccess::\-dao::\-RoiPoiTable::\-getRoiID() \par model::\-dataaccess::\-dao::\-RoiPoiTable::\-getPoiID() & Superato \\ 
\midrule 
TU75 & Viene testato che sia possibile accedere a tutte le informazioni relative ad un oggetto EdgeTable. & model::\-dataaccess::\-dao::\-EdgeTable::\-getId() \par model::\-dataaccess::\-dao::\-EdgeTable::\-getStartROI() \par model::\-dataaccess::\-dao::\-EdgeTable::\-getEndROI() \par model::\-dataaccess::\-dao::\-EdgeTable::\-getDistance() \par model::\-dataaccess::\-dao::\-EdgeTable::\-getCoordinate() \par model::\-dataaccess::\-dao::\-EdgeTable::\-getTypeId() \par model::\-dataaccess::\-dao::\-EdgeTable::\-getAction() \par model::\-dataaccess::\-dao::\-EdgeTable::\-getLongDescription() & Superato \\ 
\midrule 
TU76 & Viene testato che sia possibile accedere a tutte le informazioni relative ad un oggetto EdgeTypeTable. & model::\-dataaccess::\-dao::\-EdgeTypeTable::\-getId() \par model::\-dataaccess::\-dao::\-EdgeTypeTable::\-getTypeName() & Superato \\ 
\midrule 
TU77 & Viene testato che sia possibile accedere a tutte le informazioni relative ad un oggetto CategoryTable. & model::\-dataaccess::\-dao::\-CategoryTable::\-getId() \par model::\-dataaccess::\-dao::\-CategoryTable::\-getDescription() & Superato \\ 
\midrule 
TU78 & Viene testato che sia possibile accedere a tutte le informazioni relative ad un oggetto PhotoTable. & model::\-dataaccess::\-dao::\-PhotoTable::\-getId() \par model::\-dataaccess::\-dao::\-PhotoTable::\-getUrl() \par model::\-dataaccess::\-dao::\-PhotoTable::\-getEdgeId() & Superato \\ 
\midrule 
TU79 & Viene testato che sia possibile creare e ritornare: un oggetto SQLiteBuildingDao, un oggetto SQLitePointOfInterestDao, un oggetto SQLiteRegionOfInterestDao, un oggetto SQLiteRoiPoiDao, un oggetto SQLiteEdgeDao, un oggetto SQLiteEdgeTypeDao, un oggetto SQLiteCategoryDao, un oggetto SQLitePhotoDao.  & model::\-dataaccess::\-dao::\-SQLiteDaoFactory::\-getBuildingDao() \par model::\-dataaccess::\-dao::\-SQLiteDaoFactory::\-getPointOfInterestDao() \par model::\-dataaccess::\-dao::\-SQLiteDaoFactory::\-getRegionOfInterestDao() \par model::\-dataaccess::\-dao::\-SQLiteDaoFactory::\-getRoiPoiDao() \par model::\-dataaccess::\-dao::\-SQLiteDaoFactory::\-getEdgeDao() \par model::\-dataaccess::\-dao::\-SQLiteDaoFactory::\-getCategoryDao() \par model::\-dataaccess::\-dao::\-SQLiteDaoFactory::\-getEdgeTypeDao() \par model::\-dataaccess::\-dao::\-SQLiteDaoFactory::\-getPhotoDao() & Superato \\ 
\midrule 
TU80 & Viene testato che sia possibile effettuare le operazioni CRUD sulla tabella "Building" del database locale. In particolare, nel caso della ricerca viene testato che sia possibile effettuarla sia tramite identificativo che tramite major dell'edificio e che sia possibile ottenere le informazioni di tutte le mappe presenti sul database locale. & model::\-dataaccess::\-dao::\-SQLiteBuildingDao::\-insertBuilding() \par model::\-dataaccess::\-dao::\-SQLiteBuildingDao::\-deleteBuilding() \par model::\-dataaccess::\-dao::\-SQLiteBuildingDao::\-findBuildingById() \par model::\-dataaccess::\-dao::\-SQLiteBuildingDao::\-findBuildingByMajor() \par model::\-dataaccess::\-dao::\-SQLiteBuildingDao::\-findAllBuildings() \par model::\-dataaccess::\-dao::\-SQLiteBuildingDao::\-updateBuilding() & Superato \\ 
\midrule 
TU81 & Viene testato che, dato un oggetto di tipo Cursor che rappresenta il risultato della query sul database locale, sia possibile creare un oggetto BuildingTable. & model::\-dataaccess::\-dao::\-SQLiteBuildingDao::\-cursorToType() & Superato \\ 
\midrule 
TU82 & Viene verificato che, dato il major di un edificio, sia possibile verificare se la sua mappa è presente nel database locale. & model::\-dataaccess::\-dao::\-SQLiteBuildingDao::\-isBuildingMapPresent() & Superato \\ 
\midrule 
TU83 & Viene testato che sia possibile effettuare le operazioni CRUD sulla tabella "POI" del database locale. In particolare, nel caso della ricerca viene testato che, dato il major di un edificio, sia possibile recuperare tutti gli oggetti PointOfInterestTable che rappresentano i PointOfInterest di quell'edificio. & model::\-dataaccess::\-dao::\-SQLitePointOfInterestDao::\-insertPointOfInterest() \par model::\-dataaccess::\-dao::\-SQLitePointOfInterestDao::\-deletePointOfInterest() \par model::\-dataaccess::\-dao::\-SQLitePointOfInterestDao::\-findPointOfInterest() \par model::\-dataaccess::\-dao::\-SQLitePointOfInterestDao::\-findAllPointsWithMajor() \par model::\-dataaccess::\-dao::\-SQLitePointOfInterestDao::\-updatePointOfInterest() & Superato \\ 
\midrule 
TU84 & Viene testato che, dato un oggetto di tipo Cursor che rappresenta il risultato della query sul database locale, sia possibile creare un oggetto PointOfInterestTable. & model::\-dataaccess::\-dao::\-SQLitePointOfInterestDao::\-cursorToType() & Superato \\ 
\midrule 
TU85 & Viene testato che sia possibile effettuare le operazioni CRUD sulla tabella "ROI" del database locale. In particolare, nel caso della ricerca viene testato che, dato il major di un edificio, sia possibile recuperare tutti gli oggetti RegionOfInterestTable che rappresentano le RegionOfinterest di quell'edificio. & model::\-dataaccess::\-dao::\-SQLiteRegionOfInterestDao::\-insertRegionOfInterest() \par model::\-dataaccess::\-dao::\-SQLiteRegionOfInterestDao::\-deleteRegionOfInterest() \par model::\-dataaccess::\-dao::\-SQLiteRegionOfInterestDao::\-findRegionOfInterest() \par model::\-dataaccess::\-dao::\-SQLiteRegionOfInterestDao::\-findAllRegionsWithMajor() \par model::\-dataaccess::\-dao::\-SQLiteRegionOfInterestDao::\-updateRegionOfInterest() & Superato \\ 
\midrule 
TU86 & Viene testato che, dato un oggetto di tipo Cursor che rappresenta il risultato della query sul database locale, sia possibile creare un oggetto RegionOfInterestTable. & model::\-dataaccess::\-dao::\-SQLiteRegionOfInterestDao::\-cursorToType() & Superato \\ 
\midrule 
TU87 & Viene testato che sia possibile effettuare le operazioni CRUD sulla tabella "ROIPOI" del database locale. In particolare, nel caso della ricerca o della rimozione viene testato che, dato l'identificativo di un PointOfInterest, sia possibile recuperare o eliminare tutti gli oggetti RegionOfInterestTable che rappresentano tutte le RegionOfInterest che lo contengono e viceversa.  & model::\-dataaccess::\-dao::\-SQLiteRoiPoiDao::\-insertRoiPoi() \par model::\-dataaccess::\-dao::\-SQLiteRoiPoiDao::\-deleteRoiPoisWherePoi() \par model::\-dataaccess::\-dao::\-SQLiteRoiPoiDao::\-deleteRoiPoisWhereRoi() \par model::\-dataaccess::\-dao::\-SQLiteRoiPoiDao::\-findAllRegionsWithPoi() \par model::\-dataaccess::\-dao::\-SQLiteRoiPoiDao::\-findAllPointsWithRoi() \par model::\-dataaccess::\-dao::\-SQLiteRoiPoiDao::\-updateRoiPoi() & Superato \\ 
\midrule 
TU88 & Viene testato che, dato un oggetto di tipo Cursor che rappresenta il risultato della query sul database locale, sia possibile creare un oggetto RoiPoiTable. & model::\-dataaccess::\-dao::\-SQLiteRoiPoiDao::\-cursorToType() & Superato \\ 
\midrule 
TU89 & Viene testato che sia possibile effettuare le operazioni CRUD sulla tabella "Edge" del database locale. In particolare, nel caso della ricerca viene testato che, dato il major di un edificio, sia possibile recuperare tutti gli oggetti EdgeTable che rappresentano gli Edge di quell'edificio. & model::\-dataaccess::\-dao::\-SQLiteEdgeDao::\-insertEdge() \par model::\-dataaccess::\-dao::\-SQLiteEdgeDao::\-deleteEdge() \par model::\-dataaccess::\-dao::\-SQLiteEdgeDao::\-findEdge() \par model::\-dataaccess::\-dao::\-SQLiteEdgeDao::\-findAllEdgesOfBuilding() \par model::\-dataaccess::\-dao::\-SQLiteEdgeDao::\-updateEdge() & Superato \\ 
\midrule 
TU90 & Viene testato che, dato un oggetto di tipo Cursor che rappresenta il risultato della query sul database locale, sia possibile creare un oggetto EdgeTable. & model::\-dataaccess::\-dao::\-SQLiteEdgeDao::\-cursorToType() & Superato \\ 
\midrule 
TU91 & Viene testato che sia possibile effettuare le operazioni CRUD sulla tabella "Category" del database locale. & model::\-dataaccess::\-dao::\-SQLiteCategoryDao::\-insertCategory() \par model::\-dataaccess::\-dao::\-SQLiteCategoryDao::\-deleteCategory() \par model::\-dataaccess::\-dao::\-SQLiteCategoryDao::\-findCategory() \par model::\-dataaccess::\-dao::\-SQLiteCategoryDao::\-updateCategory() & Superato \\ 
\midrule 
TU92 & Viene testato che, dato un oggetto di tipo Cursor che rappresenta il risultato della query sul database locale, sia possibile creare un oggetto CategoryTable. & model::\-dataaccess::\-dao::\-SQLiteCategoryDao::\-cursorToType() & Superato \\ 
\midrule 
TU93 & Viene testato che sia possibile effettuare le operazioni CRUD sulla tabella "EdgeType" del database locale. & model::\-dataaccess::\-dao::\-SQLiteEdgeTypeDao::\-insertEdgeType() \par model::\-dataaccess::\-dao::\-SQLiteEdgeTypeDao::\-deleteEdgeType() \par model::\-dataaccess::\-dao::\-SQLiteEdgeTypeDao::\-findEdgeType() \par model::\-dataaccess::\-dao::\-SQLiteEdgeTypeDao::\-updateEdgeType() & Superato \\ 
\midrule 
TU94 & Viene testato che, dato un oggetto di tipo Cursor che rappresenta il risultato della query sul database locale, sia possibile creare un oggetto EdgeTypeTable. & model::\-dataaccess::\-dao::\-SQLiteEdgeTypeDao::\-cursorToType() & Superato \\ 
\midrule 
TU95 & Viene testato che sia possibile effettuare le operazioni CRUD sulla tabella "Photo" del database locale. In particolare, nel caso della ricerca viene testato che, dato l'identificativo di un Edge, sia possibile recuperare tutti gli oggetti PhotoTable che rappresentano le foto di quell'Edge. & model::\-dataaccess::\-dao::\-SQLitePhotoDao::\-insertPhoto() \par model::\-dataaccess::\-dao::\-SQLitePhotoDao::\-deletePhoto() \par model::\-dataaccess::\-dao::\-SQLitePhotoDao::\-findPhoto() \par model::\-dataaccess::\-dao::\-SQLitePhotoDao::\-findAllPhotosOfEdge() \par model::\-dataaccess::\-dao::\-SQLitePhotoDao::\-updatePhoto() & Superato \\ 
\midrule 
TU96 & Viene testato che, dato un oggetto di tipo Cursor che rappresenta il risultato della query sul database locale, sia possibile creare un oggetto PhotoTable. & model::\-dataaccess::\-dao::\-SQLitePhotoDao::\-cursorToType() & Superato \\ 
\midrule 
TU97 & Viene testato che sia possibile effettuare le operazioni CRUD sull'intero database locale. & model::\-dataaccess::\-dao::\-SQLDao::\-insert() \par model::\-dataaccess::\-dao::\-SQLDao::\-delete() \par model::\-dataaccess::\-dao::\-SQLDao::\-query() \par model::\-dataaccess::\-dao::\-SQLDao::\-update() \par model::\-dataaccess::\-dao::\-SQLDao::\-rawQuery() & Superato \\ 
\midrule 
TU98 & Viene testato che vengano creati il database e le sue tabelle e che venga effettuata la popolazione iniziale delle tabelle. & model::\-dataaccess::\-dao::\-MapsDbHelper::\-onCreate() & Superato \\ 
\midrule 
TU99 & Viene testato che venga aggiornato il database in seguito all'aggiunta o alla rimozione di una tabella. & model::\-dataaccess::\-dao::\-MapsDbHelper::\-onUpgrade() & Superato \\ 
\midrule 
TU100 & Viene testato che sia possibile recuperare l'URL del database remoto. & model::\-dataaccess::\-dao::\-MapsDbHelper::\-getRemoteDatabaseURL() & Superato \\ 
\midrule 
TU101 & Viene testato che Compass sia effettivamente un in ascolto dei sensori & model::\-compass::\-Compass::\-registerListener() & Superato \\ 
\midrule 
TU102 & Viene testato che Compass possa terminare l'ascolto dei sensori & model::\-compass::\-Compass::\-unregisterListener() & Superato \\ 
\midrule 
TU103 & Viene testato che il valore lastCoordinate nel tempo non cambi con i sensori spenti e cambi con i sensori attivi & model::\-compass::\-Compass::\-getLastCoordinate() & Superato \\ 
\midrule 
TU104 & Viene verificato che sia possibile recuperare le informazioni di un ContentProvider per effettuare una ricerca utilizzando la classe SearchSuggestionProvider & presenter::\-SearchSuggestionsProvider::\-query() \par presenter::\-SearchSuggestionsProvider::\-getType() \par presenter::\-SearchSuggestionsProvider::\-insert() \par presenter::\-SearchSuggestionsProvider::\-update() \par presenter::\-SearchSuggestionsProvider::\-onCreate() & N.I. \\ 
\midrule 
TU105 & Viene verificato che sia possibile recuperare il nome dell'edificio da InformationManager e che tale nome sia passato ad HomeView & presenter::\-HomeActivity::\-updateBuildingName() & Superato \\ 
\midrule 
TU106 & Viene verificato che sia possibile recuperare la descrizione dell'edificio da InformationManager e che tale descrizione sia passata ad HomeView & presenter::\-HomeActivity::\-updateBuildingDescription() & Superato \\ 
\midrule 
TU107 & Viene verificato che sia possibile recuperare le ore di apertura dell'edificio da InformationManager e che tale orario sia passato ad HomeView & presenter::\-HomeActivity::\-updateBuildingOpeningHours() & Superato \\ 
\midrule 
TU108 & Viene verificato che sia possibile recuperare le categorie di POI dell'edificio da InformationManager e che tali categorie siano passate ad HomeView & presenter::\-HomeActivity::\-updatePoiCategoryList() & Superato \\ 
\midrule 
TU109 & Viene verificato che sia possibile recuperare l'indirizzo dell'edificio da InformationManager e che tale indirizzo sia passato ad HomeView & presenter::\-HomeActivity::\-updateBuildingAddress() & Superato \\ 
\midrule 
TU110 & Viene verificato che sia possibile recuperare i nomi dei POI dell'edificio da InformationManager & presenter::\-HomeActivity::\-enableSuggestions() & Superato \\ 
\midrule 
TU111 & Viene verificato che sia possibile mostrare le categorie di POI & presenter::\-HomeActivity::\-showPoisCategory() & Superato \\ 
\midrule 
TU112 & Viene verificato che sia possibile mostrare le prefrenze utente & presenter::\-HomeActivity::\-showPreferences() & Superato \\ 
\midrule 
TU113 & Viene verificato che sia possibile mostrare la guida dell'applicazione & presenter::\-HomeActivity::\-showHelp() & Superato \\ 
\midrule 
TU114 & Viene verificato che sia possibile avviare la navigazione & presenter::\-HomeActivity::\-startNavigation() & Superato \\ 
\midrule 
TU115 & Viene verificato che sia possibile mostrare le mappe salvate nel databese locale & presenter::\-HomeActivity::\-showLocalMaps() & Superato \\ 
\midrule 
TU116 & Viene verificato che sia possibile far partire la navigazione utilizzando l'identificativo di un POI appartenente ad una certa categoria & presenter::\-PoiCategoryActivity::\-startNavigation() & Superato \\ 
\midrule 
TU117 & Viene verificato che sia possibile gestire le informazioni di navigazione & presenter::\-NavigationActivity::\-pathError() \par presenter::\-NavigationActivity::\-informationUpdate() & Superato \\ 
\midrule 
TU118 & Viiene verificato che sia possibile visualizzare le informazioni dettagliate di navigazione & presenter::\-NavigationActivity::\-showDetailedInformation() & Superato \\ 
\midrule 
TU119 & Viene verificato che sia possibile gestire un'immagine utilizzando la classe ImageAdapter & presenter::\-ImageAdapter::\-getCount() \par presenter::\-ImageAdapter::\-getItem() \par presenter::\-ImageAdapter::\-getView() & Superato \\ 
\midrule 
TU120 & Viene verificato che sia possibile gestire un insieme di immagini utilizzando la classe ImageListFragment & presenter::\-ImageListFragment::\-onItemClick() \par presenter::\-ImageListFragment::\-newInstance() \par presenter::\-ImageListFragment::\-onCreateView() & Superato \\ 
\midrule 
TU121 & Viene verificato che sia possibile gestire le opzioni sviluppatore utilizzando la classe MainDeveloperPresenter & presenter::\-MainDeveloperPresenter::\-isDeveloper() \par presenter::\-MainDeveloperPresenter::\-startDeveloperUnlocker() \par presenter::\-MainDeveloperPresenter::\-startDeveloperOptions() & Superato \\ 
\midrule 
TU122 & Viene verificato che sia possibile gestire lo sblocco delle opzioni sviluppatore utilizzando la classe DeveloperUnlockerActivity & presenter::\-DeveloperUnlockerActivity::\-unlockDeveloper() & Superato \\ 
\midrule 
TU123 & Viene verificato che sia possibile avviare un nuovo log e accedere ai log salvati sul dispositivo utilizzando la classe MainDeveloperActivity & presenter::\-MainDeveloperActivity::\-showDetailedLog() \par presenter::\-MainDeveloperActivity::\-startNewLog() & Superato \\ 
\midrule 
TU124 & Viene verificato che sia possibile salvare le preferenze utente utilizzando la classe PreferencesActivity & presenter::\-PreferencesActivity::\-savePreferences() & N.I. \\ 
\midrule 
TU125 & Viene verificato che vengano visualizzate le mappe locali terminato il download di una mappa utilizzando la classe MapDownloaderActivity & presenter::\-MapDownloaderActivity::\-downloadFinished() & N.I. \\ 
\midrule 
TU126 & Viene verificato che sia possibile effettuare il download di una mappa utilizzando la classe Remote MapManager & presenter::\-RemoteMapManagerActivity::\-downloadMap() & N.I. \\ 
\midrule 
TU127 & Viene verificato che sia possibile aggiornare o rimuovere una mappa già presente sul dispostivo utilizzando la classe LocalMapActvitiy utilizzando LocalMapActivity & presenter::\-LocalMapActivity::\-updateMap() \par presenter::\-LocalMapActivity::\-deleteMap() & N.I. \\ 
\midrule 
TU128 & Viene verificato che venga visualizzata la guida dell'applicativo utilizzando HelpActivity & presenter::\-HelpActivity::\-onCreate() & N.I. \\ 
\midrule 
TU129 & Viene verificato che venga visualizzata la foto selezionata dall'utente utizzando ImageDetailActivity & presenter::\-ImageDetailActivity::\-onCreate() & Superato \\ 
\midrule 
TU130 & Viene verificato che venga visualizzata la lista di POI rilevati dal dispositivo utilizzando la classe NearbyPoiActivity & presenter::\-NearbyPoiActivity::\-onCreate() & Superato \\ 
\midrule 
TU131 & Viene verificato che sia possibile gestire le foto e la descrizione dettagliata relativa ad una certa istruzione utilizzando la classe DetailedInformationActivity & presenter::\-DetailedInformationActivity::\-updatePhoto() \par presenter::\-DetailedInformationActivity::\-updateDetailedDescription() & Superato \\ 
\midrule 
TU132 & Viene verificato che sia possibile gestire la lista di immagini di un certo POI utilizzando la classe ImageAdapter & presenter::\-ImageAdapter::\-getCount() \par presenter::\-ImageAdapter::\-getItem() \par presenter::\-ImageAdapter::\-getItemId() \par presenter::\-ImageAdapter::\-getView() & Superato \\ 
\midrule 
TU133 & Viene verificato che venga visualizzata la schermata iniziale dell'applicazione utilizzando la classe MainActivity & presenter::\-MainActivity::\-onCreate() & Superato \\ 
\midrule 
TU134 & Viene verificato che venga arrestata l'attività di log e che venga salvato il log utilizzando la classe LogginActivity & presenter::\-LoggingActivity::\-stopLogging() & Superato \\ 
\midrule 
TU135 & Viene verificato che sia possibile eliminare un log salvato utilizzando la classe LogInformationActivity & presenter::\-LogInformationActivity::\-deleteLog() & Superato \\ 
\midrule 
TU136 & Viene verificato che sia possibile gesitre la lista di indicazioni utili per raggiungere una certa destinazione utilizzando la classe NavigationAdapter & presenter::\-NavigationAdapter::\-getCount() \par presenter::\-NavigationAdapter::\-getItem() \par presenter::\-NavigationAdapter::\-getItemId() \par presenter::\-NavigationAdapter::\-getView() & Superato \\ 
\midrule 
TU137 & Viene verificato che vengano visualizzate le informazioni di un edificio & view::\-HomeViewImp::\-setBuildingName() \par view::\-HomeViewImp::\-setBuildingOpeningHours() \par view::\-HomeViewImp::\-setPoiCategoryListAdapter() \par view::\-HomeViewImp::\-setBuildingAddress() & Superato \\ 
\midrule 
TU138 & Viene verificato che sia possibile visualizzare le istruzioni di navigazione & view::\-NavigationViewImp::\-setInstructionAdapter() \par view::\-NavigationViewImp::\-refreshInstructions() & Superato \\ 
\midrule 
TU139 & Viene verificato che vengano visualizzate le categorie dei POI & view::\-PoiCategoryViewImp::\-setPoiListAdapter() & Superato \\ 
\midrule 
TU140 & Viene verificato che vengano visualizzate le indicazioni dettagliate e le foto associate ad un arco & view::\-DetailedInformationViewImp::\-setPhoto() \par view::\-DetailedInformationViewImp::\-setDetailedDescription() & Superato \\ 
\midrule 
TU141 & Viene verificato che vengano visualizzati tutti i POI associati ad un certo ROI & view::\-NearbyPoiViewImp::\-setAdapter() & Superato \\ 
\midrule 
TU142 & Viene verificato che sia possibile visualizzare una immagine & view::\-ImageDetailViewImp::\-setAdapter() & Superato \\ 
\midrule 
TU143 & Viene verificato che sia possibile visualizzare la guida & view::\-HelpViewImp::\-setHelp() & N.I. \\ 
\midrule 
TU144 & Viene verificato che siano visualizzate le impostazioni per gestire le mappe locali & view::\-LocalMapManagerViewImp::\-refreshMaps() \par view::\-LocalMapManagerViewImp::\-setAdapter() & N.I. \\ 
\midrule 
TU145 & Viene verificato che siano visualizzate le impostazioni per gestire le mappe remote & view::\-RemoteMapManagerViewImp::\-setRemoteMaps() & N.I. \\ 
\midrule 
TU146 & Viene verificato che sia visualizzata la schermata di dowload di una mappa & view::\-MapDownloaderView::\-setDowloadingMap() \par view::\-MapDownloaderView::\-setProgressDowload() & N.I. \\ 
\midrule 
TU147 & Viene verificato che siano visualizzate le impostazioni per gestire le preferenze riguardanti l'applicazione & view::\-PreferencesViewImp::\-setPathPreferences() \par view::\-PreferencesViewImp::\-setInstructionPreferences() & N.I. \\ 
\midrule 
TU148 & Viene verificato che sia possibile inserire un codice per sbloccare le funzionalità sviluppatore e visualizzare un messaggio di errore in caso di codice errato & view::\-DeveloperUnlockerViewImp::\-showWrongCode() & Superato \\ 
\midrule 
TU149 & Viene verificato che vengano visualizzati i log salvati & view::\-MainDeveloperViewImp::\-setLogsAdapter() & Superato \\ 
\midrule 
TU150 & Viene verificato che vengano visualizzati gli identificati dei beacon circostanti & view::\-LoggingViewImp::\-setBeaconListAdapter() & Superato \\ 
\midrule 
TU151 & Viene verificato che vengano visualizzato il dettaglio di un log e che questo possa essere eliminato & view::\-LogInformationViewImp::\-setBeaconAdapter() & Superato \\ 
\midrule 
TU153 & Viene testato che sia possibile creare e ritornare un'istanza di DaoFactoryHelper. & model::\-dataaccess::\-dao::\-DaoFactoryHelper::\-getInstance() & Superato \\ 
\midrule 
TU154 & Viene testato che, dato un oggetto JsonObject che possiede gli stessi valori di un oggetto PhotoTable, sia possibile costruire un oggetto PhotoTable e inserirlo nel database locale. & model::\-dataaccess::\-service::\-EdgeService::\-convertAndInsertPhoto() & Superato \\ 
\midrule 
TU155 & Viene testato che tramite un oggetto InformationManagerImp venga lanciata un'eccezione di tipo NoBeaconSeenExcetpion nel caso in cui si cerchi di recuperare la mappa di un edificio o i POI circostanti senza aver rilevato alcun beacon & model::\-InformationManagerImp::\-getBuildingMap() \par model::\-InformationManagerImp::\-getNearbyPOIs() & Superato \\ 
\midrule 
TU156 & Viene testato che sia possibile aggiungere e rimuovere un listener ad un oggetto InformationManagerImp. In particolare viene anche testato che nel caso in cui sia registrato almeno un listener venga avvertito nel caso in cui siano disponiobili informazioni o ci siano problemi nel reperire informazioni & model::\-InformationManagerImp::\-removeListener() \par model::\-InformationManagerImp::\-addListener() & Superato \\ 
\midrule 
TU157 & Viene testato che sia possibile recuperare le RegionOfInterest che sono già state associate ai PointOfInterest vicini & model::\-dataaccess::\-service::\-RegionOfInterestService::\-getTracedRois() & Superato \\ 
\midrule 
TU158 & Viene testato che sia possibile impostare le RegionOfInterest che sono già state associate i PointOfInterest vicini & model::\-dataaccess::\-service::\-RegionOfInterestService::\-setTracedRois() & Superato \\ 
\midrule 
TU159 & Viene testato che sia possibile recuperare gli identificativi di tutti i PointOfInterest associati ad una specifica RegionOfInterest & model::\-dataaccess::\-service::\-RegionOfInterestService::\-findAllPointsWithRoi() & Superato \\ 
\midrule 
TU160 & Viene testato che sia possibile recuperare gli identificativi di tutte le RegionOfInterest associate ad uno specifico PointOfInterest & model::\-dataaccess::\-service::\-PointOfInterestService::\-findAllRegionsWithPoi() & Superato \\ 
\midrule 
TU161 & Viene testato che sia possibile verificare la presenza sul database remoto della mappa di un edificio & model::\-dataaccess::\-service::\-BuildingService::\-isRemoteMapPresent() & Superato \\ 
\arrayrulecolor{black}
\bottomrule
\caption{Tabella descrizione test unità} \\
\end{longtabu}
\end{appendices}
\end{document}