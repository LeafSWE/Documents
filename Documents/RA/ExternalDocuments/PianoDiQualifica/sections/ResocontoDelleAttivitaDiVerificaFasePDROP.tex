\documentclass[../PianoDiQualifica.tex]{subfiles}
\definecolor{dkgreen}{RGB}{0,160,0}


\begin{document}
\begin{appendices}
\section{Resoconto delle attività di verifica - fase PDROP}
All'interno di questa fase\g, secondo quanto riportato nel documento \pianodiprogetto, sono previsti più momenti in cui viene attivato il processo\g\ di verifica. Si è cercato di riportare in questa sezione tutti i risultati che sono stati ottenuti durante questi momenti. Ove fosse necessario, si sono tratte anche delle conclusioni sui risultati ottenuti e su come essi possono essere migliorati.\\
Per una maggiore comprensione, i risultati ottenuti dalle metriche verranno evidenziati nel seguente modo: 
\begin{itemize} 
	\item {\color{red}Negativo} 
	\item Accettabile 
	\item {\color{dkgreen}Ottimale}
\end{itemize}  
	
	\subsection{Verifica sui processi}
		\subsubsection{Processo di documentazione}
			\paragraph{Miglioramento costante}
			All'inizio della fase\g\ PDROP il processo\g\ di documentazione si posizionava al livello 3 della scala CMM\g.
			Il gruppo durante questa fase ha mantenuto il livello 3 perchè non ci sono stati importanti miglioramenti al processo e perchè la quantità di documenti prodotta non raggiungeva un volume considerevole.
			
			\paragraph{Rispetto della pianificazione}
			In questa fase tutti gli incrementi ai documenti hanno rispettato la pianificazione prevista. Questo risultato positivo è dovuto al fatto che la suddivisione dei compiti e delle attività è avvenuta in maniera consona.\\ 
			Per una visione generale dei risultati si osservi la colonna \textbf{MPC2}  della tabella \ref{tab:esiti_metriche_per_il_processo_di_documentazione} e della tabella \ref{tab:esiti_metriche_processi}.
			
			\paragraph{Rispetto del budget}
			Dato il risultato positivo della pianificazione, i costi sono stati contenuti  e non si sono avute perdite.\\
			Per una visione generale dei risultati si osservi la colonna \textbf{MPC3}  della tabella \ref{tab:esiti_metriche_per_il_processo_di_documentazione} e della tabella \ref{tab:esiti_metriche_processi}.
			
			\paragraph{Riepilogo}\
			\begin{table}[H]
				\centering
				\begin{tabular}{l * {5}{c}}
					\toprule
					\textbf{Documento} & \textbf{MPC2} & \textbf{MPC3}\\
					\midrule
					\textit{Piano di progetto v6.00}  & \color{dkgreen}{0\%} & \color{dkgreen}{0\%} \\
					\textit{Norme di progetto v6.00}  & \color{dkgreen}{0\%} & \color{dkgreen}{0\%} \\
					\textit{Analisi dei requisiti v6.00}  & \color{dkgreen}{0\%} & \color{dkgreen}{0\%} \\
					\textit{Piano di qualifica v6.00}  & \color{dkgreen}{0\%} & \color{dkgreen}{0\%} \\
					\textit{Glossario v6.00}  & \color{dkgreen}{0\%} & \color{dkgreen}{0\%} \\
					\textit{Definizione di prodotto v3.00}  & \color{dkgreen}{0\%} & \color{dkgreen}{0\%} \\
					\textit{Manuale utente v2.00}  & \color{dkgreen}{0\%} & \color{dkgreen}{0\%} \\
					\textit{Manuale sviluppatore v2.00}  & \color{dkgreen}{0\%} & \color{dkgreen}{0\%} \\
					\textit{Totale} & \color{dkgreen}{0\%} & \color{dkgreen}{0\%} \\
					\bottomrule
				\end{tabular}
				\caption{Esiti delle metriche per il processo di documentazione}
				\label{tab:esiti_metriche_per_il_processo_di_documentazione}
			\end{table}
						
		\subsubsection{Processo di verifica}
			\paragraph{Miglioramento costante}
			Il gruppo non ha rilevato miglioramenti tali da raggiungere il quarto livello CMM\g, pertanto il processo\g\ di verifica rimane al terzo livello (Defined).
			
			\paragraph{Rispetto della pianificazione}
			Il processo di verifica ha rispettato la pianificazione perchè i processi da cui dipendeva non hanno subito ritardi e il processo stesso ha seguito il suo normale corso.
			Per una visione generale dei risultati si osservi la colonna \textbf{MPC2}  della tabella \ref{tab:esiti_metriche_processi}.
			
			\paragraph{Rispetto del budget}
			Per quanto riguarda il processo di verifica i costi sono stati contenuti e hanno rispettato quelli preventivati.\\
			Per una visione generale dei risultati si osservi la colonna \textbf{MPC3}  della tabella \ref{tab:esiti_metriche_processi}.
	
		\subsubsection{Processo di codifica}
			\paragraph{Miglioramento costante}
			Il livello CMM\g del processo di codifica è ancora uguale a 2.Il gruppo si è impegnato a rendere il processo proattivo ma non ha raggiunto i risultati sperati. Il processo quindi soffre ancora di qualche mancanza.\\
			
			\paragraph{Rispetto della pianificazione}
			Nonostante i problemi riscontrati il processo è riuscito a rispettare la pianificazione.
			Per una visione generale dei risultati si osservi la colonna \textbf{MPC2}  della tabella \ref{tab:esiti_metriche_processi}.
			
			\paragraph{Rispetto del budget}
			Il processo di codifica rispetto ai risultati precedenti è riuscito ad allinearsi ai costi preventivati e ad non andare in perdita.\\
			Per una visione generale dei risultati si osservi la colonna \textbf{MPC3}  della tabella \ref{tab:esiti_metriche_processi}.
			
			\subsubsection{Riepilogo}
			
			\begin{table}[H]
				\centering
				\begin{tabular}{l * {4}{c}}
					\toprule
					\textbf{Processo} & \textbf{MPC1} & \textbf{MPC2} & \textbf{MPC3} \\
					\midrule
					\textit{Processo di documentazione} & 3 & \color{dkgreen}{0\%} &  \color{dkgreen}{0\%} \\
					\textit{Processo di verifica} & 3 & \color{dkgreen}{0\%} &  \color{dkgreen}{0\%} \\
					\textit{Processo di codifica} & 2 & \color{dkgreen}{0\%} &  \color{dkgreen}{0\%} \\
					\bottomrule
				\end{tabular}
				\caption{Esiti delle metriche sui processi per la fase PDROP}
				\label{tab:esiti_metriche_processi}
			\end{table}
			
			
	\subsection{Verifica sui prodotti}
	In questa sezione verranno riportati i dati emessi dalle procedure di controllo della qualità di prodotto\g.
	
		\subsubsection{Documenti}
		In questa sezione vengono riportati gli esiti delle attività di verifica svolte sui documenti.\\
		Tali esiti sono strettamente correlati agli obiettivi di qualità dei documenti enunciati alla sezione \ref{ObiettiviDiQualità} del presente documento.
			
			\paragraph{Leggibilità e comprensibilità}
			I documenti in media hanno un indice di leggibilità accettabile. I documenti che risultano avere un indice di leggibilità ottimale sono il \glossario, il \manualeutente\ e il \manualesviluppatore.
			La media dei risultati ottenuti è presente alla colonna \textbf{MPRD1} della tabella \ref{tab:esiti_metriche_sui_documenti}. 
			
			\paragraph{Correttezza ortografica}
			In seguito al controllo ortografico automatico eseguito sui documenti sono stati rilevati 6 errori ortografici. Questi sono stati successivamente corretti avendo così un risultato \textbf{ottimale} per la metrica.
			Il risultato è riportato è presente alla colonna \textbf{MPRD2} della tabella \ref{tab:esiti_metriche_sui_documenti}.  
			
			
			\paragraph{Correttezza concettuale}
			 Durante la verifica manuale sono stati rilevati 3 errori concettuali. Essi sono stati tutti corretti raggiungendo così l'obiettivo \textbf{ottimale}.\\
			 Il risultato è presente alla colonna \textbf{MPRD3} della tabella \ref{tab:esiti_metriche_sui_documenti}.  
			
			\paragraph{Riepilogo}\
			\begin{table}[H]
				\centering
				\begin{tabular}{l * {4}{c}}
					\toprule
					  & \textbf{MPRD1} & \textbf{MPRD2} & \textbf{MPRD3}\\
					\midrule
					\textit{Totale documenti} & 58,15 & \color{dkgreen}{0\%} & \color{dkgreen}{0\%} \\
					\bottomrule
				\end{tabular}
				\caption{Esiti delle metriche sui documenti per la fase PDROP}
				\label{tab:esiti_metriche_sui_documenti}
			\end{table}
			
			
		\subsubsection{Software}
		In questa sezione vengono riportati gli esiti delle attività di verifica svolte sul software.\\
		Tali esiti sono strettamente correlati agli obiettivi di qualità dei documenti enunciati alla sezione \ref{ObiettiviDiQualità} del presente documento.
		
			\paragraph{Funzionalità obbligatorie}
			I requisiti obbligatori non sono stati completamente soddisfatti. Tra questi vi sono quelli di qualità che richiedono la stesura della documentazione riguardante le prove sperimentali. Per quanto riguarda invece i requisiti funzionali ne rimane ancora uno non soddisfatto.\\
			I risultati sono presenti nella tabella \ref{tab:esiti_metriche_sul_software}.
			
			\paragraph{Funzionalità desiderabili}
			Sono stati implementati circa i tre quarti dei requisiti desiderabili. Il valore risulta ancora negativo, però le tendenze risultano positive e quindi si ha una buona probabilità che tutti i requisiti desiderabili vengano soddisfatti.\\
			I risultati sono presenti nella tabella \ref{tab:esiti_metriche_sul_software}.
			
			\paragraph{Manutenibilità e Comprensibilità del codice}
			La qualità del codice secondo le metriche prese in considerazione risulta essere relativamente buona. I risultati più critici derivano dalla metrica che valuta il grado di accoppiamento. Da questo si deduce che le classi siano molto accoppiate tra di loro. Questo indica che bisognerà tenere sotto controllo questo parametro.\\
			I risultati sono presenti nella tabella \ref{tab:esiti_metriche_sul_software}.
			
			\paragraph{Copertura dei test richiesti}
			I test implementati e superati sono circa la metà di quelli previsti. L'esito quindi risulta ancora negativo. Le previsioni basate sulla tendenza risultano essere positive.  
			I risultati sono presenti nella tabella \ref{tab:esiti_metriche_sul_software}.
			
			\paragraph{Robustezza}
			In seguito ai vari stress test eseguiti, possiamo affermare che il prodotto riesce ad evitare nel 80\% dei casi un'interruzione critica. L'esito risulta pertanto accettabile.
			I risultati sono presenti nella tabella \ref{tab:esiti_metriche_sul_software}.
			
			\paragraph{Funzionamento senza interruzioni}	
			Il prodotto è in grado di gestire nel 80\% dei casi delle situazioni anomale e garantire il continuo e corretto funzionamento. Il valore riscontrato risulta pertanto accettabile.
			I risultati sono presenti nella tabella \ref{tab:esiti_metriche_sul_software}.
			
			\subsubsection{Riepilogo}
			\begin{table}[H]
				\centering
				\begin{tabular}{l * {4}{c}}
					\toprule
					\textbf{Metrica} & \textbf{Valore} & \textbf{Valore medio} & \textbf{Eccedenze (\%) }\\
					\midrule
					\textit{Copertura Requisti Obbligatori}  & \color{red}{77,67\%} & & \\
					\textit{Copertura Requisti Desiderabili}  & \color{red}{76,92\%} & & \\
					\textit{Test Passati Richiesti} & \color{red}{46,06\%} & & \\
					\textit{Numero di statement per metodo} & 53 & \color{dkgreen}{4,25} & 0\% \\
					\textit{Numero di parametri per metodo} & 9 & \color{dkgreen}{0,80} & 0\% \\
					\textit{Numero di campi dati per classe} & 10 & \color{dkgreen}{2,65} & 0\% \\
					\textit{Grado di accoppiamento}  & & \color{red}{10,11} & 37,6\% \\
					\textit{Cyclomatic number} & & \color{dkgreen}{1,69} & 0,75\% \\
					\textit{Adequacy of variable names}  & \color{dkgreen}{100\%} & & \\
					\textit{Average Module Size} & & \color{dkgreen}{19,84} & 13\% \\
					\textit{Breakdown Avoidance}  & \color{dkgreen}{80\%} & & \\
					\textit{Failure Avoidance}  & \color{dkgreen}{80\%} & & \\
					\bottomrule
				\end{tabular}
				\caption{Esiti delle metriche sul software per la fase PDROP}
				\label{tab:esiti_metriche_sul_software}
			\end{table}	
\end{appendices}
\end{document}